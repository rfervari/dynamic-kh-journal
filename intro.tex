Over the last years, a new family of epistemic languages for reasoning about \emph{knowing how} assertions rather than over \emph{knowing that} have received much attention~\cite{Pavese22}. Intuitively, an agent knows how to achieve~$\varphi$ given~$\psi$ if she has at her disposal a suitable \emph{course of action} guaranteeing that $\varphi$ will be the case, whenever she is in a situation in which $\psi$ holds. The concept of knowing how is important both from a philosophical perspective and from a computer science point of view. For instance, it can be seen as a formal account for automated planning and strategic reasoning in AI (see, e.g.,~\cite{KandA15}).

Most traditional approaches for representing \emph{knowing how} rely on combining logics of \emph{knowing that} with logics of action (see,
e.g.,~\cite{Mccarthy69,Les00,HerzigT06}). However,
while a combination of operators for \emph{knowing that} and \emph{ability} (e.g.,~\cite{wiebeetal:2003}) produces a \emph{de dicto} reading of the concept (\emph{``the agent knows she has an action that guarantees the goal''}), a proper notion of \emph{``knowing how to achieve $\varphi$''} requires a \emph{de re} clause (\emph{``the agent has an action that she knows guarantees the goal''}; see~\cite{JamrogaA07,Herzig15} for a discussion).
Based on these considerations, \cite{Wang15lori,Wang2016} introduced
a new framework based on a \emph{knowing how} binary modality
$\kh(\psi,\varphi)$. At the semantic level, this language is interpreted over relational models --- called in this context labeled transition systems (LTSs).
In these models, relations describe the actions an
agent has at her disposal for their execution (in some sense, her \emph{abilities}).
%An edge labeled $a$ going from state $w$ to state $u$ indicates that the agent can execute action $a$ to go from $w$ to $u$. Thus,
Then,
$\kh(\psi,\varphi)$ holds if and only if there is a ``proper plan'' (a sequence of
actions satisfying certain constraints) in the LTS that
unerringly leads from every $\psi$-state only to $\varphi$-states.

While variants of this idea have been explored in the literature (see, e.g.,~\cite{Li17,LiWang17,FervariHLW17,Wang19a}),
%weakens the contraints on the required plan, \cite{LiWang17} adds requirements on the plan's intermediate states, \cite{FervariHLW17} combines it with a \emph{knowing that} operator.}
most of them share a fundamental characteristic: relations are interpreted as the agent's available actions; and the abilities of an agent depend \emph{only} on what these actions can achieve. The framework presented in~\cite{AFSVQ21,AFSVQ23report} changed this underlying idea by adding a notion of ``indistinguishability'' between plans, related to the notion of \emph{strategy indistinguishability} of, e.g.,~\cite{JamrogaH04,Belardinelli14}. Two main insights are introduced in these articles, that try better model the knowledge a certain agent have about her environment.  The first insight is simple, while certain actions might be available in a given environment, they might not be available \emph{to a given agent} (e.g., she might not know about them, or did not consider them at the given time). Hence, not all possible plans are available to all agents.  More importantly, she might consider some of these actions or plans \emph{indistinguishable}\footnote{The exact meaning of ``indistinguishable'' is left open. In particular, it does not necessarily mean that the agent cannot tell the plans apart, it might just be that she considers the differences not relevant.} from some others. In such cases, having a proper plan $\plan$ that leads from any $\psi$-state to only $\varphi$-states is not enough. Instead, the agent also needs for \emph{all her available plans that she cannot distinguish from $\plan$} to satisfy such requirements. As argued in~\cite{AFSVQ21}, the benefits of these new semantics are threefold. First, it provides an epistemic indistinguishability-based  view of an agent's abilities. Second, it enables us to deal with multi-agent scenarios in a more natural way. Third, this new perspective leads to a natural definition of operators that represent dynamic aspects of \emph{knowing how}, more aligned with \emph{dynamic epistemic logic} (DEL)~\cite{DELbook}.
%
% This is in sharp contrast with standard epistemic logic (EL)~\cite{Hintikka:kab}, where relational models have two kinds of information: ontic facts about a given situation (represented by the current state in the model), and the particular perspective that agents have (represented by the possible states available in the model, and their respective indistinguishability relation between them). If one would like to mirror the situation of EL, it seems natural that \emph{knowing how} should be defined in terms of some kind of indistinguishability over the  information provided by an LTS.  Such an extended model would be able to capture both the abilities of an agent as given by her available actions, together with the (in)abilities that arise when considering two different actions/plans/executions indistinguishable.
%
%Syntactically, its main component is a binary modality $\khi(\psi,\varphi)$, for each agent~$i$.  Semantically, the crucial idea is the inclusion of an indistinguishability relation between plans over an \lts for each agent $i$, on top of the \lts.
%
%The new structures are called \emph{uncertainty-based \lts} (\ults). Then, formulas are interpreted over \ultss. In this proposal, an agent may have different alternatives at her disposal to try to achieve a goal, all ``as good as any other'' (and in that sense indistinguishable) as far as she can tell. In this way, \ultss aim to reintroduce the notion of epistemic indistinguishability, now at the level of plans.
%
This paper focuses on the latter point. We will make use of the in\-dis\-tin\-guisha\-bil\-i\-ty-based semantics to investigate some dynamic operators describing changes in the agents' abilities and/or the agent's perception about them, and hence in their corresponding epistemic states. As said, our goal is to imitate ideas that lead to dynamic updates in standard epistemic logic, in the context of knowing how.

Standard ``knowing that'' is interpreted using Kripke models~\cite{mlbook,HML}: relational structures in which the elements of the domain (called possible states/situations/worlds) represent the different ways the real world could be, and in which a binary relation (the epistemic indistinguishability relation) represents the agent's uncertainty by indicating the states the agent considers indistinguishable from each one. Then, the agent is said to know that a proposition $\varphi$ is true if and only if $\varphi$ holds in all the states she cannot distinguish from the real one. It is typically assumed that the indistinguishability relation is reflexive, transitive and symmetric (i.e., an equivalence relation), reflecting in this way that the real situation is always a possibility, that if it is possible for a situation to be possible, then the situation is possible, and that indistinguishability is bidirectional, respectively. Besides its simplicity, this representation of knowledge based on relational indistinguishability has an added value: actions working towards increasing the agent's knowledge can be straightforwardly represented as model operations that remove edges (and thus reduce the agent's uncertainty). This is the main idea behind public announcement logics~\cite{Plaza89:lopc}, one of the first works on dynamic epistemic logic, the field that studies how the attitudes of an agent (or a set of agents) change through diverse informational actions. For the case of knowing how, in~\cite{AFSVQ21,AFSVQ23report} it is put forwards that indistinguishability can be defined over plans rather than over states. In turn, the underlying LTS represents \emph{ontic} information which is common to all agents, while the indistinguishability relation models the \emph{epistemic} perception of each particular agent. Thus, an informational action that modifies the LTS represents an ontic update, while epistemic updates should be achieved by updating the indistinguishability relation for an agent. 
To the best of our knowledge, this is the first time in which this theme is addressed (except by the brief discussion introduced in~\cite{Wang2016} about announcements in the context of knowing how, and the work in~\cite{AFSV22} that this paper extends). 

We start by investigating operators that update models based on some sort of \emph{announcement}, in the spirit of~\cite{Plaza89:lopc}. By following closely standard definitions of announcements, the operation results in an ontic update in our setting. This is due to the fact that accessibility relations and states in the LTS represents objective information available for the agents. Then, we will define operations that perform epistemic updates. In particular, we will discuss ways in which the indistinguishability relation between plans can be refined, causing some change in the agents' knowledge. With these proposals we illustrate what are the difficulties of dealing with this kind of modalities in the context of knowing how logics. More precisely, we will present the main challenges that raise in axiomatizing these new dynamic modalities. We discuss both how the failure of the uniform substitution property in the logics, and the weak expressive power of the base static logic are obstacles in defining a complete axiomatization for some dynamic modalities. As a consequence, as a novelty with respect to~\cite{AFSV22}, we introduce a new proposal, in which we extend the underlying (static) knowing how logic, with basic modalities $\mlbox{a}$, for each action $a$, and with a dynamic modality $\srefbox{\plan}$, revealing the possibility of executing $\plan$, and differentiating it from the rest of the available plans. With this additional expressivity, we are able to obtain a completeness result for the resulting dynamic logic, via reduction axioms.
\medskip

\noindent
\textbf{Outline:} This article is organized as follows. \Cref{sec:basic} recalls the syntax, semantics and a complete axiomatization of the multi-agent \emph{knowing how} logic from~\cite{AFSVQ21}, discussing also a corresponding notion of bisimulation~\cite{AFSVQ23report}. Interestingly, we show that one of the axioms presented in~\cite{AFSVQ21,AFSVQ23report} can be removed as it is derivable from the others. In~\Cref{sec:ontic} ontic updates are investigated, based on public announcements~\cite{Plaza89:lopc} and arrow updates~\cite{KooiR11}. We discuss the properties of the operations, and provide reduction axioms over a very restricted class of models. In~\Cref{sec:epistemic-basic} we provide alternatives for epistemic updates, and discuss some of their semantic properties. In particular, we prove that uniform substitution does not hold in general. \Cref{sec:extension} introduces a new proposal in which epistemic updates are defined over an extended language, and provides a complete axiomatization via reduction axioms. Moreover, we show that the satisfiability problem for the resulting logics is decidable, using filtrations. Finally, in~\Cref{sec:final} we offer some final remarks and discuss future lines of work.
