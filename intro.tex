%\begin{textonuevo}
% #4.1: knowing that + its dynamics + how those ideas can be used for knowing how (KH)
The notion of \emph{knowing that} has been one of the fundamental study subjects in Epistemology. It concerns the knowledge an agent might have about the truth-value of propositions, and as such has driven most of the field's research agenda, with some authors even claiming that it subsumes many other forms of knowledge~(see, e.g., \cite{BoenLycan75,stanley2001knowing,Snowdon2004}). 

 
One of the most successful tools for studying this concept has been Epistemic Logic (EL; \cite{Hintikka:kab}), a formal logical framework which, on the semantic side, typically relies on pointed Kripke models~\cite{mlbook,HML}. In more detail, a Kripke model is a relational structure whose domain's elements (called possible states/situations/worlds) represent the different ways the real world can be, and in which a binary \emph{epistemic indistinguishability} relation represents the agent's uncertainty. Then, a pointed Kripke model is a Kripke model with a distinguished evaluation state, usually understood as describing the actual situation. With this, knowledge about the truth-value of propositions is defined as follows: at a given state $w$, the agent knows that a proposition $\varphi$ is true if and only if $\varphi$ holds in all the states she cannot distinguish from $w$. Asking for the indistinguishability relation to be reflexive, transitive and Euclidean (i.e., an equivalence relation) makes knowledge truthful, positively and negatively introspective. EL has proved to be a useful tool for developments in, e.g., Philosophy~\cite{rfe,Holliday2018}, Computer Science~\cite{RAK}, AI~\cite{elfaics} and Economics~\cite{egepgt}. In achieving this, one of its most appealing aspects is that it can be used to reason about information change. Indeed, actions that increase the agent's knowledge can be straightforwardly represented as model operations that remove edges, thus reducing the agent's uncertainty. Further research on this idea has given rise to Dynamic Epistemic Logic (DEL), a field that studies how the attitudes of a set of agents change through diverse informational actions (see, e.g.,~\cite{DELbook,vanBenthem2011ldii}). 

%\smallskip

% #1: motivating KH
Still, knowing that is not the only form of knowledge an agent might have/use. In fact, the last few years have witnessed the emergence of formal frameworks for reasoning about knowing whether, knowing how, knowing why, knowing what, and so on (see~\cite{Wang16} for an overview). Among them, knowing how (e.g.,~\cite{Pavese22}) is one of the most important, as it concerns the \emph{ability} an agent has to achieve a given outcome. Intuitively, an agent knows how to achieve~$\varphi$ given~$\psi$ if she has at her disposal a suitable \emph{course of action} guaranteeing that $\varphi$ will be the case whenever she is in a situation in which $\psi$ holds. It is worthwhile to study  this notion not only from a philosophical perspective, but also from a computer science point of view, as the concept can be seen, for example, as a formal account for automated planning and strategic reasoning in AI (see, e.g.,~\cite{KandA15}).

% #2: knowing that (KT) + ability is not enough, and a brief description of LTS semantics
Most traditional approaches for representing knowing how rely on combining logics of knowing that with logics of action (see, e.g.,~\cite{Mccarthy69,Les00,HerzigT06}). However, while a combination of operators for knowing that and \emph{ability} (e.g.,~\cite{wiebeetal:2003}) produces a \emph{de dicto} reading of the concept (\emph{``the agent knows she has an action that guarantees the goal''}), a proper notion of \emph{``knowing how to achieve $\varphi$''} requires a \emph{de re} clause (\emph{``the agent has an action that she knows guarantees the goal''}; see~\cite{JamrogaA07,Herzig15} for a discussion). Based on these considerations, \cite{Wang15lori,Wang2016} introduced a framework based on a knowing how binary modality $\kh(\psi,\varphi)$. At the semantic level, this language is also interpreted over relational models --- called labeled transition systems (LTSs) in this context. The difference with respect to the EL setting is that, here, relations describe the actions an agent considers she has at her disposal (in some sense, her \emph{epistemic abilities}).
%An edge labeled $a$ going from state $w$ to state $u$ indicates that the agent can execute action $a$ to go from $w$ to $u$. Thus,
Then, $\kh(\psi,\varphi)$ holds if and only if there is a ``proper plan'' (a sequence of actions satisfying certain constraints) in the LTS that unerringly leads from every $\psi$-state only to $\varphi$-states.


% #3: introducing indistinguishability and discussing its benefits
While variants of this idea have been explored in the literature (see, e.g.,~\cite{Li17,LiWang17,FervariHLW17,Wang19a,LiW24}),
%weakens the contraints on the required plan, \cite{LiWang17} adds requirements on the plan's intermediate states, \cite{FervariHLW17} combines it with a \emph{knowing that} operator.}
most of them share the fundamental feature mentioned above: relations in the model are interpreted as the agent's epistemic take on her available actions. Due to this, the epistemic abilities of an agent depend \emph{only} on what these actions can achieve. The framework presented in~\cite{AFSVQ21,AFSVQ23report} changed this by adding a notion of indistinguishability between plans, related to the notion of \emph{strategy indistinguishability} (e.g.,~\cite{JamrogaH04,Belardinelli14}). Two main insights are introduced in these articles. The first is simple: while certain plans might be available in a given environment, they might not be available \emph{to a given agent} (e.g., she might not know about them, or did not consider them at the given time). Hence, not all possible plans are available to all agents.  More importantly, she might consider some of these plans \emph{indistinguishable} from some others.\footnote{The exact meaning of ``indistinguishable'' is left open. In particular, it does not necessarily mean that the agent cannot tell the plans apart, it might just be that she considers the differences irrelevant.} In such cases, having available a proper plan $\plan$ that leads from any $\psi$-state to only $\varphi$-states is not enough: the agent also needs for \emph{all the plans she cannot distinguish from $\plan$} to satisfy the same requirement. 
As argued in~\cite{AFSVQ21}, the benefits of these new semantics are threefold. First, it provides an indistinguishability-based view of an agent's epistemic abilities, akin to the EL approach for modeling knowing that. Second, it can deal with multi-agent scenarios more naturally. Third, it leads to a natural definition of operators representing dynamic aspects of knowing how, akin to the DEL approach for modeling dynamics of knowing that.
%
% This is in sharp contrast with standard epistemic logic (EL)~\cite{Hintikka:kab}, where relational models have two kinds of information: ontic facts about a given situation (represented by the current state in the model), and the particular perspective that agents have (represented by the possible states available in the model, and their respective indistinguishability relation between them). If one would like to mirror the situation of EL, it seems natural that \emph{knowing how} should be defined in terms of some kind of indistinguishability over the  information provided by an LTS.  Such an extended model would be able to capture both the abilities of an agent as given by her available actions, together with the (in)abilities that arise when considering two different actions/plans/executions indistinguishable.
%
%Syntactically, its main component is a binary modality $\khi(\psi,\varphi)$, for each agent~$i$.  Semantically, the crucial idea is the inclusion of an indistinguishability relation between plans over an \lts for each agent $i$, on top of the \lts.
%
%The new structures are called \emph{uncertainty-based \lts} (\ults). Then, formulas are interpreted over \ultss. In this proposal, an agent may have different alternatives at her disposal to try to achieve a goal, all ``as good as any other'' (and in that sense indistinguishable) as far as she can tell. In this way, \ultss aim to reintroduce the notion of epistemic indistinguishability, now at the level of plans.
%
%\end{textonuevo}

%\smallskip


This article focuses on the latter point, using this in\-dis\-tin\-guisha\-bil\-i\-ty-based semantics to study several dynamic operators describing changes in the agents' epistemic abilities. To the best of our knowledge, this is the first time this theme is addressed (except by the brief discussion in~\cite{Wang2016} about knowing how and announcements, and the work in~\cite{AFSV22} that this article extends). The proposals follow the DEL approach, using changes in the model to represent changes in the situation they describe. In doing so, we take advantage of the fact that, in these structures, there is a clear distinction between ontic and epistemic information. Indeed, while the underlying LTS contains \emph{ontic} facts common to all agents (the available actions as well as their effects), the indistinguishability relation over plans represents the \emph{epistemic} perception of each particular agent (the plans an agent considers available, as well as her capacity to distinguish between them). Thus, model-update operations affecting different parts of the model have a clear-cut interpretation: while changes in the LTS can be seen as ontic updates (possibly with epistemic consequences), changes in plan indistinguishability relation can be seen as direct epistemic ones. 

%-----------------------
% The previous text

%\fer{Still to revise from here to the end of the intro}The paper starts by investigating operators that update models based on some sort of \emph{announcement}, in the spirit of~\cite{Plaza89:lopc}. By following closely standard definitions of announcements, the operation results in an ontic update in our setting. This is because accessibility relations and states in the LTS represent objective information available for the agents. Then, we will define operations that perform epistemic updates. In particular, we will discuss ways in which the indistinguishability relation between plans can be refined, causing some change in the agents' knowledge. With these proposals, we illustrate the difficulties of dealing with this kind of modalities in the context of knowing how logics. More precisely, we will present the main challenges that arise in axiomatizing these new dynamic modalities. We discuss both how the failure of the uniform substitution property in the logics, and the weak expressive power of the base static logic are obstacles in defining a complete axiomatization for some dynamic modalities. As a consequence, as a novelty with respect to~\cite{AFSV22}, we introduce a new proposal, in which we extend the underlying (static) knowing how logic, with basic modalities $\mlbox{a}$, for each action $a$, and with a dynamic modality $\srefbox{\plan}$, revealing the possibility of executing $\plan$, and differentiating it from the rest of the available plans. With this additional expressivity, we can obtain a completeness result for the resulting dynamic logic, via reduction axioms.

%\smallskip

%\noindent\textbf{Outline:} This article is organized as follows. \Cref{sec:basic} recalls the basic definitions of the indistinguishability-based knowing how setting (\cite{AFSVQ21,AFSVQ23report}), including results that are useful in the rest of this manuscript.  Interestingly, we show that one of the axioms presented in~\cite{AFSVQ21,AFSVQ23report} can be removed as it is derivable from the others. In~\Cref{sec:ontic}, ontic updates are investigated, based on public announcements~\cite{Plaza89:lopc} and arrow updates~\cite{KooiR11}. We discuss the properties of the operations, providing reduction axioms over a very restricted class of models. In~\Cref{sec:epistemic-basic} we provide alternatives for epistemic updates, and discuss some of their semantic properties. In particular, we prove that uniform substitution does not hold in general. \Cref{sec:extension} introduces a new proposal in which epistemic updates are defined over an extended language, and provides a complete axiomatization via reduction axioms. Moreover, we show that the satisfiability problem for the resulting logics is decidable, using filtrations. Finally, in~\Cref{sec:final} we offer some final remarks and discuss future lines of work.

%-----------------------
% an initial attempt

%The paper starts by investigating \emph{ontic} change, and the focus is two-fold: first, an operation removing states (in the spirit of~\cite{Plaza89:lopc}), and then one removing edges (in the spirit of~\cite{KooiR11}). Still, the main focus is \emph{epistemic} change. In the initial phase, the focus is two-fold too: first, an operation eliminating uncertainty between specific plans, and then one that quantifies over the former by quantifying over plan refinement. 

%The modalities used for describing the effects of all these operations increase the expressivity of the basic language, and thus an axiomatisation via reduction axioms is not possible. In the case of ontic updates, this leads to an alternative strategy: restrict the setting to a particular class of models. In the case of epistemic updates, this leads to a second phase in which the basic language is extended with a standard modality for each basic action. In this new setting, a further model-update operation is studied, for which a reduction-axiom-based axiomatisation is provided.

%\smallskip

%\noindent\textbf{Outline:} This article is organized as follows. \Cref{sec:basic} recalls the basic definitions of the indistinguishability-based \emph{knowing how} setting, including results that are useful in the rest of this manuscript. (Interestingly, we show that one of the axioms is derivable from the others.) \Cref{sec:ontic} investigates ontic updates via (variations of) well-known state-removing and edge-removing operations, then provides reduction axioms over a restricted class of models. \Cref{sec:epistemic-basic} is the initial exploration of alternatives for epistemic updates, and \Cref{sec:extension} extends the basic setting to introduce a new proposal in which epistemic updates are defined over an extended language. \Cref{sec:final} closes by offering some final remarks and discussing future lines of work.

%-----------------------
% new version

\medskip

\begin{mrevised}
The rest of the article is organized as follows. \Cref{sec:basic} recalls the basic definitions of the indistinguishability-based knowing how setting \cite{AFSVQ21,AFSVQ23report}, including some new results. In particular, we show that one of the axioms presented in the axiomatization introduced in \cite{AFSVQ21,AFSVQ23report} is derivable from the others, while a new axiom is needed to fix a small gap in the completeness argument. We also discuss some results concerning the expressive power of the basic knowing how logic over particular classes of models. 

\Cref{sec:ontic,sec:epistemic-basic,sec:extension} constitute the core of the contribution. Indeed, \Cref{sec:ontic}, investigates \emph{ontic} updates by introducing model operations that affect the ontic components of the models: first, an operation that removes states (in the spirit of~\cite{Plaza89:lopc}), and then one that removes edges (in the spirit of~\cite{KooiR11}). Then, \Cref{sec:epistemic-basic,sec:extension} are devoted to \emph{epistemic} updates. While \Cref{sec:epistemic-basic} discusses two operations for eliminating uncertainty between plans (the first works over two given plans; the second quantifies over the parameters of the first),\footnote{The dynamic operators associated with these updates were first introduced in~\cite{AFSV22}. Here we provide a more detailed analysis and complete proofs.} \Cref{sec:extension}
introduces an epistemic update that makes a given plan distinguishable from any other. \Cref{sec:final} closes the article by offering some final remarks and discussing future lines of work.

Among the results presented in \Cref{sec:ontic,sec:epistemic-basic,sec:extension}, one can highlight the following. First, it is  shown that the modalities capturing the model updates from \Cref{sec:ontic,sec:epistemic-basic} increase the expressivity of the basic knowing how language. Hence, an axiomatization via reduction axioms is not possible. For the ontic updates, reduction axioms can be defined when we restrict the logic to a particular class of models, but this approach does not work for the epistemic modalities. The modality associated with the update of \Cref{sec:extension} also increases the expressive power, and hence no reduction axioms exist. However, reduction axioms can be obtained by further extending the language with standard modalities for each basic action. 
 
\end{mrevised}
