\begin{comment}
In this paper we investigated dynamic modalities in the context of \emph{knowing how} logics. Our starting point has been the uncertainty-based semantics from~\cite{AFSVQ21,AFSVQ23report}. 
Building on this, we studied two forms of updates: ontic updates, via annoucement-like and arrow-update-like modalities; and epistemic updates, refining the perception of an agent regarding her own abilities. For the operators encompassed in the former family, we showed that in their general form, they increase the expressivity of the logic. Moreover, we restricted ourselves to a particular class of models and we provided axiomatizations via reduction axioms. For the latter family, we discussed some preliminary thoughts and semantic properties of each operator, and we discussed some limitations of this setting. Concretely, we showed that the presented logics lack uniform substitution, a typical property that is used to obtain complete axiomatizations. To deal with this issue, we presented an extended, underlying static language, and then proposed some dynamic modalities for epistemic updates. This contribution is completely novel with respect to~\cite{AFSV22}. Here, we presented a modality that makes a single plan  distinguishable from any other from the perspective of the agents. 
% three modalities, each of them refining the other. First, we presented a modality that makes a single action distinguishable from any other from the perspective of the agents. The second one, generalizes the first one, by allowing plans instead of single actions. Finally, we presented a semi-private version of the modality, in which the distinction of a plan is made only to a particular agent. 
In this extended setting, apart from showing how it could be useful for the examples we introduced, we obtained sound and complete axiomatizations via reduction axioms, and showed that the satisfiability problem for these logics is decidable. 
\end{comment}

This paper investigated dynamic modalities in the context of \emph{knowing how} logics. Our starting point has been the uncertainty-based semantics from~\cite{AFSVQ21,AFSVQ23report}. 
Building on this, we studied two forms of updates: ontic updates (via model operations for removing worlds and for removing edges) and, more extensively, epistemic updates (via model operations for refining the agents' uncertainty over plans). For the operators encompassed in the former family, we showed that in their general form, they increase the expressivity of the logic. Then, we provided axiomatizations via reductions axioms by restricting ourselves to a particular class of models. For those in the latter family, we started by presenting an operation for distinguishing between two given plans, discussing some of its properties and building an action for arbitrary refinement on top of it. The issue of complete axiomatization over the class of all models is still elusive in this setting, and this leads to the final proposal in this text: an extension of the underlying static language (with a normal modality for each basic action) and an arguably `more robust' epistemic action (which makes a single plan distinguishable from any other). For this extended setting, we provided sound and complete axiomatizations via reduction axioms, and showed that the satisfiability problem for these logics is decidable via filtrations. 

To the best of our knowledge, this is the first attempt to establish a theory of dynamic epistemic logics for knowing how. We argue that the semantics provided in~\cite{AFSVQ21,AFSVQ23report} is the crucial aspect for succeeding in this goal. For the so-called ontic updates, we make use of existing dynamic modalties (announcements-like and arrow-updates-lile modalities) to perform changes in the knowledge of the agents. Conceptually, these operations are natural choices, since knowledge is still obtained from the ability of an agent to access certain worlds. From a technical side, we found a limitation in providing axiomatizations, partly due to the fact that modalities cannot talk explicitely about plans. However, we obtained axiomatizations over restricted classes of models, similarly as those considered in e.g.~\cite{Li21} for a local knowing how modality.

Another aspect unexploited by the ontic proposals, is the dimension involving plan indistinguishability, the most distinctive feature of our semantics. In this regard, we proposed modalities that remove indistinguishability between plans, reducing the uncertainty of an agent. Similar to what happened to very general dynamic modalities (see e.g.~\cite{ArecesFH15}), axiomatizing these operators turns to be challenging. We showed that for instance in our proposals, uniforme substitution does not hold. To overcome this issue, we propose a new logic featuring two novel operations: a basic modality $[a]$, and a novel dynamic modality that distinguishes the effect of a plan from the rest. The latter emerges as an alternative of more expressive operators, but in which the information about a single plan is revealed, instead of information about two plans being different. For the former, the benefits are twofold. First, it enables us to explicitely talk about the execution of actions. Second, as a by product it provides us the expressivity to obtain an axiomatization via reduction axioms.

\begin{comment}
To the best of our knowledge, this is the first attempt to establish a theory of dynamic epistemic logics for knowing how. We argue that the semantics provided in~\cite{AFSVQ21,AFSVQ23report} is the crucial aspect for succeeding in this goal. Moreover, our work opens the path to study other dynamic operators in this context. For instance, we could define dynamic modalities based on action models, like those in~\cite{BaltagMS98,DELbook,GalimullinA22}. 
Also, it would be interesting to explore alternative techniques for obtaining proof systems without a general rule of substitution, for instance, by building a dynamic logic over a hybrid logic semantics (see e.g.~\cite{BenthemMZ2022}). Finally, we would like to characterize the exact complexity of the dynamic logics we introduced.
\end{comment}


While this paper answers some questions, some others remain unanswered. 
Some of those involve our specific proposals. 
%Regarding the ontic dynamic operators, one would like to know whether there are reduction axioms for the `standard' state-removing and arrow-update operators in $\cultsba$ or other alternative classes of models.\fer{El enunciado anterior se puede borrar, dependiendo de si agreagamos algo sobre esto en la sección 3.1 y 3.2} 
Concerning the extension of the `static' \emph{knowing how} setting, one wonders the precise relationship, expressivity-wise, between the new language $\KHiMLlogic$ and other known languages (e.g., $\mathsf{L}_{\square,\A}$). In fact, following that line of thought, one also wonders whether the $\kh$ modality is definable from other `more basic' operators (e.g., standard normal modalities plus nominals and/or operators quantifying over actions) and whether this would yield a more expressive language on which reduction axioms for the introduced dynamic modalities exist (cf.~\cite{BenthemMZ2022}). Also, we would like to characterize the exact complexity of the introduced dynamic logics.

But one can also take a broader look, and study other (maybe more general) dynamic operators that change the agents' abilities. For instance, one can use ideas from~\cite{BaltagMS98} to define actions of semi-private ability-change (cf.~\cite{GalimullinA22}). To the best of our knowledge, this is the first attempt of establishing a theory of dynamic epistemic logics for the notion of \emph{knowing how}, and we hope this will be followed by other proposals.
 
