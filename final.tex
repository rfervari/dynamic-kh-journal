\begin{comment}
In this paper we investigated dynamic modalities in the context of \emph{knowing how} logics. Our starting point has been the uncertainty-based semantics from~\cite{AFSVQ21,AFSVQ23report}. 
Building on this, we studied two forms of updates: ontic updates, via annoucement-like and arrow-update-like modalities; and epistemic updates, refining the perception of an agent regarding her own abilities. For the operators encompassed in the former family, we showed that in their general form, they increase the expressivity of the logic. Moreover, we restricted ourselves to a particular class of models and we provided axiomatizations via reduction axioms. For the latter family, we discussed some preliminary thoughts and semantic properties of each operator, and we discussed some limitations of this setting. Concretely, we showed that the presented logics lack uniform substitution, a typical property that is used to obtain complete axiomatizations. To deal with this issue, we presented an extended, underlying static language, and then proposed some dynamic modalities for epistemic updates. This contribution is completely novel with respect to~\cite{AFSV22}. Here, we presented a modality that makes a single plan  distinguishable from any other from the perspective of the agents. 
% three modalities, each of them refining the other. First, we presented a modality that makes a single action distinguishable from any other from the perspective of the agents. The second one, generalizes the first one, by allowing plans instead of single actions. Finally, we presented a semi-private version of the modality, in which the distinction of a plan is made only to a particular agent. 
In this extended setting, apart from showing how it could be useful for the examples we introduced, we obtained sound and complete axiomatizations via reduction axioms, and showed that the satisfiability problem for these logics is decidable. 
\end{comment}

In this article, we investigated dynamic modalities in the context of knowing how logics. Our starting point has been the uncertainty-based semantics from~\cite{AFSVQ21,AFSVQ23report}, where indistinguishability is defined over plans instead of over states as in the framework of e.g.~\cite{FervariHLW17,LiW24}. Building on this, we studied two kinds of model updates: ontic updates (via model operations for removing states and edges) and, more extensively, epistemic updates (via model operations for refining the agents' uncertainty over plans). 

We show that, in general, the logics obtained when these operators are added to the basic knowing how logic $\KHilogic$ have increased expressive power.  As a result, the logics cannot be axiomatized by reduction axioms. 
In most cases, we were able to define a complete axiomatization using reduction axioms by restricting the class of models considered.  We introduced the class of models $\cultsba$ (\textbf{BA} stands for `basic actions') as the class of models $\modults = \tup{\W, \R, \Unc, \V}$ in which, for all $i \in \AGT$, we have that $\plans \in \Unc(i)$ implies $\plans \subseteq \ACT$.  $\cultsba$ could be interpreted as a more abstract representation of the abilities of the agents, where every plan is modeled as a single atomic action.  This class bears resemblance to restricted classes considered in~\cite{Li21} for a local knowing how modality.
Interestingly, while the basic knowing how logic $\KHilogic$ cannot distinguish between the class of all \ults and $\cultsba$, this is not the case in the presence of the dynamic operators we introduced. 

For the ontic updates, we considered dynamic modalities well investigated in the DEL literature (see, e.g.,~\cite{Plaza89:lopc,KooiR11,DELbook}) that perform changes in the knowledge of the agents encoded by the \lts. These modalities take new meanings in the context of \ults{s}.  Conceptually, these operations are natural choices, since part of the knowing how knowledge of an agent arises from the ability she has to access certain worlds through certain plans.  These updates concern changes in the set of states of the model and in the accessibility relations.  It is interesting to reflect on the impact that these updates have on the different logics discussed in the article.  In DEL, updates on states and edges are clearly epistemic updates, as states other than the current state are situations the agent considers possible, and the edges linking them are indistinguishability relations connecting states the agent cannot discern.  Hence, eliminating states and edges results in additional knowledge for the agents.  In the original presentation of a knowing how operator with semantics defined over \lts, edges can be considered as encoding the agent's epistemic take on her abilities: the actions she considers possible, their preconditions and effects.  Once more, removing states and edges has epistemic impact.  


When \lts{s} are replaced by \ults{s}, states and edges encode ontic information. In other words, they stand for the actions available in the considered situation, their preconditions and effects, independently of the knowledge the agents have about them.  
%
%Remember that, in the knowing how
%setting, states other than the evaluation point and the accessibility relations linking them, do not represent epistemic information (as they do in DEL). Rather, they encode the network of situations reachable by the actions agents can perform. 
%
Thus, in the uncertainty-based knowing how setting, the elimination of states and relations indicates not that they are no longer epistemically considered possible, but rather that they are no longer available in the current representation of the world. 
%Contrast this situation with the one in the original knowing how setting of, e.g.~\cite{Wang15lori,Wang2016}, where the relations define the agents’
%abilities. Under that semantics, removing states corresponds to both an ontic
%and an epistemic change: the effect of available actions changes, and hence so do the
%agents’ abilities. 
%However, in the \ults-based semantics, relations provide only
%ontic information; t
Thus, removing states produces an ontic change that might affect,
indirectly, the agents’ epistemic abilities. 


Ultimately, the main focus of the article is on epistemic updates.  We started by presenting an operation that distinguishes between two given plans and discussing some of its properties. In turn, this operator can be used to define one for arbitrary refinement. The issue of complete axiomatization (both over the class of all models and over some restricted class like $\cultsba$) is still elusive in this setting.  
Our final proposal for an epistemic dynamic modality is the operator $\srefbox{\plan}$ that makes the plan $\plan$ distinguishable from any other, for all agents. In this case, we propose the language $\PlanReflogic$ that also extends the underlying static language with a normal modality for each basic action. For this logic, we provide sound and complete axiomatizations via reduction axioms over the class of all \ults{s}, and show that the satisfiability problem is decidable via filtrations.  

\begin{comment}
Another aspect unexploited by the ontic proposals, is the dimension involving plan indistinguishability, the most distinctive feature of our semantics. In this regard, we proposed modalities that remove indistinguishability between plans, reducing the uncertainty of an agent. Similar to what happened to very general dynamic modalities (see e.g.~\cite{ArecesFH15}), axiomatizing these operators turns to be challenging. We showed that for instance in our proposals, uniforme substitution does not hold. To overcome this issue, we propose a new logic featuring two novel operations: a basic modality $[a]$, and a novel dynamic modality that distinguishes the effect of a plan from the rest. The latter emerges as an alternative of more expressive operators, but in which the information about a single plan is revealed, instead of information about two plans being different. For the former, the benefits are twofold. First, it enables us to explicitly talk about the execution of actions. Second, as a by product it provides us the expressivity to obtain an axiomatization via reduction axioms.
\end{comment}

\begin{comment}
To the best of our knowledge, this is the first attempt to establish a theory of dynamic epistemic logics for knowing how. We argue that the semantics provided in~\cite{AFSVQ21,AFSVQ23report} is the crucial aspect for succeeding in this goal. Moreover, our work opens the path to studying other dynamic operators in this context. For instance, we could define dynamic modalities based on action models, like those in~\cite{BaltagMS98,DELbook,GalimullinA22}. 
Also, it would be interesting to explore alternative techniques for obtaining proof systems without a general rule of substitution, for instance, by building a dynamic logic over a hybrid logic semantics (see e.g.~\cite{BenthemMZ2022}). Finally, we would like to characterize the exact complexity of the dynamic logics we introduced.
\end{comment}


%Regarding the ontic dynamic operators, one would like to know whether there are reduction axioms for the `standard' state-removing and arrow-update operators in $\cultsba$ or other alternative classes of models.\fer{El enunciado anterior se puede borrar, dependiendo de si agreagamos algo sobre esto en la sección 3.1 y 3.2} 
This article should be considered as taking the first steps on a systematic study of dynamic operators in the setting of knowing how.  It explores a number of proposals for both ontic and epistemic dynamic modalities. 
Among the introduced languages, $\PlanReflogic$ is,  to our knowledge, the first result of its class, showcasing an interesting proposal for a dynamic logic for knowing how, with an \ults-based semantics, with a sound and complete axiomatization and a decidable satisfiability problem. 

Concerning the extension of the `static' knowing how setting, one wonders the precise relationship, expressivity-wise, between the new language $\KHiMLlogic$ and other known languages (e.g., $\mathsf{L}_{\square,\A}$). In fact, following that line of thought, one also wonders whether the $\kh$ modality is definable from other `more basic' operators (e.g., standard normal modalities plus nominals and/or operators quantifying over actions) and whether this would yield a more expressive language on which reduction axioms for the introduced dynamic modalities exist (see~\cite{BenthemMZ2022}). Also, we would like to characterize the exact complexity of the introduced dynamic logics.

A broader look is also possible, which would involve investigating other (maybe more general) dynamic operators that change the agents' epistemic abilities. For instance, one can use ideas from~\cite{BaltagMS98,GalimullinA22} to define actions of semi-private ability-change.  
