In this paper we investigated dynamic modalities in the context of \emph{knowing how} logics. Our starting point has been the uncertainty-based semantics from~\cite{AFSVQ21,AFSVQ23report}. 
Building on this, we studied two forms of updates: ontic updates, via annoucement-like and arrow-update-like modalities; and epistemic updates, refining the perception of an agent regarding her own abilities. For the operators encompassed in the former family, we showed that in their general form, they increase the expressivity of the logic. Moreover, we restricted ourselves to a particular class of models and we provided axiomatizations via reductions axioms. For the latter family, we discussed some preliminary thoughts and semantic properties of each operator, and we discussed some limitations of this setting. Concretely, we showed that the logics presented lack uniform substitution, a typicall property that is used to obtain complete axiomatizations. In order to deal with this issue, we presented an extended, underlying static language, and then proposed some dynamic modalities for epistemic updates. This contribution is completely novel with respect to~\cite{AFSV22}. Here, we presented a modality that makes a single plan  distinguishable from any other from the perspective of the agents. 
% three modalities, each of them refining the other. First, we presented a modality that makes a single action distinguishable from any other from the perspective of the agents. The second one, generalizes the first one, by allowing plans instead of single actions. Finally, we presented a semi-private version of the modality, in which the distinction of a plan is made only to a particular agent. 
In this extended setting, apart from showing how it could be useful for the examples we introduced, we obtained sound and complete axiomatizations via reduction axioms, and showed that the satisfiability problem for these logics is decidable. 

To the best of our knowledge, this is the first attempt of establishing a theory of dynamic epistemic logics for knowing how. We argue that the semantics provided in~\cite{AFSVQ21,AFSVQ23report} is the crucial aspect for succeeding in this goal. Moreover, our work opens the path to study other dynamic operators in this context. For instance, we could define dynamic modalities based on action models, like those in~\cite{BaltagMS98,DELbook,GalimullinA22}. 
Also, it would be interesting to explore alternative techniques for obtaining proof systems without a general rule of substitution, for instance, by building a dynamic logic over a hybrid logic semantics (see e.g.~\cite{BenthemMZ2022}). Finally, we would like to characterize the exact complexity of the dynamic logics we introduced.
