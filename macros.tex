% lists
\newlist{inlineenum}{enumerate*}{1}
\setlist[inlineenum,1]{label=\textbf{\textit{(\roman*)}}}

\newlist{compactitemize}{itemize}{1}
\setlist[compactitemize,1]{label=$\bullet$, leftmargin=1em, itemsep=0em}


% Environments
\theoremstyle{plain}
\newtheorem{theorem}{Theorem}
\newtheorem{proposition}[theorem]{Proposition}
\newtheorem{lemma}[theorem]{Lemma}
\newtheorem{corollary}[theorem]{Corollary}
\newtheorem{acknowledgements}[theorem]{Acknowledgements}
\newtheorem{claim}[theorem]{Claim}

\theoremstyle{definition}
\newtheorem{definition}[theorem]{Definition}
% \newtheorem{observation}[theorem]{Observation}

\theoremstyle{remark}
% \newtheorem{fact}[theorem]{Fact}
% \newtheorem{question}[theorem]{Question}
\newtheorem{example}[theorem]{Example}
%\newtheorem{remark}[theorem]{Remark}
% \newtheorem{sketch}[theorem]{Sketch}
% \newenvironment{nscenter}
%  {\parskip=2pt\par\nopagebreak\centering}
%  {\par\noindent\ignorespacesafterend}

\newenvironment{spcenter}
{\begin{center}\vspace{1ex}}
{\vspace{1ex}\end{center}}

\newenvironment{ctabular}[1]
{\begin{center}\begin{tabular}{#1}}
{\end{tabular} \end{center}}

\newenvironment{ltabular}[1]
{\begin{flushleft}\begin{tabular}{#1}}
{\end{tabular} \end{flushleft}}

\newenvironment{lttabular}[1]
{\begin{flushleft}\begin{tabular}[t]{#1}}
{\end{tabular} \end{flushleft}}

\newcommand{\nsparagraph}[1]{\noindent\textbf{#1}\,}
\newcommand{\ssparagraph}[1]{\bigskip\noindent\textbf{#1}\,}

\newcommand{\ti}[1]{\textit{#1}}
% \newcommand{\bs}[1]{\boldsymbol{#1}}
\newcommand{\ITM}[1]{{\bfseries \itshape (#1)}}

%Class of models
\newcommand{\classults}{\mathbf{M}}
\newcommand{\cultsfnu}{\mathbf{M}^{\mathbf{F}}_{\mathbf{NU}}}
\newcommand{\cultsfd}{\mathbf{M}_{\mathbf{FD}}}
\newcommand{\cultsnu}{\mathbf{M}_{\mathbf{NU}}}
\newcommand{\cultsac}{\mathbf{M}_{\mathbf{AC}}}
\newcommand{\cultsba}{\mathbf{M}_{\mathbf{BA}}}

\newcommand{\sults}{\mathbf{M}^1}

%Models
\newcommand{\modlts}{\mathcal{S}}
\newcommand{\modults}{\mathcal{M}}
\newcommand{\cmodel}{\modults^\Gamma}
\newcommand{\annsemantics}[2]{#1_{!#2}}
\newcommand{\annmodel}[1]{\modults_{!#1}}
\newcommand{\annmodelchi}{\annmodel{\chi}}
\newcommand{\arrowsemantics}[2]{#1_{{#2}}}
\newcommand{\arrowmodel}[1]{\modults_{{#1}}}
\newcommand{\arrowmodelU}{\arrowmodel{U}}

\newcommand{\clts}{\lts^\Gamma}
\newcommand{\D}[1]{\operatorname{D}_{#1}}
\newcommand{\DS}[1]{\operatorname{P}_{#1}}

\newcommand{\W}{\operatorname{W}}
\newcommand{\R}{\operatorname{R}}
\newcommand{\N}{\operatorname{N}}
\renewcommand{\S}{\operatorname{\asda}}
\newcommand{\Unc}{\operatorname{U}}
\newcommand{\Sother}{\operatorname{\mathbb{T}}}
\newcommand{\Ster}{\operatorname{\mathbb{U}}}
\newcommand{\V}{\operatorname{V}}
\newcommand{\compose}{{;}}
\newcommand{\REACH}[1]{(\leadsto_{#1}^*)}

\newcommand{\lts}{\textup{LTS}\xspace}
\newcommand{\ltss}{{\lts}s\xspace}
\newcommand{\ults}{\textup{LTS$^\text{\textup{U}}$}\xspace}
\newcommand{\ultss}{{\ults}s\xspace}

% Logical and proof symbols
\newcommand{\mldiam}[1]{\tup{#1}}
\newcommand{\mlbox}[1]{[#1]}
\newcommand{\kh}{{\sf Kh}}
\newcommand{\khi}{{\sf Kh}_i}
\newcommand{\lh}[2]{\tup{#1,#2}}
\newcommand{\blh}[2]{[#1,#2]}
\newcommand{\dialhepis}[1]{\tup{#1}}
\newcommand{\boxlhepis}[1]{[#1]}
\newcommand{\learn}{{\sf L}}
\newcommand{\arefdiam}{\tup{\not\sim}}
\newcommand{\arefbox}{[\not\sim]}
\newcommand{\gann}[1]{[!#1]}
\newcommand{\dgann}[1]{\tup{!#1}}
\newcommand{\refdiam}[2]{\tup{#1{\,\not\sim\,}#2}}
\newcommand{\refbox}[2]{[{#1{\,\not\sim\,}#2}]}
% \newcommand{\sarrowdiam}[2]{\tup{\tup{#1,#2}}}
% \newcommand{\sarrowbox}[2]{[\tup{#1,#2}]}
% \newcommand{\arrowdiam}[1]{\tup{#1}}
\newcommand{\arrowbox}[1]{[#1]}
\newcommand{\srefdiam}[1]{\tup{!#1}}
\newcommand{\srefbox}[1]{[!#1]}
\newcommand{\E}{{\sf E}}
\newcommand{\A}{{\sf A}}
\newcommand{\past}{\lozenge^{-1}}
\newcommand{\bpast}{\Box^{-1}}
\newcommand{\arbitrary}{{\mathsf{X}}}

\newcommand{\card}[1]{|{#1}|}
\newcommand{\subs}[1]{[#1]}
\newcommand{\natnum}{\mathbb{N}}
\newcommand{\intint}[2]{[#1 \mathop{...} #2]}

\newcommand\utimes{\mathbin{\ooalign{$\cup$\cr%
   \hfil\raise0.42ex\hbox{$\scriptscriptstyle\times$}\hfil\cr}}}
% \newcommand\utimes{\mathop{\ooalign{$\bigcup$\cr%
%    \hfil\raise0.36ex\hbox{$\scriptscriptstyle\boldsymbol{\times}$}\hfil\cr}}}

\newcommand{\PAL}{\textsf{PAL}\xspace}
\newcommand{\KHlogic}{\mathsf{L}_\kh}
\newcommand{\KHilogic}{\mathsf{L}_{\khi}}
\newcommand{\KHiMLlogic}{\mathsf{L}_{\khi,\square}}
\newcommand{\PAKHilogic}{\mathsf{PAL}_{\khi}}
\newcommand{\LHlogic}{\mathsf{L_{Lh}}}
\newcommand{\Reflogic}{\mathsf{L_{Ref}}}
\newcommand{\AReflogic}{\mathsf{L_{ARef}}}
\newcommand{\PlanReflogic}{\mathsf{L}_{\khi,\square,\srefbox{\plan}}}
\newcommand{\AtomReflogic}{\mathsf{L}_{\khi,\square,\srefbox{a}}}
\newcommand{\AgentReflogic}{\mathsf{L}_{\khi,\square,\srefbox{\plan,i}}}
% \newcommand{\SArrowlogic}{\mathsf{L_{SAU}}}
\newcommand{\AUL}{\mathsf{AUL}}
\newcommand{\AUKHilogic}{\AUL_{\khi}}
\newcommand{\KHaxiom}{\mathcal{L}_{\kh}}
\newcommand{\KHiaxiom}{\mathcal{L}_{\khi}\xspace}
\newcommand{\axset}{\mathcal{L}}
\newcommand{\axm}[1]{\textsf{#1}}

% completeness proof
\newcommand{\smcs}{\Upphi}
\newcommand{\restkh}[1]{#1\vert_{\kh}}
\newcommand{\restnkh}[1]{#1\vert_{\lnot\kh}}
\newcommand{\restkhi}[1]{#1\vert_{\khi}}
\newcommand{\restnkhi}[1]{#1\vert_{\lnot\khi}}
\newcommand{\resta}[1]{#1\vert_{\A}}
\newcommand{\restna}[1]{#1\vert_{\lnot\A}}
\newcommand{\restarbitrary}[1]{#1|_{\arbitrary}}
\newcommand{\restnarbitrary}[1]{#1|_{\lnot\arbitrary}}

% Tuples \tup{x,y} = <x,y>
\newcommand{\tup}[1]{\langle #1 \rangle}
% Tuples \ttup{x,y} = <<x,y>>
\newcommand{\ttup}[1]{\langle\!\langle \mathrm{#1} \rangle\!\rangle}

% Sets \set{x,y} = {x,y}
\newcommand{\set}[1]{\{ #1 \}}
% Sets with condition \setof{y}{xRy} = {y | xRy}
\newcommand{\setof}[2]{\{ #1  \mid #2 \}}

\newcommand{\ddiam}[1]{\ttup{#1}}
\newcommand{\pow}[1]{\mathcal P #1}
\newcommand{\setpaths}[1]{\mathcal P^{#1}}
\newcommand{\seqprop}{\bar{p}}
\newcommand{\seq}[1]{{\sf Seq}(#1)}
\newcommand{\lmodel}[3]{{#1}^{#2}_{#3}}

% Arrows
\renewcommand{\iff}{\mbox{\it iff}}
\newcommand{\siffs}{\;\iff\;}
\newcommand{\iffdef}{\ensuremath{\mbox{\it iff}_{\mbox{\tiny\it  def}}}}
\newcommand{\siffdefs}{\iffdef}
\newcommand{\ra}{\rightarrow}
\newcommand{\To}{\Rightarrow}
\newcommand{\lra}{\leftrightarrow}
\renewcommand{\implies}{\ra}
\newcommand{\inter}{\overset{int}{\longrightarrow}}
\newcommand{\rewtr}{\overset{\Tr}{\longrightarrow}}
\newcommand{\darrow}{\leftrightarrow}

% Definitions
\newcommand{\PROP}{{\rm \sf Prop}\xspace}
\newcommand{\FORM}{{\rm \sf Form}\xspace}
\newcommand{\AGT}{{\rm \sf Agt}\xspace}
\newcommand{\ACT}{{\rm \sf Act}\xspace}
\newcommand{\plans}{\uppi}
\newcommand{\plan}{\sigma}

% bisimulation
\newcommand{\bisim}{\xspace\mathrel{\raisebox{.5ex}{\small\ensuremath{\underline{\!\leftrightarrow\!}}}}\xspace}

% modal equivalence
\newcommand{\modequiv}{\leftrightsquigarrow}
\newcommand{\planequiv}{\mathrel{\leftrightarrows}}

% truth-set
\newcommand{\truthset}[2]{\llbracket #2 \rrbracket^{#1}}
\newcommand{\truthsetsyn}[2]{\mathopen{\{\!\vert} #2 \mathopen{\vert\!\}^{#1}}}

% format
\newcommand{\itm}[1]{{\bfseries \itshape (#1)}}



\newcommand{\exec}{\operatorname{E}}
\newcommand{\stexec}{\operatorname{SE}}


% arrows for executability
\newcommand{\ultsExec}{\Rightarrow}
\newcommand{\ultsExecStrat}[1]{\stackrel{#1}{\ultsExec}}
\newcommand{\ultsExecAgi}{\stackrel{i}{\ultsExec}}


\newcommand{\learnset}[3]{\operatorname{learn}(#1, #2, #3)}
\newcommand{\splitstr}[4]{#1 \leadsto^{#3}_{#4} #2}


%Partition
% To write notes in the text
\newcommand{\raul}[1]{\todo[color=red!30]{{\bf Raul:} {\footnotesize #1}}\xspace}
\newcommand{\bigraul}[1]{\todo[inline,color=red!30]{{\bf Raul:} {\footnotesize #1}}}

\newcommand{\fer}[1]{\todo[color=blue!20]{{\bf Fer:} {\footnotesize #1}}\xspace}
\newcommand{\bigfer}[1]{\todo[inline,color=blue!20]{{\bf Fer:} {\footnotesize #1}}}

\newcommand{\carlos}[1]{\todo[color=green!20]{{\bf Carlos:} {\footnotesize #1}}\xspace}
\newcommand{\bigcarlos}[1]{\todo[inline,color=green!20]{{\bf Carlos:} {\footnotesize #1}}}

\newcommand{\andres}[1]{\todo[color=cyan!20]{{\bf ARS:} {\footnotesize #1}}\xspace}
\newcommand{\bigandres}[1]{\todo[inline,color=cyan!20]{{\bf ARS:} {\footnotesize #1}}}

\newenvironment{comfer}
{ \color{blue} }
{ \normalcolor }


\newenvironment{comraul}
{ \color{red} }
{ \normalcolor }

\newcommand{\colornuevo}{orange}
\newcommand{\nuevo}[1]{\textcolor{\colornuevo}{#1}}

\newenvironment{textonuevo}
{\color{\colornuevo}}
{\normalcolor}

\newcommand{\hlight}[1]{{\setlength{\fboxsep}{2pt}\colorbox{yellow!75!red}{#1}}}
%{\sethlcolor{yellow!25!red}\hl{#1}}

\newcommand{\nota}[1]{\textcolor{red}{#1}}
\newcommand{\notabis}[1]{\textcolor{blue}{#1}}


%Abbreviated Cref
\Crefname{definition}{Def.}{Defs.}
\Crefname{equation}{Eq.}{Eqs.}
\Crefname{figure}{Fig.}{Figs.}
\Crefname{proposition}{Prop.}{Props.}
\Crefname{theorem}{Thm.}{Thms.}
\Crefname{example}{Ex.}{Exs.}
\Crefname{corollary}{Cor.}{Cors.}
\Crefname{inlineenumi}{Item}{Items}
\Crefname{enumi}{Item}{Items}
\Crefname{section}{Sec.}{Secs.}
\Crefname{appendix}{App.}{Apps.}

\newcommand{\Crefitem}[1]{\Cref{#1}}



\usepackage{tikz}
%\usetikzlibrary{automata,arrows,backgrounds,positioning,fit,calc,matrix,decorations.pathmorphing}
\usetikzlibrary{arrows,decorations,shapes,automata,positioning,decorations.pathmorphing}

\tikzset{
  every picture/.style = {
    thick,
    >=stealth',
    node distance = 1.5em and 3em,
  }
  ,
  cross line/.style = {
    preaction = {
      draw=white,
      -,
      line width=4pt
    }
  }
  ,
  state/.style = {
    rectangle,
    rounded corners = 5pt,
    font = \footnotesize,
    draw,
    minimum width = 1em,
    minimum height = 1em
  }
  , % labels of states
  label-state/.style = {
    sloped,
    font = \scriptsize,
    label distance = -2pt
  }
  , % labels of edges
  label-edge/.style = {
    font = \scriptsize,    
    label distance = -2pt
  }
  ,
  within/.style = {
    fill = white,
    inner sep = 2pt
  }
  ,  
}

% Complexity
\newcommand{\NP}{{\rm\textsf{NP}}\xspace}
\newcommand{\Poly}{{\rm\textsf{P}}\xspace}
\newcommand{\PSPACE}{{\rm\textsf{PSpace}}\xspace}
\newcommand{\EXPSPACE}{{\rm\textsf{ExpSpace}}\xspace}
\newcommand {\EXPTIME} {\rm\textsf{ExpTime}\xspace}
\newcommand {\NEXPTIME} {\rm\textsf{NExpTime}\xspace}
\newcommand{\PH}{{\rm\textsf{PH}}\xspace}
