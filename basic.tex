%\nsparagraph{Syntax and semantics.}
This section recalls the syntax and semantics of the knowing how framework as presented in~\cite{AFSVQ21} as well as a complete axiomatization and a suitable notion of bisimulation (both from~\cite{AFSVQ23report}). It also discusses some complexity and expressivity results. Throughout the text, let $\PROP$ be a countable set of propositional symbols, $\ACT$ a denumerable set of action symbols, and $\AGT$ a non-empty finite set of agents.

\medskip

\begin{definition}\label{def:khsyntax}
Formulas of the language $\KHilogic$ are defined by the following grammar:
\begin{spcenter}
  $\varphi ::= p \mid \neg\varphi \mid \varphi\vee\varphi \mid \khi(\varphi,\varphi)$,
\end{spcenter}
with $p \in \PROP$ and $i\in\AGT$. Other Boolean connectives are defined as usual. Formulas as $\khi(\psi,\varphi)$ state that \emph{``when $\psi$ is the case, agent $i$ knows how to make $\varphi$ true''}. Define also the abbreviations $\A\varphi:=\khi(\neg\varphi,\bot)$ for an arbitrary $i$, and $\E\varphi:=\neg\A\neg\varphi$.
% as established in~\cite{AFSVQ21}, they behave exactly as the universal and existential modalities (see, e.g.~\cite{GorankoP92}), respectively.
\end{definition}
\medskip

In~\cite{Wang15lori,Wang2016}, formulas are interpreted
over \emph{labeled transition systems} (\ltss): relational
models in which each (basic) relation indicates the source and target
of a particular type of action the agent can perform.
In the setting introduced in~\cite{AFSVQ21}, \lts{s} are extended with a notion of \emph{uncertainty} between plans.

% \begin{definition}[Labeled Transition Systems]\label{def:abmap}
%   A \emph{labeled transition system (LTS)} is a tuple
%   $\modlts=\tup{\W,\R,\V}$ where $\W$ is a non-empty set of states (called the domain and
%   denoted by $\D{\lts}$),
%   $\R = \setof{\R_a \subseteq \W \times \W}{a \in \ACT}$ is a
%   collection of binary relations on $\W$, and $\V:\W \to 2^\PROP$ is a
%   labeling function. For $w\in\W$, the pair $(\modlts,w)$ is a
%   \emph{pointed LTS}, with parentheses usually dropped.
% \end{definition}
\medskip

\begin{definition}[Actions and plans]
Let $\ACT^*$ be the set of finite sequences over $\ACT$. Elements of
$\ACT^*$ are called \emph{plans}, with $\epsilon$ being the
\emph{empty plan}. %Let $\ACT^+ := \ACT^* \setminus  \set{\epsilon}$.
Given $\plan \in \ACT^*$, we use $\card{\plan}$ to denote its length (in particular, $\card{\epsilon} = 0$). For
$0 \le k \le \card{\plan}$, the plan $\plan_k$ is $\plan$'s initial
segment up to (and including) the $k$th position (with
$\plan_0 := \epsilon$). For $0 < k \le \card{\plan}$, the action
$\plan[k]$ is the one in $\plan$'s $k$th position.
\end{definition}

\medskip

\begin{definition}[Uncertainty-based \lts]\label{def:ults}
%Let \AGT be a finite non-empty set of agents.
An \emph{uncertainty-based \lts (\ults)} for $\PROP$, $\ACT$ and $\AGT$ is a tuple $\modults = \tup{\W,\R,\Unc,\V}$ where $\W$ is a non-empty set of states (the domain, also denoted by $\D{\modults}$), $\R = \setof{\R_a \subseteq \W \times \W}{a \in \ACT}$ is a collection of binary relations on $\W$, $\Unc = \setof{\Unc(i) \subseteq 2^{\ACT^*}\setminus \set{\emptyset}}{i \in \AGT}$ assigns to every agent a non-empty collection of pairwise disjoint non-empty sets of plans (i.e., $\Unc(i) \neq \emptyset$, $\plans_1, \plans_2 \in \Unc(i)$ with $\plans_1 \neq \plans_2$ implies $\plans_1 \cap \plans_2 = \emptyset$, and $\emptyset \notin \Unc(i)$) and $\V:\W \to 2^\PROP$ is the valuation function. The tuple $\tup{\W,\R,\V}$ is called an LTS. Given an \ults $\modults$ and $w \in \D{\modults}$, the pair $(\modults,w)$ (parentheses usually dropped) is called a \emph{pointed \ults}.
\end{definition}
\medskip


Intuitively, $\DS{i} = \bigcup_{\plans \in \Unc(i)} \plans$ is the set of plans agent $i$ has at her disposal (alternatively, is aware of), and each $\plans \in \Unc(i)$ is an indistinguishability class. As discussed in~\cite{AFSVQ21}, there is a one-to-one correspondence between each $\Unc(i)$ and an indistinguishability relation ${\sim_i} \subseteq \DS{i} \times \DS{i}$ describing the agent's \emph{uncertainty} over her available plans ($\plan_1 \sim_i \plan_2$ iff there is $\plans \in \Unc(i)$ such that $\set{\plan_1, \plan_2} \subseteq \plans$). The presentation used here simplifies some of the definitions that will follow.

Given her uncertainty over (a subset of) $\ACT^*$, the epistemic abilities of an agent $i$ depend not on what a single plan can achieve, but rather on what a set of them can guarantee.

\medskip

\begin{definition}
Given $\R=\setof{\R_a \subseteq \W \times \W}{a\in\ACT}$ and $\plan \in \ACT^*$, define $\R_\plan \subseteq \W \times \W$  inductively as: $\R_\epsilon := \set{(w,w) \mid w \in W}$ and $\R_{{\plan}a} := \R_\plan \circ \R_a$ (first $\R_{\plan}$ and then $\R_a$). Then, for $\plans \subseteq \ACT^*$ and $U \cup \{u \} \subseteq \W$, define $\R_\plans := \bigcup_{\plan \in \plans} \R_{\plan}$, $\R_{\plans}(u) := \bigcup_{\plan \in \plans} \R_\plan(u)$, and $\R_{\plans}(U) := \bigcup_{u \in U} \R_{\plans}(u)$.
\end{definition}

\medskip

In what follows, we introduce the notion of strong executability of plans (see, e.g.,~\cite{Wang15lori,AFSVQ23report} for further discussions), a condition which determines that a given plan (or a set of them) is appropriate for achieving a certain goal.

\medskip


\begin{definition}[Strong executability of plans]\label{def:plans-exec}
Let $\modults=\tup{\W, \R, \Unc, \V}$ be an $\ults$, with $\R=\setof{\R_a\subseteq \W \times \W}{a\in\ACT}$. A \emph{plan} $\plan \in \ACT^*$ is \emph{strongly executable} (SE) at $u \in \W$ if and only if $v \in \R_{\plan_k}(u)$ implies $\R_{\plan[k+1]}(v) \neq \emptyset$ for every $k \in \intint{0}{\card{\plan}-1}$.
The set $\stexec^\modults(\plan):= \set{w\in\W \mid \plan \mbox{ is SE at }w}$ contains the states in $\modults$ where $\plan$ is SE. Then, a \emph{set of plans} $\plans \subseteq \ACT^*$ is \emph{strongly executable} at $u \in \W$ if and only if \emph{every} plan $\plan \in \plans$ is \emph{strongly executable} at~$u$.
The set $\stexec^\modults(\plans) = \bigcap_{\plan \in \plans} \stexec^\modults(\plan)$ contains the states in $\modults$ where $\plans$ is SE.
\end{definition}
\medskip


Thus, while \emph{a plan} is strongly executable (at a state) when \emph{all its partial executions} (from that state) %(including~$\epsilon$) 
can be completed, \emph{a set of plans} is strongly executable when \emph{all its plans} are strongly executable. When the model is clear from context, we drop the superscript $\modults$, simply writing $\stexec(\plan)$ and $\stexec(\plans)$.

% \smallskip

Now, we have all the ingredients to define the semantics for $\KHilogic$.
\medskip

\begin{definition}\label{def:sem-esm}
Let $\modults=\tup{\W,\R,\Unc,\V}$ be an \ults and $w \in \W$.
The satisfiability relation $\models$ for $\KHilogic$ is inductively defined as:
\begin{spcenter}
$\begin{array}{l@{\ \ \ }c@{\ \ \  }l}
\modults, w \models p & \text{iff} & p \in \V(w) \\
\modults, w \models \neg\varphi & \text{iff} & \modults, w \not\models \varphi \\
\modults, w \models \psi\vee\varphi & \text{iff} & \modults, w \models \psi \mbox{ or }\modults, w \models \varphi \\
\modults, w \models \khi(\psi,\varphi) & \text{iff} & \text{there is } \plans \in \Unc(i) \;\text{such that:} \\
& & \ \ \text{\bfseries \itshape (i)} \; \truthset{\modults}{\psi} \subseteq \stexec(\plans),\; \text{and} \\
& & \ \ \text{\bfseries \itshape (ii)} \; \R_\plans(\truthset{\modults}{\psi}) \subseteq \truthset{\modults}{\varphi},
\end{array}$
\end{spcenter}
with $\truthset{\modults}{\chi} := \setof{w\in\W}{\modults,w\models\chi}$. Any $\plans$ making true the existential statement in the semantic clause of $\khi(\psi,\varphi)$ is called a \emph{witness} for $\khi(\psi,\varphi)$. Define $\modults\models\varphi$ iff  $\truthset{\modults}{\varphi}=\W$, and $\models\varphi$ iff $\modults\models\varphi$, for all \ults $\modults$. These notions are extended as expected for all the logics in the rest of the article.
\end{definition}

\medskip

Some comments are useful here. First note how the modality $\A$ (respectively, $\E$), defined by abbreviation in~\Cref{def:khsyntax}, is actually the universal (respectively, existential) \emph{global modality} from, e.g.,~\cite{GorankoP92}. Indeed, for every model $\modults$ and every state $w$, $\modults,w\models\A\varphi$ ($\modults,w\models\E\varphi$) holds if and only if $\varphi$ is true in every (some) state in $\modults$ (see~\cite{AFSVQ21,AFSVQ23report} for details). %This will be of use in the rest of the paper.
Second, notice that $\KHilogic$ is a very `simple' language. In particular, even though the $\khi$ modality has fairly complex requirements for its witness $\plans$, the language is `blind' to the actual actions that appear in the semantics.  Indeed, as we are going to discuss below, the sets of plans in a given model can be drastically changed without affecting the agents' abilities as described by the language. 

\medskip

% \begin{example}\label{ex:aircraft}
%     An unmanned aircraft (the agent $i$) needs to fly from safe zones (states
%     labeled by the propositional symbol $s$) to safe zones while avoiding
%     turbulent     areas ($t$-states). There are two actions: fly to
%     the west ($w$) or fly to the east ($e$). Order of the actions
%     matters: from the leftmost state in the \ults $\modults$ below (states
%     showing only atoms true at them), a plan $we$ (first west then east)
%     leads to safe zones, but $ew$ does not. To reach a safe
%     zone from a turbulent one, the aircraft needs to pass through a
%     non-turbulent zone first. This is achieved by flying east again, and
%     finally west.

%       \begin{nscenter}
%       \begin{tikzpicture}[->]
%         \node [state] (w1) {$s$};
%         \node [state, below right = of w1] (w2) {};
%         \node [state, right = of w2] (w3) {$s$};
%         \node [state, above = of w3] (w4) {};
%         \node [state, above right = of w1] (w5) {$t$};
%         \node [state, right = of w5] (w6) {};

%         \path (w1) edge[bend right] node [label-edge, below] {$w$} (w2)
%               (w2) edge[bend right] node [label-edge, above] {$e$} (w1)
%               (w1) edge node [label-edge, above] {$e$} (w5)
%               (w2) edge[bend right] node [label-edge, below] {$e$} (w3)
%               (w3) edge[bend right] node [label-edge, above] {$w$} (w2)
%               (w4) edge node [label-edge, right] {$w$} (w3)
%               (w5) edge node [label-edge, below] {$e$} (w4)
%                    edge node [label-edge, above] {$w$} (w6);
%       \end{tikzpicture}
%       \begin{picture}(90,0)
%       \put(10,40){$\Unc(i) = \left\{
%         \begin{array}{c}
%         \{ew\},\\
%         \{we, eew\}
%         \end{array}
%       \right\}$}
%       \end{picture}\vspace*{-3mm}
%     \end{nscenter}
%     The diagram shows, on the right, the set of indistinguishable actions in $\Unc(i)$, containing two sets: $\plans_1=\set{ew}$ and
%     $\plans_2=\set{we,eew}$. Notice that $\modults\models\khi(s,s)$,
%     i.e., the aircraft knows how to reach a safe zone, given it is at a
%     safe zone: for each $s$-state, there is a plan in $\plans_2$
%     which leads the aircraft only to $s$-states. This is due to the fact
%     that the agent \emph{distinguishes} $ew$ from the other plans.
%   \end{example}

\begin{example}\label{ex:cook}
Consider a simple scenario in which two agents $i$ and $j$ attempt to bake a good cake (represented by~$g$). Suppose they follow a similar recipe, and they have all the ingredients ($h$). The recipe states that $g$ is achieved via the following steps: adding eggs ($e$), beating the eggs ($b$), adding flour ($f$), adding milk ($m$), stirring these ingredients ($s$) and finally baking the preparation ($p$). Thus, the plan needed to achieve $g$ is $\mathit{ebfmsp}$, whenever the agents have all the ingredients ($h$). Agent $i$, an experienced chef, is aware that this is the way to get a good cake. On the other hand, agent $j$  has no cooking experience, so she considers that the order of the steps does not matter (e.g., she thinks she can add milk before adding the flour).
\begin{spcenter}
\hspace*{-1cm}\begin{tikzpicture}[->]
    \node [state] (w1) {$h$};
    \node [left  = 0.35cm of w1] (m) {$\modults$};
    \node [state, right = of w1] (w2) {\phantom{$h$}};
    \node [state, right = of w2] (w3) {\phantom{$h$}};
    \node [state,  right = of w3] (w4) {\phantom{$h$}};
    \node [state, right = of w4] (w5) {$g$};

    \path (w1) edge[right] node [label-edge, above] {$e$} (w2)
        %(w2) edge[bend right] node [label-edge, above] {$e$} (w1)
        (w2) edge[->,loop above] node {\small$b$} (w2)
        (w2) edge[right] node [label-edge, above] {$f$} (w3)
        (w3) edge[right] node [label-edge, above] {$m$} (w4) 
        (w4) edge[->,loop above] node {\small$s$} (w4)
        (w4) edge node [label-edge, above] {$p$} (w5);
        %(w5) edge node [label-edge, below] {$e$} (w4)
            %    edge node [label-edge, above] {$w$} (w6);
\end{tikzpicture}
\begin{picture}(90,0)
    \small
\put(15,20){$\Unc(i) = \left\{
    \begin{array}{c}
    \{\mathit{ebfmsp}\}
    \end{array}
\right\}$}
\put(15,5){$\Unc(j) = \left\{
    \begin{array}{c}
    \{\mathit{ebfmsp},\mathit{ebmfsp}\}
    \end{array}
\right\}$}
\end{picture}
\end{spcenter}
The diagram shows, on the right, the set of indistinguishable plans in $\Unc(i)$ and in~$\Unc(j)$. %, containing two sets: $\plans_1=\set{ew}$ and   $\plans_2=\set{we,eew}$.
Agent $i$ knows how to bake a good cake, provided she has all the ingredients (i.e., $\modults\models\khi(h,g)$). This is because agent $i$ \emph{distinguishes} $\mathit{ebfmsp}$ as the ``good plan''. On the other hand, $j$ considers that adding milk and adding flour can be done in any order (i.e., $\mathit{ebfmsp}$ and $\mathit{ebmfsp}$ are indistinguishable), leading to $\modults\not\models\kh_j(h,g)$, as the plan
$\mathit{ebmfsp}$ is not strongly executable in the model.
\end{example}

\medskip 

%\ssparagraph{Bisimulations.} 
The notion of \emph{bisimulation} is a crucial tool for understanding the expressive power of a modal language. Here we recall the notion of bisimulation for $\KHilogic$ over \ultss~\cite{AFSVQ23report}. We start by providing some useful abbreviations.

\medskip

\begin{definition}\label{def:notation}
Let $\modults=\tup{\W,\R,\Unc,\V}$ be an \ults over \PROP, \ACT and \AGT. Take $\plans \in 2^{(\ACT^*)}$, $U, T \subseteq \W$ and $i \in \AGT$.
\begin{itemize} \itemsep 0pt
    \item Write $U \ultsExecStrat{\plans} T$ $\siffdefs$ $U \subseteq \stexec(\plans)$ and $\R_{\plans}(U) \subseteq T$.

    \item Write $U \ultsExecAgi T$ $\siffdefs$ there is $\plans \in \Unc(i)$ such that $U \ultsExecStrat{\plans} T$.
\end{itemize}
Additionally, $U \subseteq \W$ is propositionally definable in $\modults$ if and only if there is a propositional formula $\varphi$ such that $U = \truthset{\modults}{\varphi}$.
\end{definition}

\medskip

Now we are ready to introduce the notion of bisimulation for $\KHilogic$. 

\medskip 

\begin{definition}[$\KHilogic$-bisimulation]\label{def:bisim-khi}
Let $\modults = \tup{\W,\R,\Unc,\V}$ and $\modults' = \tup{\W',\R',\Unc',\V'}$ be $\ultss$.
%Take $Z \subseteq \W \times \W'$, $u \in \W$, $U \subseteq \W$, $u' \in \W'$, and $U' \subseteq \W'$.
%\begin{itemize}\itemsep 0cm
%  \item Let %$Z(u)$, $Z(U) \subseteq \W'$ as
%      \begin{nscenter}
%        \begin{tabular}{@{}c@{}}
%          $Z(u) := \setof{u' \in \W'}{uZu'}$, $Z(U) := \bigcup_{u \in U} Z(u)$.
%        \end{tabular}
%      \end{nscenter}
%
%  \item Let % $Z^{-1}(u)$, $Z^{-1}(U) \subseteq \W$ as
%      \begin{nscenter}
%        \begin{tabular}{@{}c@{}}
%          $Z^{-1}(u') := \setof{u \in \W}{uZu'}$; $Z^{-1}(U') := \bigcup_{u' \in U'} Z(u')$.
%        \end{tabular}
%      \end{nscenter}
%    \end{itemize}
A non-empty $Z \subseteq \W \times \W'$ is called an $\KHilogic$-bisimulation between $\modults$ and $\modults'$ if and only if $wZw'$ implies all of the following:
\begin{itemize} 
    \item \textbf{Atom}: $\V(w)=\V'(w')$.

    \item \textbf{$\khi$-Zig}: for any \emph{propositionally} definable $U \subseteq \W$, if $U \ultsExecAgi T$ for some $T \subseteq \W$, then there is $T' \subseteq \W'$ such that: 
%      \begin{nscenter}
%      \begin{multicols}{2}
%        \begin{enumerate}
        1) $Z(U) \ultsExecAgi T'$, and
        2) $T' \subseteq Z(T)$.
%        \end{enumerate}
%      \end{multicols}
%    \end{nscenter}

    \item \textbf{$\khi$-Zag}: % analogous to \textbf{$\khi$-Zig}.
    for any \emph{propositionally} definable $U' \subseteq \W'$, if $U' \ultsExecAgi T'$ for some $T' \subseteq \W'$, then there is $T \subseteq \W$ such that: 
%      \begin{nscenter}
%      \begin{multicols}{2}
%        \begin{enumerate}
        1) $Z^{-1}(U') \ultsExecAgi T$, and
        2) $T \subseteq Z^{-1}(T')$.
%        \end{enumerate}
%      \end{multicols}
%    \end{nscenter}

    \item \textbf{$\A$-Zig}: for all $u\in\W$ there is a $u'\in\W'$ such that $uZu'$.

    \item \textbf{$\A$-Zag}: for all $u'\in\W'$ there is a $u\in\W$ such that $uZu'$.
\end{itemize}
We write $\modults,w \bisim \modults',w'$ when there is an
$\KHilogic$-bisimulation $Z$ between $\modults$ and $\modults'$ such that
$wZw'$.
\end{definition}

\medskip

The following theorem establishes a classical adequacy result.

\medskip

\begin{theorem}[\cite{AFSVQ23report}]
Let $\modults,w$ and $\modults',w'$ be two \ultss. $\modults,w\bisim\modults',w'$ implies $\modults,w\models\varphi$ iff $\modults',w'\models\varphi$, for all $\KHilogic$-formulas $\varphi$. %Moreover, if $\modults$ and $\modults'$ are finite, the converse also holds.
\end{theorem}

\medskip 

% \ssparagraph{Axiomatization.} 

Another important element to consider is the axiom system for $\KHilogic$~\cite{AFSVQ21,AFSVQ23report}. 

\begin{table}[t]
\begin{tabular}{l@{\quad \quad  }l@{\quad}l}
\toprule
$\mbox{Axioms}$
& \axm{Taut}  & $\vdash \varphi \mbox{ for $\varphi$ a propositional tautology}$ \\
& \axm{DistA} & $\vdash \A(\varphi\ra\psi) \ra (\A\varphi \ra \A\psi)$ \\

& \axm{TA}    & $\vdash \A\varphi \ra \varphi$ \\
& \axm{4KhA}  & $\vdash \khi(\psi,\varphi) \ra \A\khi(\psi,\varphi)$ \\
& \axm{5KhA}  & $\vdash \neg\khi(\psi,\varphi) \ra \A\neg\khi(\psi,\varphi)$ \\
& \axm{KhA}   & $\vdash \left(\A(\chi \rightarrow \psi) \land \khi(\psi,\varphi) \land \A(\varphi \rightarrow \theta)\right) \rightarrow \khi(\chi, \theta)$ \\
& \axm{G} & $\vdash \khi(\varphi,\bot) \ra \kh_j(\varphi,\bot)$ \\
\midrule
\mbox{Rules}
&  \axm{MP}   & $\mbox{From $\vdash \varphi$ and $\vdash \varphi \rightarrow \psi$ infer $\vdash \psi$ }$ \\
&  \axm{NecA} & $\mbox{From $\vdash \varphi$ infer $\vdash \A\varphi$}$ \\
\bottomrule
\end{tabular}
\caption{Axiomatization $\axset_{\khi}$ for $\KHilogic$ w.r.t.\ $\ultss$.}\label{tab:khiaxiom}
\end{table}

\medskip

\begin{theorem}[\cite{AFSVQ21,AFSVQ23report}]\label{th:khi-completeness}
The axiom system from~\Cref{tab:khiaxiom} is sound and strongly complete with respect to the class of all \ultss.
\end{theorem}

\medskip

It is worth noticing two differences between the system in \Cref{tab:khiaxiom} and the one presented in~\cite{AFSVQ23report}. First, the formula $\left(\E\psi \land \khi(\psi,\varphi)\right) \rightarrow \E\varphi$, called \axm{KhE} and part of the axiom system presented in \cite{AFSVQ23report}, is omitted here. This is because it is derivable in the system, as proved below by showing the derivability of the equivalent formula $(\E\psi \wedge \khi(\psi,\varphi) \wedge \A\neg\varphi) \ra \bot$. One can start with
% , $\left(\E\psi \land \khi(\psi,\varphi) \land \A \neg\varphi \right) \rightarrow \bot$, using \axm{KhA}. %\raul{To write down the syntactic proof in detail here.}
%
% By \axm{TAUT}, it is equivalent to prove that
%
%
% \begin{equation}\label{eq:khe0}
% \vdash (\E\psi \wedge \khi(\psi,\varphi) \wedge \A\neg\varphi) \ra \bot.
% \end{equation}
%
% For this, we establish a series of implications and, using transitivity, we get \axm{KhE}. By applying \axm{TAUT}, we have that
%
\begin{spcenter}
\begin{small}
$\begin{array}{lll}
%\begin{align*}
                  & \vdash (\E\psi \wedge \khi(\psi,\varphi) \wedge \A\neg\varphi) \ra (\E\psi \wedge \khi(\psi,\varphi) \wedge \A\neg\varphi)                                 & \axm{TAUT} \\ 
  \Leftrightarrow & \vdash (\E\psi \wedge \khi(\psi,\varphi) \wedge \A\neg\varphi) \ra (\E\psi \wedge \A(\psi \ra \psi) \wedge \khi(\psi,\varphi) \wedge \A(\varphi \ra \bot)) & \axm{TAUT},\axm{NECA}
%\end{align*}
\end{array}$
\end{small}
\end{spcenter}
%
%
% \begin{equation*}
% \vdash (\E\psi \wedge \khi(\psi,\varphi) \wedge \A\neg\varphi) \ra (\E\psi \wedge \khi(\psi,\varphi) \wedge \A\neg\varphi).
% \end{equation*}
% %
% Taking \axm{TAUT} ($\vdash \neg\varphi \lra (\varphi \ra \bot)$, $\vdash \psi \ra \psi$) and \axm{NECA}, we replace $\A\neg\varphi$ by $\A(\varphi \ra \bot)$ and add $\A(\psi \ra \psi)$, which is a tautology. Thus,
% %
% \begin{equation}\label{eq:khe1}
% \vdash (\E\psi \wedge \khi(\psi,\varphi) \wedge \A\neg\varphi) \ra (\E\psi \wedge \A(\psi \ra \psi) \wedge \khi(\psi,\varphi) \wedge \A(\varphi \ra \bot)).
% \end{equation}
%
Then, by using an instance of \axm{KhA}, we have that
\begin{spcenter}
\begin{small}
$\begin{array}{lll}
%\begin{align*}
    & \vdash (\A(\psi \ra \psi) \wedge \khi(\psi,\varphi) \wedge \A(\varphi \ra \bot)) \ra (\khi(\psi,\bot)) & \\
    \Leftrightarrow &  \vdash (\E\psi \wedge \A(\psi \ra \psi) \wedge \khi(\psi,\varphi) \wedge \A(\varphi \ra \bot)) \ra (\E\psi \wedge \khi(\psi,\bot)) & \axm{TAUT} \\
    \Leftrightarrow & \vdash (\E\psi \wedge \A(\psi \ra \psi) \wedge \khi(\psi,\varphi) \wedge \A(\varphi \ra \bot)) \ra (\neg\A\neg\psi \wedge \A\neg\psi) & \mbox{Def. } \A,\E \\
    \Leftrightarrow & \vdash (\E\psi \wedge \A(\psi \ra \psi) \wedge \khi(\psi,\varphi) \wedge \A(\varphi \ra \bot)) \ra \bot & \mbox{Def. } \bot
%\end{align*}
\end{array}$
\end{small}
\end{spcenter}
%
% \begin{equation*}
% \vdash (\A(\psi \ra \psi) \wedge \khi(\psi,\varphi) \wedge \A(\varphi \ra \bot)) \ra (\khi(\psi,\bot)).
% \end{equation*}
% %
% Adding $\E\psi$ to both sides (by using an instance of \axm{TAUT}),
% %
% \begin{equation*}
% \vdash (\E\psi \wedge \A(\psi \ra \psi) \wedge \khi(\psi,\varphi) \wedge \A(\varphi \ra \bot)) \ra (\E\psi \wedge \khi(\psi,\bot)).
% \end{equation*}
% %
% By definition of $\E$ y $\A$,
% %
% \begin{equation*}
% \vdash (\E\psi \wedge \A(\psi \ra \psi) \wedge \khi(\psi,\varphi) \wedge \A(\varphi \ra \bot)) \ra (\neg\A\neg\psi \wedge \A\neg\psi).
% \end{equation*}
% %
% And by definition of $\bot$,
% %
% \begin{equation}\label{eq:khe2}
% \vdash (\E\psi \wedge \A(\psi \ra \psi) \wedge \khi(\psi,\varphi) \wedge \A(\varphi \ra \bot)) \ra \bot.
% \end{equation}
%
Given the implications proved above, \axm{KhE} follows by transitivity.

\begin{mrevised}The second difference  is the presence of the axiom \axm{G} (standing for `global' formulas).  This axiom indicates that the validity of formulas of the form $\khi(\varphi,\bot)$ is shared among all agents.  Notice that these formulas are particular, as they describe conditions that would lead to the impossible case of a goal satisfying $\bot$.  The axiom states that these formulas do not depend on the particular agent under consideration. 
%	It states that knowing how formulas that are trivially validated for an agent (in the sense that as the goal to be achieved is impossible, ), are also trivially validated for other agents. In a sense, some form of `general knowledge' is preserved between agents. 
This axiom needs to be added as a consequence of a gap in the arguments  in~\cite[Prop.~4]{AFSVQ23report}. The validity called \axm{SCond} cannot be proved without axiom \axm{G} (since the universal modality $\A$ is defined in that article as a disjunction, the last step in~\cite[Prop.~4]{AFSVQ23report} is not an equivalence). By adding \axm{G} here, we fix this issue.
\end{mrevised}

% \ssparagraph{Complexity.} 

Strong completeness of the axiomatization above is established in~\cite{AFSVQ21,AFSVQ23report} via a canonical model construction (modulo the correction just mentioned). In addition, starting from the canonical model, a careful selection function can be used to prove a small (polynomial) model property that leads to the following complexity results.

\medskip 

\begin{theorem}[\cite{AFSVQ21,AFSVQ23report}]
    The model-checking problem for $\KHilogic$ is in \Poly, while the satisfiability problem for $\KHilogic$ is \NP-complete.
\end{theorem}


\medskip

We mentioned before that the semantic interpretation for $\khi$ is `blind' to certain aspects of its witness. In fact, as the completeness proof in~\cite{AFSVQ21} shows, it cannot distinguish between the class of all arbitrary \ults and the class in which, for every agent $i$, every set in $\Unc(i)$ is a set of one-step plans (i.e., a set of basic actions). Formally, we can define this class as follows: 
\medskip 

\begin{definition}\label{def:class-m-one}
	Define $\cultsba$ (\textbf{BA} stands for `basic actions') as the class of models $\modults = \tup{\W, \R, \Unc, \V}$ in which, for all $i \in \AGT$, we have that $\plans \in \Unc(i)$ implies $\plans \subseteq \ACT$.
\end{definition}

\medskip 

$\cultsba$ (denoted $\sults$ in~\cite{AFSV22}) is a restricted class of models, which could be interpreted as a more abstract representation of the actions available to agents. In this class, every plan is modeled as a single atomic action (similar to what is done in formal verification using the so-called `path abstraction' technique, see e.g.~\cite{AbrahamJWKB10} for an example). 

\medskip	
	
\begin{proposition}
	A formula in $\KHilogic$ is valid over the class of all $\ults$ if and only if 
	it is valid over $\cultsba$.
\end{proposition}	
\medskip

The last proposition can be interpreted as a limitation on the expressivity of $\KHilogic$: since actions do not appear in its syntax, the logic is unable to distinguish between models in which witnesses are arbitrary plans and those in which they are all single atomic actions.


% Moreover, a small (polynomial) model property can be proved using selection functions and, as a corollary, the satisfiability and model checking problems of $\KHilogic$ are \NP-complete and \Poly, respectively~\cite{AFSVQ21,AFSVQ23report}. As a consequence, we have the following corollary.

% \medskip

% \begin{corollary}\label{cor:satcultsker} \carlos{Este resultado esta fuera de lugar acá. No se estiende su relación con complejidad ni su utilidad.}
% Let $\cultsfnu $ be the class of all finite \ultss such that the agents consider only basic actions distinguishable from each other (\textbf{FNU}): \raul{there are two notations, \textbf{NU} and \textbf{FNU}. Are both necessary? or are they always used together?}
% \[
% \cultsfnu := \setof{\modults}{\modults \text{ is a finite \ults and for all } i \in \AGT,\ \Unc(i) \subseteq  \setof{\set{a}}{a \in \ACT}}.
% \]
% Then, $\varphi$ is satisfiable if and only if $\varphi$ is satisfiable in $\cultsfnu$, i.e., there is $\modults \in \cultsfnu$ and $w \in \D{\modults}$ such that $\modults,w \models \varphi$.
% \end{corollary}
