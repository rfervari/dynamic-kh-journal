% Let us start by consider \PAL-like announcements in the context of knowing how logics. For the $\kh$ modality first introduced in~\cite{Wang15lori}, such an announcement would result in updating the labeled transition system in which the formula is interpreted, producing a change in the abilities of the agent. In that setting, these abilities coincide with her knowledge.

%As discussed in the previous sections, our goal is to take ideas from standard epistemic logic in order to perform operations that update an agent's knowing how.

A typical way of updating a model  
is by removing states, as is done in \PAL~\cite{Plaza89:lopc,DELbook}.  This model update is typically realized with the public announcement operator $[\chi]$,
which, over an $\ults$ $\modults=\tup{\W,\R,\set{\Unc(i)}_{i\in\AGT},\V}$, can be defined as:
\[
\modults,w\models[\chi]\varphi \ \ \iff \ \  \modults,w\models\chi \mbox{ implies } \modults_{\chi},w\models\varphi,
\]
with $\modults_\chi=\tup{\W_\chi,\R_\chi,\Unc_\chi,\V_\chi}$ where:
\[
\begin{array}{l@{\quad \quad \quad }l}
\W_\chi = \set{w \mid \modults,w\models\chi} & (\R_\chi)_a = \R_a \cap \W_\chi^2, \mbox{ for each } a\in\ACT \\
\V_\chi(w) = \V(w), \mbox{ for all } w\in\W_\chi & \Unc_\chi = \set{\Unc(i)}_{i\in\AGT}.
\end{array}	
\]

As we can see, $\modults_\chi$ is obtained by taking $\truthset{\modults}{\chi}$ as the new domain, and by restricting relations and the valuation accordingly (the indistinguishability classes $U(i)$ are left unchanged).
In the original \emph{knowing how} setting from~\cite{Wang15lori}, the relations define the agent's abilities. Thus, under that semantics this kind of update corresponds to both an ontic and an epistemic change (available actions change, and hence so do the agent's abilities). However, in the \ults-based semantics, relations provide only ontic information; thus, this update operation produces an \emph{ontic} change, but not an epistemic one.

We can prove that this update operator adds expressivity to $\KHilogic$ (a similar result was established in~\cite{Wang2016} for a $\kh$ modality with intermediate constraints).

\medskip 

\begin{proposition}\label{prop:pal-exp}
Adding $[\chi]$ to $\KHilogic$ increases its expressive power.
\end{proposition}

\begin{proof}
  The two \ultss $\modults$ and $\modults'$ (with
  $\Unc(i)=\Unc'(i)=\set{\set{a}})$ below are bisimilar and hence
  indistinguishable in $\KHilogic$. 

\begin{center}
    \begin{tikzpicture}[->]
      \node [state,label=left:\small$w$] (w1) {$p,q$};
      \node [state, below = of w1] (w2) {$p,q$};
      \node [state, dashed, right = of w1] (w3) {\phantom{$p$}};
      \node [state, below = of w3] (w4) {$p$};
      \node[left = of w1] (m) {$\modults$};

      \path (w1) edge node [label-edge, left] {$a$} (w2)
             (w2) edge[loop left] node [label-edge, left] {$a$} (w2)
             (w1) edge[dashed] node [label-edge, above] {$a$} (w3)
             (w4) edge node [label-edge, below] {$a$} (w2);

        \node[right = of w3] (phan) {};
        \node [state,right = of phan,label=left:\small$w'$] (w1) {$p,q$};
        \node [state, below = of w1] (w2) {$p,q$};
        \node [state, dashed, right = of w1] (w3) {\phantom{$p$}};
        \node [state, below = of w3] (w4) {$p$};
        \node[right = of w3] (mp) {$\modults'$};

        \path (w2) edge[loop left] node [label-edge, left] {$a$} (w2)
            (w1) edge[dashed] node [label-edge, above] {$a$} (w3)
            (w4) edge node [label-edge, below] {$a$} (w2);

    \end{tikzpicture}
  \end{center}
However, $\modults,w\models[p]\khi(p,q)$ whereas
$\modults',w'\not\models[p]\khi(p,q)$. Dashed lines indicate nodes and edges removed after $[p]$.
\end{proof}

A consequence of~\Cref{prop:pal-exp} is that the modality for \PAL-like updates is not reducible to the base logic. This makes sense, as the underlying static logic ($\KHilogic$) only expresses properties relative to the existence of a way to achieve certain target states from certain origin states. There is no way to characterize the updates produced by $[\chi]$ with the expressive power provided by the $\khi$ modality. This is in contrast with what happens when these modalities are added to standard epistemic logic, where reduction axioms can be defined (see, e.g.,~\cite{DELbook}).


Is it possible to define an alternative, \PAL-like update operator, for which reduction axioms exists in $\KHilogic$? We will answer this question below.

\medskip 

\begin{definition}\label{def:pakhsyntax}
Formulas of the language $\PAKHilogic$ are given by
\[
	\varphi ::= p \mid \neg\varphi \mid \varphi\vee\varphi \mid
\khi(\varphi,\varphi)\mid \gann{\varphi}\varphi,
\]
with $p \in \PROP$ and $i\in\AGT$.
%Other Boolean connectives are defined as usual. We define $\tup{!\psi}\varphi := \neg\gann{\psi}\neg\varphi$.
\end{definition}

\medskip 

\begin{definition}\label{def:annupdate}
Let $\modults = \tup{\W, \R, \Unc, \V}$ be an \ults, and let $\chi$ be a $\PAKHilogic$-formula. We define $\annmodelchi = \tup{\annsemantics{\W}{\chi}, \annsemantics{\R}{\chi}, \annsemantics{\Unc}{\chi}, \annsemantics{\V}{\chi}}$, where:
\begin{itemize}
\item $\annsemantics{\W}{\chi} = \truthset{\modults}{\chi}$,
\item $(\annsemantics{\R}{\chi})_a = \setof{(w,v) \in \R_a}{w \in \truthset{\modults}{\chi} \text{, } \R_a(w) \subseteq \truthset{\modults}{\chi}}$ for every $a \in \ACT$,
\item $\annsemantics{\Unc}{\chi} = \Unc$, and 
\item $\annsemantics{\V}{\chi}(w) = \V(w)$ (for all $w\in\annsemantics{\W}{\chi}$).
\end{itemize}

We extend the satisfaction relation $\models$ from~\Cref{def:sem-esm} with the case:
\[
	\modults,w \models \gann{\chi}\varphi \ \ \iff \ \ \modults,w \models \chi \mbox{ implies } \annmodelchi,w \models \varphi.
\]
\end{definition}


The only difference between the $\annmodelchi$ introduced above and the standard $\modults_{\chi}$ (which is the restriction of $\modults$ to the states satisfying $\chi$) is in the
definition of the relations. In the proposal here, a stronger condition is needed for an $a$-edge from a state $w \in \truthset{\modults}{\chi}$ to survive after the update: if $\R_a(w) \not \subseteq \truthset{\modults}{\chi}$ then $(\annsemantics{\R}{\chi})_a(w) = \emptyset$, but if $\R_a(w) \subseteq \truthset{\modults}{\chi}$ then $(\annsemantics{\R}{\chi})_a(w) = \R_a(w)$. Notice that in this context, the elimination of some states indicates that the situations they describe are no longer \emph{reachable}, rather than no longer \emph{possible}
\footnote{The two forms of model update discussed here bear a resemblance to the two forms of updating neighbourhood models from~\cite{MaS18}. Recall that a neighbourhood model~\cite{Scott1970,Montague1970} is given by: a non-empty domain $\W$, an atomic valuation, and a neighbourhood function $\N:\W \to 2^{2^{\W}}$, assigning a set of sets of states to each possible state. Let $U \subseteq \W$ be a non-empty set of states. On the one hand, the \emph{$U$-intersection} submodel defined in~\cite{MaS18} has $U$ as its domain, with its neighbourhood function built by restricting each set in a neighbourhood to the new domain, analogous to what $\modults_{\chi}$ (a standard announcement) does.
%\footnote{Formally, given $U \subseteq W$, define $N^{\cap U}(u) := \set{X \cap U \mid X \in N(u)}$ for each $u \in U$.}
On the other hand, the \emph{$U$-subset} submodel therein also has $U$ as its domain, but its neighbourhood function is built by keeping only those sets that are already a subset of the new domain, analogous to what $\annsemantics{\modults}{\chi}$ does. We argue that this second approach is more appropriate in the context of knowing how.}.

Even with this, more restricted, version of update, the resulting logic
fails to have reduction axioms as the following proposition shows.

\medskip 

\begin{proposition}\label{prop:exppal}
	$\PAKHilogic$ is more expresive than $\KHilogic$ over arbitrary \ultss.
\end{proposition}

\begin{proof}
	Let $\modults$ and $\modults'$ be the single agent models depicted below (states and edges depicted with dashed lines are those removed in $\annmodel{r}$ and $\annmodel{r}'$, respectively), with $\Unc(i):=\set{\set{ab}}$ and $\Unc'(i):=\set{\set{a}}$:

	%\centerline{DESCOMENTAR IMAGENES ABAJO!!!}
	\begin{center}
		\begin{tikzpicture}[->, grow' = right, level/.style={sibling distance = 3em/#1}, level distance = 3.5em]
		\node at (0,1) [state,label=left:$w$] (p) {$p,r$} ;
		\node at (2,0) [state,dashed] (nr) {\phantom{$q,r$}};
		\node at (2,2) [state] (q) {$q,r$};
		\node[left = of p] (m) {$\modults$};

		\path (p) edge [dashed, above] node {$a$} (q);
		\path (q) edge [loop right] node [right] {$b$} (q);
		\path (p) edge [dashed,above] node {$a$} (nr);
		\path (nr) edge [dashed,left] node {$b$} (q);
		\end{tikzpicture}
		\hspace{2cm}
		\begin{tikzpicture}[->, grow' = right, level/.style={sibling distance = 3em/#1}, level distance = 3.5em]
		\node[state] at (0,1) (p) [label=left:$w'$]{$p,r$};
		\node[state,dashed] at (2,0) (nr) {\phantom{$\neg r$}};
		\node[state] at (2,2) (q) {$q,r$};
		\node[right = 7em of p] (m) {$\modults'$};
		\path (p) edge [above] node {$a$} (q);
		\path (p) edge [dashed,above] node {$b$} (nr);
		\end{tikzpicture}
\end{center}

	Both models are $\KHilogic$-bisimilar (\Cref{def:bisim-khi}); hence, they satisfy the same formulas in $\KHilogic$. However, $\modults,w \not\models \gann{r}\khi(p,q)$ since $\modults,w \models r$ and $\annmodel{r},w \not\models \khi(p,q)$, whereas $\modults',w' \models \gann{r}\khi(p,q)$ since $\modults',w' \models r$ and $\annmodel{r}',w \models \khi(p,q)$.
	%
	% \begin{nscenter}
	%   \scalebox{.7}{
	% \begin{tikzpicture}[->, grow' = right, level/.style={sibling distance = 3em/#1}, level distance = 3.5em]
	% \node[state] at (0,1) (p) [label=left:$w$]{$p,r$};
	% \node[state] at (2,2) (q) {$q,r$};
	% \path (q) edge [loop above] node {$b$} (q);
	% \end{tikzpicture}
	% \hspace{2cm}
	% \begin{tikzpicture}[->, grow' = right, level/.style={sibling distance = 3em/#1}, level distance = 3.5em]
	% \node[state] at (0,1) (p) [label=left:$w'$]{$p,r$};
	% \node[state] at (2,2) (q) {$q,r$};
	% \path (p) edge [above] node {$a$} (q);
	% \end{tikzpicture}
	%   }
	% \end{nscenter}
\end{proof}

By furthermore restricting the class of models in which we will evaluate formulas, we are able to obtain reasonable reduction axioms.

Note that \ultss contain a set $\Unc(i)$ of sets of plans for each agent $i$, which determines the perception of the agent with respect to her abilities.
For instance, it may be the case that two plans $ab$ and $cd$ belong to some $\plans\in\Unc(i)$, i.e., they are indistinguishable for agent $i$.
In~\cite{AFSVQ21} it has been shown that the logic cannot distinguish between the class of arbitrary \ultss, and the class of models where each $\plans\in\Unc(i)$ is a singleton with $\plans \subseteq \ACT$.
This is no longer the case in the presence of $\gann{\chi}$ (as the proof of~\Cref{prop:exppal} shows).

\medskip 

\begin{definition}\label{def:class-m-one}
Define $\cultsba$ as the class of models $\modults = \tup{\W, \R, \Unc, \V}$ such that for all $i \in \AGT$ and $\plans \in \Unc(i)$, $\plans \subseteq \ACT$.
\end{definition}

\medskip 

The class $\cultsba$ (denoted as $\sults$ in \cite{AFSVQ21}) constitutes a restricted class of models, which could correspond, for example, to a more abstract representation of the abilities of the agents, in which a course of action is modeled as a single action. 
The reduction axioms from~\Cref{tab:palaxiom} are valid in the class of models $\cultsba$. Moreover, we can use them to eliminate announcements by iteratively replacing the innermost occurrence of a $\gann{\chi}$ modality. Thus, we get completeness for $\PAKHilogic$.

\begin{table}[t]
\begin{tabular}{l@{\quad}l}
\toprule
\axm{RAtom} & $\vdash \gann{\chi}p \leftrightarrow (\chi \implies p)$ \\
\axm{R$\neg$} & $\vdash \gann{\chi}\neg\varphi \leftrightarrow (\chi \implies \neg\gann{\chi}\varphi)$ \\
\axm{R$\vee$} & $\vdash \gann{\chi}(\varphi\vee\psi) \leftrightarrow \gann{\chi}\varphi \vee\gann{\chi}\psi$ \\
\axm{RKh} & $\vdash \gann{\chi}\khi(\varphi,\psi) \leftrightarrow (\chi \implies \khi(\chi \wedge \gann{\chi}\varphi,\chi \wedge \gann{\chi}\psi))$ \\
\axm{RE$_{\gann{}}$} & $\text{From } \vdash \varphi \leftrightarrow \psi \text{ derive } \vdash \gann{\chi}\varphi \leftrightarrow \gann{\chi}\psi$ \\
\bottomrule
\end{tabular}
\caption{Reduction axioms $\axset_{\PAKHilogic}$.}\label{tab:palaxiom}
\end{table}

\medskip 

In order to establish the validity of the reduction axioms, we first need to prove the following semantic properties.

\medskip 

\begin{lemma}\label{lem:palproperties} Let $\chi$ and $\varphi$ be $\PAKHilogic$-formulas, the following equalities hold:
	\begin{enumerate}
	\item $\truthset{\modults}{\gann{\chi}\varphi} = \truthset{\modults}{\neg\chi} \cup \truthset{\annmodelchi}{\varphi}$.
	\item $\truthset{\annmodelchi}{\varphi} = \truthset{\modults}{\chi \wedge \gann{\chi}\varphi}$.
	\end{enumerate}
	\end{lemma}
	
	\begin{proof}
 For Item 1, let $w \in \truthset{\modults}{\gann{\chi}\varphi}$, we have that $\modults,w \models \gann{\chi}\varphi$. Thus, $\modults,w \models \chi$ implies $\annmodelchi,w \models \varphi$, which can be rewritten as $\modults,w \models \neg\chi$ or $\annmodelchi,w \models \varphi$. Hence, $w \in \truthset{\modults}{\neg\chi} \cup \truthset{\annmodelchi}{\varphi}$. The other inclusion behaves the same way.
 
 
For Item 2, let $w \in \truthset{\annmodelchi}{\varphi}$, we have that $w \in \annsemantics{\W}{\chi} = \truthset{\modults}{\chi}$ and $w \in \truthset{\annmodelchi}{\varphi}$. Thus, $w \in \truthset{\modults}{\chi} \cap \truthset{\annmodelchi}{\varphi}$. The previous set is in the form of $(A \cap B)$. By set reasoning, it is equal to $(A \cap (A^c \cup B))$.
	In other words, the statement above is equivalent to $w \in \truthset{\modults}{\chi} \cap (\truthset{\modults}{\neg\chi} \cup \truthset{\annmodelchi}{\varphi})$. Using the result of the first item, $w \in \truthset{\modults}{\chi} \cap \truthset{\modults}{\gann{\chi}\varphi}$. Hence, $w \in \truthset{\modults}{\chi \wedge \gann{\chi}\varphi}$.
	The other inclusion behaves the same way.
	\end{proof}

\begin{lemma}\label{lemma:palkh-valid}
	The reduction axioms from~\Cref{tab:palaxiom} are valid in $\cultsba$.
	\end{lemma}
	
	\begin{proof} The proof proceeds by cases for each axiom. We will focus only on the case of $\axm{RKh}$, while the rest follow similarly as for classical $\PAL$.

	By the definition of $\gann{\chi}$, $\modults,w \models \gann{\chi}\khi(\varphi,\psi)$ iff $\modults,w \models \chi$ implies $\annmodelchi,w \models \khi(\varphi,\psi)$. Using the definition of $\khi$, $\annmodelchi,w \models \khi(\varphi,\psi)$ iff there is $\plans \in (\annsemantics{\Unc}{\chi})(i)$ with $\plans \subseteq \ACT$ s.t. $\truthset{\annmodelchi}{\varphi} \subseteq \stexec^{\annmodelchi}(\plans)$ and $(\annsemantics{\R}{\chi})_\plans(\truthset{\annmodelchi}{\varphi}) \subseteq \truthset{\annmodelchi}{\psi}$.
	By the definition of $\annmodelchi$, $(\annsemantics{\Unc}{\chi})(i)=\Unc(i)$. Using \Cref{lem:palproperties}, $\truthset{\annmodelchi}{\varphi} = \truthset{\modults}{\chi \wedge \gann{\chi}\varphi}$ and $\truthset{\annmodelchi}{\psi} = \truthset{\modults}{\chi \wedge \gann{\chi}\psi}$.
	Thus, $\annmodelchi,w \models \khi(\varphi,\psi)$ iff there is $\plans \in \Unc(i)$ with $\plans \subseteq \ACT$ s.t. $\truthset{\modults}{\chi \wedge \gann{\chi}\varphi} \subseteq \stexec^{\annmodelchi}(\plans)$ and $(\annsemantics{\R}{\chi})_\plans(\truthset{\modults}{\chi \wedge \gann{\chi}\varphi}) \subseteq \truthset{\modults}{\chi \wedge \gann{\chi}\psi}$.
	Let $a \in \plans$ and $w \in \annsemantics{\W}{\chi}$. If $w \in \truthset{\modults}{\chi \wedge \gann{\chi}\varphi}$, then:
	\begin{itemize}
		\item $(\annsemantics{\R}{\chi})_a(w) \neq \emptyset$ (since $w \in \stexec^{\annmodelchi}(a)$) and
		\item $(\annsemantics{\R}{\chi})_a(w) \subseteq \truthset{\modults}{\chi \wedge \gann{\chi}\psi}$.
	\end{itemize}
	Using \Cref{def:annupdate}, the first item is equivalent to $w \in \truthset{\modults}{\chi}$, $\R_a(w) \subseteq \truthset{\modults}{\chi}$ and $\R_a(w) \neq \emptyset$ and with this information, $(\annsemantics{\R}{\chi})_a(w) = \R_a(w)$ that is useful for the second item.
	Hence, if $w \in \truthset{\modults}{\chi \wedge \gann{\chi}\varphi}$, then:
	\begin{itemize}
		\item $w \in \truthset{\modults}{\chi}$, $\R_a(w) \subseteq \truthset{\modults}{\chi}$ and $\R_a(w) \neq \emptyset$ and
		\item $\R_a(w) \subseteq \truthset{\modults}{\chi \wedge \gann{\chi}\psi}$.
	\end{itemize}
	Note that $w \in \truthset{\modults}{\chi}$ and $\R_a(w) \subseteq \truthset{\modults}{\chi}$ are redundant as $w \in \truthset{\modults}{\chi \wedge \gann{\chi}\varphi}$ and $\R_a(w) \subseteq \truthset{\modults}{\chi \wedge \gann{\chi}\psi}$.
	With this, if $w \in \truthset{\modults}{\chi \wedge \gann{\chi}\varphi}$, then:
	\begin{itemize}
		\item $\R_a(w) \neq \emptyset$ (thus, $w \in \stexec^\modults(a)$) and
		\item $\R_a(w) \subseteq \truthset{\modults}{\chi \wedge \gann{\chi}\psi}$.
	\end{itemize}
	Since we prove for arbitrary $a \in \plans$ and $w \in \annsemantics{\W}{\chi}$, the result yields for all $a \in \plans$ and $w \in \annsemantics{\W}{\chi}$.
	Moreover, it can be generalized for all $w \in \W$ as if $w \in \truthset{\modults}{\chi \wedge \gann{\chi}\varphi}$, then $w \in \annsemantics{\W}{\chi}$.
	Now $\annmodelchi,w \models \khi(\varphi,\psi)$ iff there is $\plans \in \Unc(i)$ with $\plans \subseteq \ACT$ s.t. $\truthset{\modults}{\chi \wedge \gann{\chi}\varphi} \subseteq \stexec(\plans)$ and $\R_\plans(\truthset{\modults}{\chi \wedge \gann{\chi}\varphi}) \subseteq \truthset{\modults}{\chi \wedge \gann{\chi}\psi}$.
	This happens iff $\modults,w \models \khi(\chi \wedge \gann{\chi}\varphi,\chi \wedge \gann{\chi}\psi)$. Since for this equivalence we prove assuming $\modults,w \models \chi$, then $\modults,w \models \chi$ implies $\annmodelchi,w \models \khi(\varphi,\psi)$ iff $\modults,w \models \chi$ implies $\modults,w \models \khi(\chi \wedge \gann{\chi}\varphi,\chi \wedge \gann{\chi}\psi)$. Thus, iff $\modults,w \models \chi \implies \khi(\chi \wedge \gann{\chi}\varphi,\chi \wedge \gann{\chi}\psi)$.
	\end{proof}
	
	

Since all the reduction axioms are valid, we can state the intended result. 

\medskip 

\begin{theorem}\label{th:palcomplete}
$\axset_{\khi}$ together with the reduction axioms for $\gann{\chi}$ in~\Cref{tab:palaxiom} are a sound and strongly complete axiomatization for $\PAKHilogic$ with respect to $\cultsba$.
\end{theorem}

\begin{proof}
The result follows by the correctness of the axioms and rule from~\Cref{tab:palaxiom} stated in~\Cref{lemma:palkh-valid} (which enables us to eliminate all the occurences of a $\gann{\chi}$ modality), together with the fact that the system in~\Cref{tab:khiaxiom} is also complete with respect to $\cultsba$ (see the proof in~\cite{AFSVQ21,AFSVQ23report}).
\end{proof}


% \begin{definition}\label{def:palredtokh}
% We introduce the following reduction axioms:
% \begin{enumerate}
% \item\label{item:palredp} $\gann{\chi}p \leftrightarrow (\chi \implies p)$;
% \item\label{item:palredneg} $\gann{\chi}\neg\varphi \leftrightarrow (\chi \implies \neg\gann{\chi}\varphi)$;
% \item\label{item:palredand} $\gann{\chi}(\varphi_1\wedge\varphi_2) \leftrightarrow \gann{\chi}\varphi_1 \wedge \gann{\chi}\varphi_2$;
% \item\label{item:palredkhi} $\gann{\chi}\khi(\varphi_1,\varphi_2) \leftrightarrow (\chi \implies \khi(\chi \wedge \gann{\chi}\varphi_1,\chi \wedge \gann{\chi}\varphi_2))$.
% \end{enumerate}
% \end{definition}

To finish this section, we prove that the satisfiability problem for $\PAKHilogic$ is decidable at least over $\cultsba$.

\medskip 

\begin{corollary}\label{cor:palsat}
The satisfiability problem for $\PAKHilogic$ over $\cultsba$ is decidable.
\end{corollary}
\begin{proof}
Let $\varphi$ be a formula of $\PAKHilogic$. By using the reduction axioms from~\Cref{tab:palaxiom} repeatedly, $\varphi$ can be translated (in a finite amount of steps) into a formula $\varphi'$, such that $\varphi$ and $\varphi'$ are equivalent in the class $\cultsba$ (\Cref{lemma:palkh-valid}). 
Since the satisfiability problem for $\KHilogic$ is decidable (\cite{AFSVQ21,AFSVQ23report}), there is a procedure, although non-deterministic, such that in a finite amount of steps determines whether $\varphi'$ is satisfiable or not.
As a result, there are two possible outcomes:
\begin{inlineenum}
\item If $\varphi'$ is satisfiable, then it is satisfiable in the class $\cultsba$, since as established in~\cite{AFSVQ21,AFSVQ23report}, every formula is satisfiable if it satisfiable in a finite model where each $\Unc(i)\subseteq\set{\set{a} \mid a\in\ACT}$, a subclass of models strictly contained in $\cultsba$.  
% As $\cultsfnu \subseteq \cultsba$, since each $\Unc(i)$ has singleton sets of actions, then $\varphi$ is satisfiable in $\cultsba$.
\item If $\varphi'$ is not satisfiable, then clearly neither it is satisfiable in $\cultsba$. %Then, $\varphi$ is not in $\cultsba$.
\end{inlineenum}
Thus, there is a procedure that decides satisfiability of $\PAKHilogic$ over $\cultsba$. 
\end{proof}
