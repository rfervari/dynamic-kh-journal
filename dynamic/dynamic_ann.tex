% Let us start by considering \PAL-like announcements in the context of knowing how logics. For the $\kh$ modality first introduced in~\cite{Wang15lori}, such an announcement would result in updating the labelled transition system in which the formula is interpreted, producing a change in the agent's abilities. In that setting, these abilities coincide with her knowledge.

%As discussed in the previous sections, our goal is to take ideas from standard epistemic logic to perform operations that update an agent's know-how.

A typical way of updating a relational model is by removing states~\cite{Plaza89:lopc,DELbook}. This model update, interpreted in the just referred literature as an announcement indicating to all agents that $\chi$ is true (hence the setting's name: \emph{public announcement logic}; \PAL), is typically described with a modal operator $[\chi]$. Given an $\ults$ $\modults=\tup{\W,\R,\Unc,\V}$, its semantic interpretation is defined as:
\begin{spcenter}
  $\modults,w\models[\chi]\varphi \ \ \iff \ \  \modults,w\models\chi \mbox{ implies } \modults_{\chi},w\models\varphi$,
\end{spcenter}
with the $\ults$ $\modults_\chi=\tup{\W_\chi,\R_\chi,\Unc_\chi,\V_\chi}$ given by $\W_\chi = \set{w \mid \modults,w\models\chi}$, $(\R_\chi)_a = \R_a \cap \W_\chi^2$ for each $a\in\ACT$, $\V_\chi(w) = \V(w)$ for all $w\in\W_\chi$, and $\Unc_\chi = {\Unc}$.
%\begin{spcenter}
%  \begin{tabular}{l@{\quad \quad \quad }l}
%    $\W_\chi = \set{w \mid \modults,w\models\chi}$ & $(\R_\chi)_a = \R_a \cap \W_\chi^2, \mbox{ for each } a\in\ACT$ \\
%    $\V_\chi(w) = \V(w), \mbox{ for all } w\in\W_\chi$ & $\Unc_\chi = {\Unc}$.
%  \end{tabular}	
%\end{spcenter}
Thus, $\modults_\chi$ is obtained by taking $\truthset{\modults}{\chi}$ as the new domain, then restricting relations and valuation accordingly (the indistinguishability classes $\Unc(i)$ are left unchanged). 

A couple of remarks might be useful here. First recall that, in the \emph{knowing how} setting, states other than the evaluation point do not represent epistemic possibilities (as they do in EL), but rather situations reachable by actions. Thus, in the \emph{knowing how} setting, the elimination of some states indicates not that they are no longer \emph{possible}, but rather that they are no longer \emph{reachable}. Moreover, in the original \emph{knowing how} setting, the relations define the agents' abilities. Hence, under that semantics, removing states corresponds to both an ontic and an epistemic change: the effect of available actions changes, and hence so do the agents' abilities~\cite{Wang2016}. However, in the \ults-based semantics, relations provide only ontic information; thus, removing states produces an \emph{ontic} change that might affect, indirectly, the agents' abilities.

In the $\ults$ setting, it can be shown that the update operator adds expressivity to $\KHilogic$ (a similar result was established in~\cite{Wang2016} for the original $\lts$-based setting).

\medskip 

\begin{proposition}\label{prop:pal-exp}
Adding $[\chi]$ to $\KHilogic$ increases the expressive power.
\end{proposition}

\begin{proof}
  The \ultss $\modults$ and $\modults'$ below (with $\Unc(i)=\Unc'(i)=\set{\set{a}}$, and with dashed lines indicating nodes and edges to be removed after an update with $p$) are $\KHilogic$-bisimilar (\Cref{def:bisim-khi}), and thus indistinguishable in $\KHilogic$. 

\begin{center}
    \begin{tikzpicture}[->]
      \node [state,label=left:\small$w$] (w1) {$p,q$};
      \node [state, below = 1em of w1] (w2) {$p,q$};
      \node [state, dashed, right = of w1] (w3) {\phantom{$p$}};
      \node [state, below = 1em of w3] (w4) {$p$};
      \node[left = of w1] (m) {$\modults$};

      \path (w1) edge node [label-edge, right] {$a$} (w2)
             (w2) edge[loop left] node [label-edge, left] {$a$} (w2)
             (w1) edge[dashed] node [label-edge, above] {$a$} (w3)
             (w4) edge node [label-edge, below] {$a$} (w2);

        \node[right = of w3] (phan) {};
        \node [state,right = of phan,label=left:\small$w'$] (w1) {$p,q$};
        \node [state, below = 1em of w1] (w2) {$p,q$};
        \node [state, dashed, right = of w1] (w3) {\phantom{$p$}};
        \node [state, below = 1em of w3] (w4) {$p$};
        \node[right = of w3] (mp) {$\modults'$};

        \path (w2) edge[loop left] node [label-edge, left] {$a$} (w2)
            (w1) edge[dashed] node [label-edge, above] {$a$} (w3)
            (w4) edge node [label-edge, below] {$a$} (w2);

    \end{tikzpicture}
  \end{center}
However, $\modults,w\models[p]\khi(p,q)$ whereas
$\modults',w'\not\models[p]\khi(p,q)$.
\end{proof}

A consequence of~\Cref{prop:pal-exp} is that the modality for \PAL-like updates, $[\chi]$, is not reducible to the underlying language $\KHilogic$. This makes sense: while $\KHilogic$ can express the agents' abilities to achieve certain goals from given situations (the modality $\khi$), it cannot talk about more specific properties of the courses of action on which the abilities rely (this is in contrast with what happens when these modalities are added to standard epistemic logic, where reduction axioms can be defined (see, e.g.,~\cite{DELbook}). In such situations, the axiomatization can be approached in different ways. For this operation, the `extend the basic language' method has already been explored in \cite{Wang2016}. Here, the strategy is to look for a variation of the operation, and then to restrict the setting to a particular class of models.

\medskip

\begin{definition}\label{def:annupdate}\label{def:pakhsyntax}
Let $\modults = \tup{\W, \R, \Unc, \V}$ be an \ults, and let $\chi$ be a formula. The $\ults$ $\annmodelchi = \tup{\annsemantics{\W}{\chi}, \annsemantics{\R}{\chi}, \annsemantics{\Unc}{\chi}, \annsemantics{\V}{\chi}}$ is given by:
\begin{itemize}
\item $\annsemantics{\W}{\chi} = \truthset{\modults}{\chi}$,
\item $(\annsemantics{\R}{\chi})_a = \setof{(w,v) \in \R_a}{w \in \truthset{\modults}{\chi} \text{ and } \R_a(w) \subseteq \truthset{\modults}{\chi}}$ for every $a \in \ACT$,
\item $\annsemantics{\Unc}{\chi} = \Unc$, and 
\item $\annsemantics{\V}{\chi}(w) = \V(w)$ (for all $w\in\annsemantics{\W}{\chi}$).
\end{itemize}
\smallskip
The language $\PAKHilogic$ extends $\KHilogic$ with formulas of the form $\gann{\chi}\varphi$, with the semantic interpretation of the new formulas given by
\begin{spcenter}
	$\modults,w \models \gann{\chi}\varphi \ \ \iff \ \ \modults,w \models \chi \mbox{ implies } \annmodelchi,w \models \varphi$.
\end{spcenter}
\end{definition}

Thus, given a model $\modults$, the only difference between the standard $\modults_{\chi}$ and the just defined $\annmodelchi$ is in the definition of 
its relations. Indeed, in the former, each $(\R_\chi)_a$ restricts the original $\R_a$ to the new domain. In the latter, however, each $(\annsemantics{\R}{\chi})_a$ is defined point-wise: it is exactly as $\R_a(w)$ if $w$ and all the states $\R_a(w)$ can reach will survive the operation ($\R_a(w) \cup \set{w} \subseteq \truthset{\modults}{\chi}$ implies $(\annsemantics{\R}{\chi})_a(w) = \R_a(w)$), and is empty otherwise ($\R_a(w) \cup \set{w} \not \subseteq \truthset{\modults}{\chi}$ implies $(\annsemantics{\R}{\chi})_a(w) = \emptyset$).

%\footnote{The two forms of model update discussed here resemble to the two forms of updating neighbourhood models, see~\cite{MaS18} for details.}
%Recall that a neighbourhood model~\cite{pacuit17} is given by: a non-empty domain $\W$, an atomic valuation, and a neighbourhood function $\N:\W \to 2^{2^{\W}}$, assigning a set of sets of states to each possible state. Let $U \subseteq \W$ be a non-empty set of states. On the one hand, the \emph{$U$-intersection} submodel defined in~\cite{MaS18} has $U$ as its domain, with its neighbourhood function built by restricting each set in a neighbourhood to the new domain, analogous to what $\modults_{\chi}$ (a standard announcement) does.
% On the other hand, the \emph{$U$-subset} submodel therein also has $U$ as its domain, but its neighbourhood function is built by keeping only those sets that are already a subset of the new domain, analogous to what $\annsemantics{\modults}{\chi}$ does. We argue that this second approach is more appropriate in the context of knowing how.}.

Even with this more restricted version of an update, the resulting logic fails to have reduction axioms: adding $\gann{\chi}$ to $\KHilogic$ increases the expressive power.

\medskip 

\begin{proposition}\label{prop:exppal}
	$\PAKHilogic$ is more expressive than $\KHilogic$ over arbitrary \ultss.
\end{proposition}

\begin{proof}
	Let $\modults$ and $\modults'$ be the single agent models depicted below, with $\Unc(i):=\set{\set{ab}}$ and $\Unc'(i):=\set{\set{a}}$:

	%\centerline{DESCOMENTAR IMAGENES ABAJO!!!}
	\begin{center}
		\begin{tikzpicture}[->, grow' = right, level/.style={sibling distance = 3em/#1}, level distance = 3.5em]
		\node at (0,1) [state,label=left:$w$] (p) {$p,r$} ;
		\node at (2,0.5) [state,dashed] (nr) {\phantom{$q,r$}};
		\node at (2,1.5) [state] (q) {$q,r$};
		\node[left = of p] (m) {$\modults$};

		\path (p) edge [dashed, above] node {$a$} (q);
		\path (q) edge [loop right] node [right] {$b$} (q);
		\path (p) edge [dashed,below] node {$a$} (nr);
		\path (nr) edge [dashed,right] node {$b$} (q);
		\end{tikzpicture}
		\hspace{2cm}
		\begin{tikzpicture}[->, grow' = right, level/.style={sibling distance = 3em/#1}, level distance = 3.5em]
		\node[state] at (0,1) (p) [label=left:$w'$]{$p,r$};
		\node[state,dashed] at (2,0.5) (nr) {\phantom{$\neg r$}};
		\node[state] at (2,1.5) (q) {$q,r$};
		\node[right = 7em of p] (m) {$\modults'$};
		\path (p) edge [above] node {$a$} (q);
		\path (p) edge [dashed,below] node {$b$} (nr);
		\end{tikzpicture}
\end{center}

Both models are $\KHilogic$-bisimilar (\Cref{def:bisim-khi}); hence, they satisfy the same formulas in $\KHilogic$. However, $\modults,w \not\models \gann{r}\khi(p,q)$ since $\modults,w \models r$ and $\annmodel{r},w \not\models \khi(p,q)$, whereas $\modults',w' \models \gann{r}\khi(p,q)$ since $\modults',w' \models r$ and $\annmodel{r}',w \models \khi(p,q)$.
	%
	% \begin{nscenter}
	%   \scalebox{.7}{
	% \begin{tikzpicture}[->, grow' = right, level/.style={sibling distance = 3em/#1}, level distance = 3.5em]
	% \node[state] at (0,1) (p) [label=left:$w$]{$p,r$};
	% \node[state] at (2,2) (q) {$q,r$};
	% \path (q) edge [loop above] node {$b$} (q);
	% \end{tikzpicture}
	% \hspace{2cm}
	% \begin{tikzpicture}[->, grow' = right, level/.style={sibling distance = 3em/#1}, level distance = 3.5em]
	% \node[state] at (0,1) (p) [label=left:$w'$]{$p,r$};
	% \node[state] at (2,2) (q) {$q,r$};
	% \path (p) edge [above] node {$a$} (q);
	% \end{tikzpicture}
	%   }
	% \end{nscenter}
\end{proof}

In models $\modults$ and $\modults'$ from the above proposition, it is useful to notice the following. First, they both make the formula $\khi(p,q)$ true, with their respective witnesses being $\set{ab}$ and $\set{a}$. But, as discussed, after eliminating worlds in which $r$ is false this is no longer the case: the formula fails in $\annmodel{r}$ but holds in $\annmodel{r}'$. Part of the reason is that, while the witness for the first contains a two-step plan, the witness for the second contains only one-step plans (i.e., actions). This makes a difference because, by removing intermediate steps, $\PAKHilogic$ can tell these two models apart. 

But, as mentioned before, the semantic interpretation for $\khi$ is `blind' to certain aspects of its witness. In fact, as the completeness proof shows~\cite{AFSVQ21}, it cannot distinguish between an arbitrary \ults and one in which, for every agent $i$, every set in $\Unc(i)$ is a singleton containing a one-step plan (i.e., a model in which $\plans \in \Unc(i)$ implies $\plans = \set{a}$ for some $a \in \ACT$). This suggests that, by restricting the class of models, one can still get agents with `the same abilities' while also stopping $\PAKHilogic$ from being able to tell $\KHilogic$-bisimilar models apart.

\medskip 

\begin{definition}\label{def:class-m-one}
Define $\cultsba$ as the class of models $\modults = \tup{\W, \R, \Unc, \V}$ in which, for all $i \in \AGT$, we have that $\plans \in \Unc(i)$ implies $\plans \subseteq \ACT$.
\end{definition}

\medskip 

The class $\cultsba$ (denoted as $\sults$ in \cite{AFSV22}) constitutes a restricted class of models, which could correspond, for example, to a more abstract representation of the abilities of the agents (every plan is modeled as a single action). This class slightly generalizes the one of the canonical model, as here we allow each $\plans$ to be any set of one-step plans (i.e., models are such that $\plans \in \Unc(i)$ implies $\plans = \set{a_1,\ldots,a_k} \subseteq \ACT$),  instead of just singletons of them. The reduction axioms from~\Cref{tab:palaxiom} are valid in the class of models $\cultsba$. Moreover, we can use them to eliminate announcements by iteratively replacing the innermost occurrence of a $\gann{\chi}$ modality. Putting all together, completeness for $\PAKHilogic$ follows.

\begin{table}[t]
\begin{tabular}{l@{\quad}l}
\toprule
\axm{RAtom} & $\vdash \gann{\chi}p \leftrightarrow (\chi \implies p)$ \\
\axm{R$\neg$} & $\vdash \gann{\chi}\neg\varphi \leftrightarrow (\chi \implies \neg\gann{\chi}\varphi)$ \\
\axm{R$\vee$} & $\vdash \gann{\chi}(\varphi\vee\psi) \leftrightarrow \gann{\chi}\varphi \vee\gann{\chi}\psi$ \\
\axm{RKh} & $\vdash \gann{\chi}\khi(\varphi,\psi) \leftrightarrow (\chi \implies \khi(\chi \wedge \gann{\chi}\varphi,\chi \wedge \gann{\chi}\psi))$ \\
\axm{RE$_{\gann{}}$} & $\text{From } \vdash \varphi \leftrightarrow \psi \text{ derive } \vdash \gann{\chi}\varphi \leftrightarrow \gann{\chi}\psi$ \\
\bottomrule
\end{tabular}
\caption{Reduction axioms $\axset_{\PAKHilogic}$.}\label{tab:palaxiom}
\end{table}

\medskip 

To establish the validity of the reduction axioms, we first need to prove the following semantic properties.

\medskip 

\begin{lemma}\label{lem:palproperties} Let $\chi$ and $\varphi$ be $\PAKHilogic$-formulas; let $\modults$ be an arbitrary \ults.
	The following equalities hold:
	\begin{multicols}{2}
	\begin{enumerate}
	\item $\truthset{\modults}{\gann{\chi}\varphi} = \truthset{\modults}{\neg\chi} \cup \truthset{\annmodelchi}{\varphi}$.
	\item $\truthset{\annmodelchi}{\varphi} = \truthset{\modults}{\chi \wedge \gann{\chi}\varphi}$.
	\end{enumerate}
	\end{multicols}
	\end{lemma}
	
	\begin{proof}
 For Item 1, let $w \in \truthset{\modults}{\gann{\chi}\varphi}$, we have that $\modults,w \models \gann{\chi}\varphi$. Thus, $\modults,w \models \chi$ implies $\annmodelchi,w \models \varphi$, which can be rewritten as $\modults,w \models \neg\chi$ or $\annmodelchi,w \models \varphi$. Hence, $w \in \truthset{\modults}{\neg\chi} \cup \truthset{\annmodelchi}{\varphi}$. The other inclusion behaves the same way.
 
 
For Item 2, let $w \in \truthset{\annmodelchi}{\varphi}$, we have that $w \in \annsemantics{\W}{\chi} = \truthset{\modults}{\chi}$ and $w \in \truthset{\annmodelchi}{\varphi}$. Thus, $w \in \truthset{\modults}{\chi} \cap \truthset{\annmodelchi}{\varphi}$. The previous set is in the form of $(A \cap B)$. By set reasoning, it is equal to $(A \cap (A^c \cup B))$.
	In other words, the statement above is equivalent to $w \in \truthset{\modults}{\chi} \cap (\truthset{\modults}{\neg\chi} \cup \truthset{\annmodelchi}{\varphi})$. Using the result of the first item, $w \in \truthset{\modults}{\chi} \cap \truthset{\modults}{\gann{\chi}\varphi}$. Hence, $w \in \truthset{\modults}{\chi \wedge \gann{\chi}\varphi}$.
	The other inclusion behaves the same way.
	\end{proof}

\begin{lemma}\label{lemma:palkh-valid}
	The reduction axioms from~\Cref{tab:palaxiom} are valid in $\cultsba$.
	\end{lemma}
	
	\begin{proof} The proof proceeds by cases for each axiom. We will focus only on the case of $\axm{RKh}$, while the rest follow similarly as for classical $\PAL$.

	By the definition of $\gann{\chi}$, $\modults,w \models \gann{\chi}\khi(\varphi,\psi)$ iff $\modults,w \models \chi$ implies $\annmodelchi,w \models \khi(\varphi,\psi)$. Using the definition of $\khi$, $\annmodelchi,w \models \khi(\varphi,\psi)$ iff there is $\plans \in (\annsemantics{\Unc}{\chi})(i)$ with $\plans \subseteq \ACT$ s.t. $\truthset{\annmodelchi}{\varphi} \subseteq \stexec^{\annmodelchi}(\plans)$ and $(\annsemantics{\R}{\chi})_\plans(\truthset{\annmodelchi}{\varphi}) \subseteq \truthset{\annmodelchi}{\psi}$.
	By the definition of $\annmodelchi$, $(\annsemantics{\Unc}{\chi})(i)=\Unc(i)$. Using \Cref{lem:palproperties}, $\truthset{\annmodelchi}{\varphi} = \truthset{\modults}{\chi \wedge \gann{\chi}\varphi}$ and $\truthset{\annmodelchi}{\psi} = \truthset{\modults}{\chi \wedge \gann{\chi}\psi}$.
	Thus, $\annmodelchi,w \models \khi(\varphi,\psi)$ iff there is $\plans \in \Unc(i)$ with $\plans \subseteq \ACT$ s.t. $\truthset{\modults}{\chi \wedge \gann{\chi}\varphi} \subseteq \stexec^{\annmodelchi}(\plans)$ and $(\annsemantics{\R}{\chi})_\plans(\truthset{\modults}{\chi \wedge \gann{\chi}\varphi}) \subseteq \truthset{\modults}{\chi \wedge \gann{\chi}\psi}$.
	Let $a \in \plans$ and $w \in \annsemantics{\W}{\chi}$. If $w \in \truthset{\modults}{\chi \wedge \gann{\chi}\varphi}$, then:
	\begin{itemize}
		\item $(\annsemantics{\R}{\chi})_a(w) \neq \emptyset$ (since $w \in \stexec^{\annmodelchi}(a)$) and
		\item $(\annsemantics{\R}{\chi})_a(w) \subseteq \truthset{\modults}{\chi \wedge \gann{\chi}\psi}$.
	\end{itemize}
	Using \Cref{def:annupdate}, the first item is equivalent to $w \in \truthset{\modults}{\chi}$, $\R_a(w) \subseteq \truthset{\modults}{\chi}$ and $\R_a(w) \neq \emptyset$ and with this information, $(\annsemantics{\R}{\chi})_a(w) = \R_a(w)$ that is useful for the second item.
	Hence, if $w \in \truthset{\modults}{\chi \wedge \gann{\chi}\varphi}$, then:
	\begin{itemize}
		\item $w \in \truthset{\modults}{\chi}$, $\R_a(w) \subseteq \truthset{\modults}{\chi}$ and $\R_a(w) \neq \emptyset$ and
		\item $\R_a(w) \subseteq \truthset{\modults}{\chi \wedge \gann{\chi}\psi}$.
	\end{itemize}
	Note that $w \in \truthset{\modults}{\chi}$ and $\R_a(w) \subseteq \truthset{\modults}{\chi}$ are redundant as $w \in \truthset{\modults}{\chi \wedge \gann{\chi}\varphi}$ and $\R_a(w) \subseteq \truthset{\modults}{\chi \wedge \gann{\chi}\psi}$.
	With this, if $w \in \truthset{\modults}{\chi \wedge \gann{\chi}\varphi}$, then:
	\begin{itemize}
		\item $\R_a(w) \neq \emptyset$ (thus, $w \in \stexec^\modults(a)$) and
		\item $\R_a(w) \subseteq \truthset{\modults}{\chi \wedge \gann{\chi}\psi}$.
	\end{itemize}
	Since we prove for arbitrary $a \in \plans$ and $w \in \annsemantics{\W}{\chi}$, the result yields for all $a \in \plans$ and $w \in \annsemantics{\W}{\chi}$.
	Moreover, it can be generalized for all $w \in \W$ as if $w \in \truthset{\modults}{\chi \wedge \gann{\chi}\varphi}$, then $w \in \annsemantics{\W}{\chi}$.
	Now $\annmodelchi,w \models \khi(\varphi,\psi)$ iff there is $\plans \in \Unc(i)$ with $\plans \subseteq \ACT$ s.t. $\truthset{\modults}{\chi \wedge \gann{\chi}\varphi} \subseteq \stexec(\plans)$ and $\R_\plans(\truthset{\modults}{\chi \wedge \gann{\chi}\varphi}) \subseteq \truthset{\modults}{\chi \wedge \gann{\chi}\psi}$.
	This happens iff $\modults,w \models \khi(\chi \wedge \gann{\chi}\varphi,\chi \wedge \gann{\chi}\psi)$. Since for this equivalence we prove assuming $\modults,w \models \chi$, then $\modults,w \models \chi$ implies $\annmodelchi,w \models \khi(\varphi,\psi)$ iff $\modults,w \models \chi$ implies $\modults,w \models \khi(\chi \wedge \gann{\chi}\varphi,\chi \wedge \gann{\chi}\psi)$. Thus, iff $\modults,w \models \chi \implies \khi(\chi \wedge \gann{\chi}\varphi,\chi \wedge \gann{\chi}\psi)$.
	\end{proof}
	
	

Since all the reduction axioms are valid, we can state the intended result. 

\medskip 

\begin{theorem}\label{th:palcomplete}
$\axset_{\khi}$ together with the reduction axioms for $\gann{\chi}$ in~\Cref{tab:palaxiom} are a sound and strongly complete axiomatization for $\PAKHilogic$ with respect to $\cultsba$.
\end{theorem}

\begin{proof}
The result follows by the correctness of the axioms and rule from~\Cref{tab:palaxiom} stated in~\Cref{lemma:palkh-valid} (which enables us to eliminate all the occurences of a $\gann{\chi}$ modality), together with the fact that the system in~\Cref{tab:khiaxiom} is also complete with respect to $\cultsba$ (see the proof in~\cite{AFSVQ21,AFSVQ23report}).
\end{proof}


% \begin{definition}\label{def:palredtokh}
% We introduce the following reduction axioms:
% \begin{enumerate}
% \item\label{item:palredp} $\gann{\chi}p \leftrightarrow (\chi \implies p)$;
% \item\label{item:palredneg} $\gann{\chi}\neg\varphi \leftrightarrow (\chi \implies \neg\gann{\chi}\varphi)$;
% \item\label{item:palredand} $\gann{\chi}(\varphi_1\wedge\varphi_2) \leftrightarrow \gann{\chi}\varphi_1 \wedge \gann{\chi}\varphi_2$;
% \item\label{item:palredkhi} $\gann{\chi}\khi(\varphi_1,\varphi_2) \leftrightarrow (\chi \implies \khi(\chi \wedge \gann{\chi}\varphi_1,\chi \wedge \gann{\chi}\varphi_2))$.
% \end{enumerate}
% \end{definition}

% Now, a valid question that comes with these results is: Do we have reduction axioms for $\cultsba$ if we use the standard model update $[\chi]$ from DEL instead of $\gann{\chi}$?
% At the time of writing, the answer seems unclear as the case for reducing $[\chi] \khi(\psi,\varphi)$ in $\cultsba$ presents some challenges at a relations level.
% Using the semantics we can get that there is $\plans \in \Unc(i)$ with $\plans \subseteq \ACT$ s.t.
% $\truthset{\modults}{\chi \wedge [\chi]\psi} \subseteq \stexec^{\modults_\chi}(\plans)$ and $(\R_{\chi})_\plans(\truthset{\modults}{\chi \wedge \gann{\chi}\psi}) \subseteq \truthset{\modults}{\chi \wedge \gann{\chi}\varphi}$.
% But with some further analysis, the parts $\stexec^{\modults_\chi}(\plans)$ and $(\R_{\chi})_\plans(\truthset{\modults}{\chi \wedge \gann{\chi}\psi})$ proved very hard to disassemble and get an expression of the form $\khi(A,B)$.
% One of the reasons we considered is that the $[\chi]$ modality has an edge-by-edge focus instead of having an group-edge one, something needed for $\khi$ formulas to have a proper translation.
% Although we are still considering that there might be a class of models more adequate for $[\chi]$, there seems to be no clear answer.

% \bigfer{Si trabajamos en la clase $\cultsba$, ¿hay axiomas de reducción para la operación estandar de PAL? Andr\'es: mencionas que te parece que no. ¿Podrías agregar aquí un par de lineas describiendo el problema? Si no estas seguro, entones hay que añadir una linea diciendo que la respuesta a eso no es clara, y luego tal vez mencionarlo en trabajo a futuro.}

It is worth noticing that, even in the class $\cultsba$, the modality $[\chi]$ (usually known in EL as `public announcement') adds expressive power to $\KHilogic$, i.e., there are no reduction axioms to eliminate its occurrences in formulas. This follows already from~\Cref{prop:pal-exp}, since the counterexample models in the proof belong to the class $\cultsba$. Therefore, the meaning of $[\chi]$ and $\gann{\chi}$ is different even in such a restricted class of models.

To finish this discussion, we prove that the satisfiability problem for $\PAKHilogic$ is decidable at least over $\cultsba$.

\medskip 

\begin{corollary}\label{cor:palsat}
The satisfiability problem for $\PAKHilogic$ over $\cultsba$ is decidable.
\end{corollary}
\begin{proof}
Let $\varphi$ be a formula of $\PAKHilogic$. By using the reduction axioms from~\Cref{tab:palaxiom} repeatedly, $\varphi$ can be translated (in a finite amount of steps) into a formula $\varphi'$, such that $\varphi$ and $\varphi'$ are equivalent in the class $\cultsba$ (\Cref{lemma:palkh-valid}). 
Since the satisfiability problem for $\KHilogic$ is decidable (\cite{AFSVQ21,AFSVQ23report}), there is a procedure, although non-deterministic, such that in a finite amount of steps determines whether $\varphi'$ is satisfiable or not.
As a result, there are two possible outcomes:
\begin{inlineenum}
\item If $\varphi'$ is satisfiable, then it is satisfiable in the class $\cultsba$, since as established in~\cite{AFSVQ21,AFSVQ23report}, every formula is satisfiable if it is  satisfiable in a finite model where each $\Unc(i)\subseteq\set{\set{a} \mid a\in\ACT}$, a subclass of models strictly contained in $\cultsba$.  
% As $\cultsfnu \subseteq \cultsba$, since each $\Unc(i)$ has singleton sets of actions, then $\varphi$ is satisfiable in $\cultsba$.
\item If $\varphi'$ is not satisfiable, then clearly neither it is satisfiable in $\cultsba$. %Then, $\varphi$ is not in $\cultsba$.
\end{inlineenum}
Thus, there is a procedure that decides satisfiability of $\PAKHilogic$ over $\cultsba$. 
\end{proof}
