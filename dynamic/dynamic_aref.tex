As mentioned, the operation $\refdiam{\plan_1}{\plan_2}$ can be seen as a particular form of (publicly) removing uncertainty: one indicates precisely the plans that can be distinguished now, and then quantifies over the different ways of doing so. The operation defined below is a more abstract one: in the spirit of other proposals that quantify over epistemic actions (e.g., the arbitrary announcements of \cite{BalbianiBDHHL08}, the arbitrary arrow updates of \cite{DitmarschHKK17}, the group announcements of \cite{GroupAJ}, the coalition announcements of \cite{AgotnesD08} and the arbitrary radical upgrades of \cite{FI24}), it quantifies over all the different ways in which the agent's indistinguishability can be refined. In the context of knowing how, this approach will be shown later to be suitable for definining a notion of ``learnability''.

\medskip

\begin{definition}\label{def:sem:aref}
Let $\modults$ be an \ults and $w\in\D{\modults}$.  % a state in $\modults$.
Then,
\[
\modults,w\models\arefdiam\varphi\ \iffdef \ \text{there are } \plan_1,\plan_2 \in \ACT^* \text{ such that } 
\modults,w \models \refdiam{\plan_1}{\plan_2}\varphi.
\]
As usual $\arefbox\varphi = \neg\arefdiam\neg\varphi$. We denote $\AReflogic$ (for ``arbitrary refinement'') as the extension of $\KHilogic$ with the modality $\arefdiam$.
\end{definition}

\medskip

Notice that knowing how operators are \emph{goal-directed}: the agent looks for a suitable course of action that makes her achieve a certain state. 
In $\AReflogic$, we can define an operator that, when possible, \emph{guarantees} that the agent \emph{learns how} to achieve a goal.
This action can be understood as a goal-directed learning how: it looks for a way to split \emph{some} existing set of plans $\plans$ in such a way that the agent knows how to achieve $\varphi$ given $\psi$. 

\medskip 

Let $\LHlogic$ (for ``learning how'') be $\KHilogic$ extended with the dynamic modality
\[
\lh{\psi}{\varphi}_i\chi := \arefdiam(\khi(\psi,\varphi) \wedge\chi),
\]
(and its `dual' $[\psi,\varphi]_i\chi := \neg\lh{\psi}{\varphi}_i\neg\chi$).
Moreover, we define $\learn_i(\psi,\varphi) := \lh{\psi}{\varphi}_i\top$ an abbreviation for \emph{``the agent $i$ can learn how to make $\varphi$ true in the presence of $\psi$''}.
Notice that $\LHlogic$ is a syntactic fragment of $\AReflogic$.

The new dynamic modality is a ternary modality expressing that the agent is able to learn how to achieve $\varphi$ given $\psi$, and that after this learning operation takes place, $\chi$ holds.
%However, as we will see below, it can be seen as a unary modality similarly to public announcements or action models.
The modality $\learn_i$ is a test of what is learnable by the agent~$i$.


\medskip

\begin{proposition}\label{prop:nolearn}
It follows from the semantics that:
\begin{enumerate}
\item\label{itm:nolearnable} $\not\models\learn_i(\varphi,\psi)$; %and $\not\models\lh{\psi}{\varphi}\kh(\psi,\varphi)$;
\item\label{itm:learnboth} $\learn_i(\varphi,\psi) \wedge \learn_i(\varphi,\neg\psi)$ is satisfiable.
%$\lh{\varphi}{\psi}\kh(\varphi,\psi)\wedge\lh{\varphi}{\psi}\kh(\varphi,\neg\psi)$ is satisfiable.
\end{enumerate}
\end{proposition}

\begin{proof}
\Cref{itm:nolearnable} shows that not everything is learnable by an agent.
%The properties of the underlying LTS
The (un)avail\-abil\-i\-ty of certain actions in an \ults restricts
what can be learnt.  Consider the following single-agent \ults $\modults$, with
the set $\Unc(i)$ shown on the right.
\begin{center}
\begin{tabular}{c}
\begin{tikzpicture}[->]
\node [state, label = {[label-state]left:$w$}] (w1) {$p$};
\node[left = of w1] (m) {$\modults$};
\node [state, right = of w1] (w2) {$p$};
\node [state, right = of w2] (w3) {$p,r$};

\path (w1) edge node [label-edge, above] {$a$} (w2)
        (w2) edge node [label-edge, above] {$b$} (w3);
\end{tikzpicture}
\end{tabular}
\begin{picture}(90,0)
\put(20,0){$\Unc(i) = \left\{
    \begin{array}{c}
        \set{ab, a}, \set{\epsilon}
    \end{array}
    \right\}$}
\end{picture} 
%
%  \begin{tabular}{c}
%  \begin{tikzpicture}[node distance = .5em and 2.5em]
%    \draw (0,0) rectangle (1.5cm,1.7cm);
%    \draw (.3cm,.3cm) rectangle (1.2cm,.7cm) node [pos = 0.5] {$a \ \ b \ \ \epsilon$};
%    \draw (.5cm,1.1cm) rectangle (.9cm,1.5cm) node [pos = 0.5] {$ab$};
%  \end{tikzpicture}
%  \end{tabular}
%\end{tabular}
\end{center}
Note that $\modults,w\not\models\khi(p,r)$.
The set $\set{ab,a}$ is not executable at every $p$-state, it is only executable at $w$.
On the other hand, $\set{\epsilon}$ is executable everywhere, but does not lead always to $r$-states.
Moreover, $\modults,w\not\models\learn_i(p,r)$.
The set $\set{\epsilon}$ cannot be refined, and no refinement of $\set{ab,a}$ does the work.
Therefore, agent $i$ cannot learn how to make $r$ true when $p$ holds.

For \Cref{itm:learnboth} consider $\modults'$ as in~\Cref{prop:expref}.
As said, $\modults',w'\not\models\khi(p,q)$.
%again $\modults$ above. %As said, $\modults,w\not\models\kh(p,r)$.
However, there is a way to learn how to achieve $q$ given $p$: it is possible to split the set $\set{a,b}$ into $\set{a}$ and $\set{b}$; hence, $\modults',w'\models \learn_i(p,q)$ (witness $\set{a}$) but also $\modults',w'\models\learn_i(p,\neg q)$ (witness $\set{b}$).
\end{proof}

\Crefitem{itm:nolearnable} shows how, in certain scenarios, there is
no room for learning. For instance, there might be no way to learn how
to cure a disease, if there is no doctor
available or no vaccine has been developed yet. \Crefitem{itm:learnboth} shows how the agent might be able
to learn not only how to make a formula true under a given condition,
but, at the same time, how to make the same formula false under the same condition.

\medskip

Let us go back to analyzing the properties of the logic $\AReflogic$, and their impact in obtaining complete axiomatizations. First, we enumerate some properties that can be expressed in the logic. In particular, the modality $\arefdiam$ is normal and serial, satisfies natural properties 
of Monotonicity and Weakening, but fails for dynamic versions of axioms 4 and 5 (usually known as \emph{positive} and \emph{negative introspection}). 

\medskip 

\begin{proposition}\label{prop:aref-normal-serial}
It follows from the semantics (\Cref{def:sem:aref}) that:
\begin{enumerate}
\item $\models \arefbox(\varphi \ra \psi) \ra (\arefbox\varphi \ra
\arefbox\psi)$.
\item If $\models \varphi$, then $\models \arefbox\varphi$.
\item $\models \arefbox\varphi \ra \arefdiam\varphi$.
%\end{enumerate}
%\end{proposition}

%\begin{proof}
%Let $\modults = \tup{\W,\R,\Unc,\V}$ and $w \in \W$:
%\begin{enumerate}
%\item Suppose $\modults,w \models \arefbox(\varphi \ra \psi)$ and $\modults,w \models \arefbox\varphi$ and let $\plan_1,\plan_2 \in \ACT^*$.
%We have that $\modults,w \models \refbox{\plan_1}{\plan_2}(\varphi \ra \psi)$ and $\modults,w \models \refbox{\plan_1}{\plan_2}\varphi$.
%Using \Cref{itm:distref} from \Cref{prop:ref-normal-serial}, $\modults,w \models \refbox{\plan_1}{\plan_2}\psi$.
%Thus, $\modults,w \models \arefbox\psi$.
%\item Suppose $\models \varphi$ and let $\plan_1,\plan_2 \in \ACT^*$.
%Using \Cref{itm:necessitationref} from \Cref{prop:ref-normal-serial}, $\models \refbox{\plan_1}{\plan_2}\varphi$.
%Thus, $\models \arefbox\varphi$.
%\item Suppose $\modults,w \models \arefbox\varphi$.
%Then, for all $\plan_1,\plan_2 \in \ACT^*$, $\modults,w \models \refbox{\plan_1}{\plan_2}\varphi$.
%Since $\ACT$ is not empty, we choose $\alpha_1, \alpha_2 \in \ACT^*$, and we have that $\modults,w \models \refbox{\alpha_1}{\alpha_2}\varphi$.
%Using \Cref{itm:serialityref} from \Cref{prop:ref-normal-serial}, $\modults,w \models \refdiam{\alpha_1}{\alpha_2}\varphi$.
%Thus, $\modults,w \models \arefdiam\varphi$.
%\end{enumerate}
%\end{proof}

%\begin{proposition}\label{prop:aref-mon-weak-intr}
%Let $\varphi,\psi$ be arbitrary $\Reflogic$-formulas.
%\begin{enumerate}
\item $\models \arefdiam \varphi \ra \arefdiam (\varphi \vee \psi)$ and $\models \arefbox \varphi \ra \arefbox (\varphi \vee \psi)$ (Monotonicity).
\item $\models \arefdiam (\varphi \wedge \psi) \ra \arefdiam \varphi$ and $\models \arefbox (\varphi \wedge \psi) \ra \arefbox \varphi$ (Weakening).
\item $\not\models \arefbox \varphi \ra \arefbox\arefbox \varphi$ (axiom 4). % (Positive Introspection).
\item $\not\models \neg\arefbox \varphi \ra \arefbox\neg\arefbox \varphi$ (axiom 5). % (Negative Introspection).
\end{enumerate}
\end{proposition}

\begin{proof}
%\begin{enumerate}
% \item Suppose $\modults,w \models \arefbox(\varphi \ra \psi)$ and $\modults,w \models \arefbox\varphi$ and let $\plan_1,\plan_2 \in \ACT^*$.
% We have that $\modults,w \models \refbox{\plan_1}{\plan_2}(\varphi \ra \psi)$ and $\modults,w \models \refbox{\plan_1}{\plan_2}\varphi$.
% Using \Cref{itm:distref} from \Cref{prop:ref-normal-serial}, $\modults,w \models \refbox{\plan_1}{\plan_2}\psi$.
% Thus, $\modults,w \models \arefbox\psi$.
% \item Suppose $\models \varphi$ and let $\plan_1,\plan_2 \in \ACT^*$.
% Using \Cref{itm:necessitationref} from \Cref{prop:ref-normal-serial}, $\models \refbox{\plan_1}{\plan_2}\varphi$.
% Thus, $\models \arefbox\varphi$.
% \item Suppose $\modults,w \models \arefbox\varphi$.
% Then, for all $\plan_1,\plan_2 \in \ACT^*$, $\modults,w \models \refbox{\plan_1}{\plan_2}\varphi$.
% Since $\ACT$ is not empty, we choose $\alpha_1, \alpha_2 \in \ACT^*$, and we have that $\modults,w \models \refbox{\alpha_1}{\alpha_2}\varphi$.
% Using \Cref{itm:serialityref} from \Cref{prop:ref-normal-serial}, $\modults,w \models \refdiam{\alpha_1}{\alpha_2}\varphi$.
% Thus, $\modults,w \models \arefdiam\varphi$.
% \end{enumerate}
Items 1 to 3 follows as $\arefdiam$ is a generalization of $\refdiam{\plan_1}{\plan_2}$ from the previous section. 
It is also easy to see that Monotonicity and Weakening are direct. For axiom~4, let us consider the model $\modults = \tup{\W,\R,\Unc,\V}$ depicted below, with $\Unc(i)\in\Unc$ be such that $\Unc(i):=\set{\set{a,b,c}}$.
\begin{center}
\begin{tikzpicture}[->, grow' = right, level/.style={sibling distance = 3em/#1}, level distance = 3.5em]
\node[state] at (0,2) (p) [label=left:$w$]{$p$};
\node[state] at (2,2) (q) {$q$};
\draw[] (p) node at (1,2) [above] {$a$} -- (q);

\node[state] at (0,1) (r) {$r$};
\node[state] at (2,1) (s) {$s$};
\draw[] (r) node at (1,1) [above] {$b$} -- (s);
\end{tikzpicture}
\end{center}

On the one hand, one can easily check that $\modults,w\models\arefbox\neg(\kh(p,q) \wedge \kh(r,s))$. On the other hand, we have that $\modults,w\models\refdiam{a}{b}\refdiam{b}{c} (\kh(p,q) \wedge \kh(r,s))$. Hence, we get $\modults,w\models\arefdiam\arefdiam(\kh(p,q) \wedge \kh(r,s))$.

For axiom~5, notice that $\modults,w \models \neg\arefbox \kh(p,q)$ but $\modults,w \models \neg\arefbox\neg\arefbox \kh(p,q)$, since $\modults,w \models \refdiam{a}{b}\arefbox \kh(p,q)$.
\end{proof}

Similarly, this modality preserves knowledge and can generate new one for propositional formulas.

\medskip

\begin{proposition}\label{prop:aref-preserves-gains}
Let $\psi,\varphi$ be propositional formulas. Then, 
\begin{enumerate}
\item\label{itm:aref:preservesknowledge} $\models \khi(\psi,\varphi) \ra \arefbox\khi(\psi,\varphi)$.
\item\label{itm:aref:gainsknowledge} If $\psi$ and $\varphi$ are satisfiable, then $\neg\khi(\psi,\varphi) \wedge \arefdiam\khi(\psi,\varphi)$ are satisfiable.
\end{enumerate}
\end{proposition}
\begin{proof}
Similar to \Cref{prop:ref-preserves-gains}.
\end{proof}

By definition, $\models \refdiam{\plan_1}{\plan_2}\varphi \implies \arefdiam\varphi$, but characterizing the exact expressivity relation between the two resulting logics requires further developments.
In particular, given the mismatch between the two languages ($\Reflogic$ is able to talk about specific plans whereas $\AReflogic$ is not), it does not seem trivial to give a translation from one logic to the other.
However, by using the same argument as in \Cref{prop:expref}, it is easy to show the following:

\medskip

\begin{proposition}\label{prop:exparef}
$\AReflogic$ is more expressive than $\KHilogic$.
\end{proposition}

\medskip

Again, we establish that uniform substitution fails in this logic, which in combination with its high expressivity, makes it difficult to axiomatize.

\medskip

\begin{proposition}\label{prop:substitution-aref}
    Uniform substitution fails in $\AReflogic$.
\end{proposition}

\begin{proof}
Similar to~\Cref{prop:substitution-ref}, and taking $p\to\arefdiam p$ as the original valid formula.
\end{proof}

The problem of defining dynamic operators and corresponding axiomatizations will be addressed in the next section. 
We finish the section by stating some expressivity connections between the
dynamic modalities we just discussed.

\medskip

\begin{proposition}\label{prop:explearn}
The following propositions are true:
\begin{enumerate}
\item\label{itm:lhkh} $\LHlogic$ is more expressive than $\KHilogic$.
\item\label{itm:lhref} $\LHlogic$ is not more expressive than $\Reflogic$.
\end{enumerate}
\end{proposition}
\begin{proof}
\Cref{itm:lhkh} is proved as \Cref{prop:expref}: the formula $\learn_i(p,q)$ distinghuishes the two \ultss.
For \Cref{itm:lhref} consider the two \ultss below:
\begin{center}
\begin{tabular}{l@{\qquad\quad}l}
\begin{tabular}{c}
    \begin{tikzpicture}[->]
    \node [state, label = {[label-state]left:$w$}] (w1) {$r$};
    \node [state, above right = 0.25em and 3em of w1] (w2) {$p$};
    \node [state, below right = 0.25em and 3em of w1] (w3) {};
    \path (w1) edge node [label-edge, above] {$a$} (w2)
                edge node [label-edge, below] {$b$} (w3);
    \node [ left = of w1] {$\modults$};
    \end{tikzpicture}
\end{tabular}
&
\begin{tabular}{c}
    \begin{tikzpicture}[->]
    \node [state, label = {[label-state]left:$w'$}] (w1) {$r$};
    \node [state, above right = 0.25em and 3em of w1] (w2) {$p$};
    \node [state, below right = 0.25em and 3em of w1] (w3) {};
    \path (w1) edge node [label-edge, above] {$c$} (w2)
                edge node [label-edge, below] {$d$} (w3);
    \node [right = 7em of w1] {$\modults'$};
    \end{tikzpicture}
\end{tabular}
\end{tabular} 
\end{center}
For each model, consider respective sets $\Unc(i)=\set{\set{a,b}}$ and $\Unc'(i)=\set{\set{c,d}}$.
Since $\LHlogic$ cannot explicitely talk about plans, $\modults,w$ and $\modults',w'$ are indistinguishable for it.
In $\Reflogic$, $\modults,w\models \dialhepis{a{\not\sim}b}\khi(r,p)$ and $\modults',w'\not\models\dialhepis{a{\not\sim}b}\khi(r,p)$.
\end{proof}

% 
% \begin{proof}
% The formula $\arefdiam\kh(p,q)$ distinguishes the two models in \Cref{prop:expref}. 
% \end{proof}

%  RF: 19/04/2022 Commented as I don't see it is needed
%
% \begin{definition}[Reachability]\label{def:reachability}
% Let $\modults = \tup{\W,\R,\Unc,\V}$ be \ultss. We write  $\splitstr{\Unc}{\Unc'}{}{}$ if there are $\plan_1,\plan_2$ such that  $\splitstr{\Unc}{\Unc'}{\plan_1}{\plan_2}$.
% We say that $\Unc'$ is finitely reachable from $\Unc$ if $\Unc\leadsto^* \Unc'$ (i.e., $\Unc'$ is reached via the reflexive-transitive closure of $\leadsto$). For $\Unc(i),\Unc'_i$,
% $\Unc(i)\leadsto\Unc'(i)$ (resp. $\Unc(i)\leadsto^*\Unc'(i)$) whenever $\Unc\leadsto\Unc'$  (resp., $\Unc\leadsto^*\Unc'$), $\Unc(i)\in\Unc$ and $\Unc'(i)\in\Unc'$.
% \end{definition}

% Note that we always have $\Unc\leadsto\Unc$, by taking since $\plan_1 = \plan_2$.

% \begin{definition}[Submodel]\label{def:submodel}
% Let $\modults = \tup{\W,\R,\Unc,\V}$ and $\modults' = \tup{\W',\R',\Unc',\V'}$ be \ultss, $\modults'$ is a submodel of $\modults$ if $\W' = \W$, $\R' = \R$, $\V' = \V$ and $\Unc\leadsto\Unc'$. 
% We write $\modults' \subseteq \modults$ if $\modults'$ is a submodel of $\modults$.
% Moreover, we denote this model as $\modults_{\Unc'}$.
% \end{definition}
%


% \bigandres{a}

% \begin{definition}
% Let $\modults=\tup{\W,\R,\Unc,\V}$ be an \ults and $U \subseteq \W$.
% $U$ is $\AReflogic$-definable if there is a formula $\varphi \in \AReflogic$ s.t. $U = \truthset{\modults}{\varphi}$.
% $U$ is propositionally-definable if there is a propositional formula $\varphi$ s.t. $U = \truthset{\modults}{\varphi}$.
% \end{definition}

% \begin{claim}
% For any $\modults=\tup{\W,\R,\Unc,\V}$, $\varphi \in \AReflogic$ and $\plan_1,\plan_2 \in \ACT^*$, there is $\Unc' \subseteq 2^{2^{(\ACT^*)}}$ s.t. $\splitstr{\Unc}{\Unc'}{\plan_1}{\plan_2}$ and $\truthset{\modults}{\refdiam{\plan_1}{\plan_2}\varphi} = \truthset{\modults^{\Unc}_{\Unc'}}{\varphi}$.
% \end{claim}
% \begin{proof}
% The case of propositional variables is direct.
% \begin{enumerate}
% \item If $\varphi = \neg\varphi_1$
% \item If $\varphi = \varphi_1 \wedge \varphi_2$
% \item If $\varphi = \khi(\varphi_1,\varphi_2)$, then either $\truthset{\modults}{\refdiam{\plan_1}{\plan_2}\varphi} = \W$ (we choose the witness $\Unc'$) or $\truthset{\modults}{\refdiam{\plan_1}{\plan_2}\varphi} = \emptyset$ (and we choose $\Unc'=\Unc$).
% Either way, the existence of $\Unc'$ is guaranteed.
% \item If $\varphi = \refdiam{\alpha_1}{\alpha_2}\varphi$
% \end{enumerate}
% \end{proof}


% \begin{claim}
% Let $\modults=\tup{\W,\R,\Unc,\V}$ be an \ults and $\varphi \in \AReflogic$ a formula.
% Then there is $\plan_1,\plan_2 \in \ACT^*$ s.t. $\truthset{\modults}{\arefdiam\varphi} = \truthset{\modults}{\refdiam{\plan_1}{\plan_2}\varphi}$.
% \end{claim}

% \begin{claim}
% Let $\modults=\tup{\W,\R,\Unc,\V}$ be an \ults and $\varphi \in \AReflogic$ a formula.
% $\truthset{\modults}{\varphi}$ is propositionally-definable.
% \end{claim}
% \begin{proof}
% The propositional variables and Boolean conectors cases are straightforward.
% Let $\varphi = \khi(\varphi_1,\varphi_2)$, then either $\truthset{\modults}{\varphi}$ is $\emptyset$ $(= \truthset{\modults}{\bot})$ or is $\W$ $(= \truthset{\modults}{\top})$.
% Thus, it is propositionally-definable.
% If $\varphi = \arefdiam \varphi_1$, then $w \in \truthset{\modults}{\varphi}$ iff $\modults,w \models \arefdiam \varphi_1$ iff $\modults,w \models \refdiam{\plan_1}{\plan_2}$

% \end{proof}

% \begin{definition}\label{def:notation-ref} 
% Let $\modults=\tup{\W,\R,\Unc,\V}$ be an \ults.
% For $\strategy \in 2^{\ACT^*}$, $U,T \subseteq \W$,
% \begin{itemize} 
%     \item write $U \ultsExecStrat{\strategy}_{\Unc} T$ {\;\iffdef\;}
%     $U \subseteq \stexec(\strategy)$ and $T = \R_{\strategy}(U)$;
%     \item write $U \ultsExecAgi_{\Unc} T$ {\;\iffdef\;}
%     if there is $\strategy \in \Unc(i)$ such that $\Unc(i)\in\Unc$ and $U \ultsExecStrat{\strategy}_{\Unc} T$.
% \end{itemize}

% Let $Z \subseteq (\W \times 2^{2^{(\ACT^*)}}) \times (\W' \times 2^{2^{(\ACT'^*)}})$ be a relation, then:
% \begin{itemize}
%     \item For $u \in \W$ and $U \subseteq \W$, define:
    
%     \begin{nscenter}
%     \begin{tabular}{@{}c@{}}
%     $Z(u,\Unc,\Unc') := \csetsc{u' \in \W'}{(u,\Unc)Z(u',\Unc')}$;
%     $Z(U,\Unc,\Unc') := \bigcup_{u \in U} Z(u,\Unc,\Unc')$.
%     \end{tabular}
%     \end{nscenter}
    
%     \item For $u' \in \W$ and $U' \subseteq \W'$, define:
    
%     \begin{nscenter}
%     \begin{tabular}{@{\!\!\!\!\!\!\!\!}c@{}}
%     $Z^{-1}(u',\Unc,\Unc') := \csetsc{u \in \W}{(u,\Unc)Z(u',\Unc')}$;
%     $Z^{-1}(U',\Unc,\Unc') := \bigcup_{u' \in U'} Z(u',\Unc,\Unc')$.
%     \end{tabular}
%     \end{nscenter}
%     \end{itemize}
% \end{definition}

% \begin{definition}[$\AReflogic$-bisimulation]\label{def:bisim-aref}
% Let $\modults=\tup{\W,\R,\Unc,\V}$ be an \ults over $\ACT$ and let $\modults'=\tup{\W',\R',\Unc',\V'}$ be an \ults over $\ACT'$.  
% A non-empty $Z \subseteq (\W \times 2^{2^{(\ACT^*)}}) \times (\W' \times 2^{2^{(\ACT'^*)}})$ is called an \emph{$\Reflogic$-bisimulation} between $\modults$ and $\modults'$ if and only if $(w,\Uncother)Z(w',\Uncother')$ implies:
% \begin{description} 
% \item[Atom:] $\V(w)=\V'(w')$.

% \item[$\kh$-Zig:] for any $U \subseteq \W$ that is $\AReflogic$-definable in $\modults_{\Uncother}$, if $U \ultsExecAgi_{\Uncother} T$ for some $T \subseteq \W$, then there is $T' \subseteq \W'$ s.t.
% \begin{ltabular}{l@{\;}l@{\qquad\qquad}l@{\;}l}
% \ITM{i}  & $Z(U,\Uncother,\Uncother') \ultsExecAgi_{\Uncother'} T'$, and &
% \ITM{ii} & $T' \subseteq Z(T,\Uncother,\Uncother')$.
% \end{ltabular}

% \item[$\kh$-Zag:] for any $U' \subseteq \W'$ that is $\AReflogic$-definable in $\modults'_{\Uncother'}$, if $U' \ultsExecAgi_{\Uncother'} T'$ for some $T' \subseteq \W'$, then there is $T \subseteq \W$ s.t.
% \begin{ltabular}{l@{\;}l@{\qquad\qquad}l@{\;}l}
% \ITM{i}  & $Z^{-1}(U',\Uncother,\Uncother') \ultsExecAgi_{\Uncother} T$, and &
% \ITM{ii} & $T \subseteq Z^{-1}(T',\Uncother,\Uncother')$.
% \end{ltabular}

% \item[$\A$-Zig:] for all $u$ in $\W$ there exists $u'$ in $\W'$ such that $(u,\Uncother)Z(u',\Uncother')$.

% \item[$\A$-Zag:] for all $u'$ in $\W'$ there exists$u$ in $\W$ such that $(u,\Uncother)Z(u',\Uncother')$.

% \item[$\arefdiam$-Zig:] for all $\plan_1,\plan_2\in\ACT^*$ and $\Uncter\subseteq 2^{(\ACT^*)}$ such that $\splitstr{\Uncother}{\Uncter}{\plan_1}{\plan_2}$, there are ${\plan'_1,\plan'_2\in\ACT'^*}$ and $\Uncter'\subseteq 2^{(\ACT'^*)}$ such that $\splitstr{\Uncother'}{\Uncter'}{\plan'_1}{\plan'_2}$ and $(w,\Uncter)Z(w',\Uncter')$.

% \item[$\arefdiam$-Zag:] for all $\plan_1',\plan_2'\in\ACT'^*$ and $\Uncter'\subseteq 2^{(\ACT'^*)}$ such that $\splitstr{\Uncother'}{\Uncter'}{\plan_1'}{\plan_2'}$, there are ${\plan_1,\plan_2\in\ACT^*}$ and $\Uncter\subseteq 2^{(\ACT^*)}$ such that $\splitstr{\Uncother}{\Uncter}{\plan_1}{\plan_2}$ and $(w,\Uncter)Z(w',\Uncter')$.
% \end{description}

% We write $\modults,w \bisim_\AReflogic \modults',w'$ when there is an $\AReflogic$-bisimulation $Z$ between $\modults$ and $\modults'$ such that $(w,\Unc)Z(w',\Unc')$.
% \end{definition}

% % \begin{proposition}\label{prop:refbisim-implies-arefbisim}
% % Let $\modults$ and $\modults'$ are \ultss such that $\ACT_{\cap}=\ACT\cap\ACT' \neq \emptyset$ and $\Unc,\Unc' \in 2^{\ACT_{\cap}}$.
% % If ($\modults,w \bisim_\Reflogic \modults',w'$), then ($\modults,w \bisim_\AReflogic \modults',w'$).
% % \end{proposition}
% % \begin{proof}
% % Suppose there are $\plan_1$, $\plan_2$ in $\ACT^*$ and $\overline{\Unc}(0) \in \REACH{\Unc}$ such that $\splitstr{\Unc(0)}{\overline{\Unc}(0)}{\plan_1}{\plan_2}$.
% % Then by \textbf{$\dialhepis{\plan_1{\not\sim}\plan_2}$-Zig}, there are $\plan'_1=\plan_1$, $\plan'_2=\plan_2$ in $\ACT'^*=\ACT^*$ and $\overline{\Unc}'_0 \in \REACH{\Unc'}$ such that $\splitstr{\Unc'_0}{\overline{\Unc}'_0}{\plan'_1}{\plan'_2}$ and $(w,\overline{\Unc}(0))Z(w',\overline{\Unc}'_0)$.
% % With this, we have \textbf{$\arefdiam$-Zig}.
% % With a similar reasoning, we can prove \textbf{$\arefdiam$-Zig}.
% % \end{proof}

% % \begin{definition}[$\AReflogic$-equivalence]\label{def:equiv-aref}
% % The pointed \ultss $\modults, w$ and $\modults', w'$ are \emph{$\AReflogic$-equivalent} (written $\modults, w \modequiv_\AReflogic \modults', w'$) if and only if $\modults, w \models \varphi$ iff $\modults', w' \models \varphi$ for all $\varphi \in \AReflogic$.
% % \end{definition}

% Then we are in position to prove the intended theorem (notice that $\modequiv_\AReflogic$ is defined simply by extending~\Cref{def:equiv-kh} to $\AReflogic$-formulas).

% \begin{theorem}[Invariance for $\AReflogic$]\label{th:arefbisim-to-arefequiv}
% Let $\modults, w$ and $\modults', w'$ be pointed \ultss.
% Then, $\modults,w \bisim_\AReflogic \modults', w' \text{ implies }\modults,w \modequiv_\AReflogic \modults',w'$.
% \end{theorem}
% \begin{proof}
% The proof of $\AReflogic$-equivalence is by structural induction on $\AReflogic$-formulas.
% The goal is to prove that $(u,\Unc(0))Z(u',\Unc(0)')$ (with the induced submodels $\modults_{\Unc(0)} \subseteq \modults$ and $\modults_{\Unc(0)'}' \subseteq \modults'$) implies $\modults_{\Unc(0)},u \modequiv_\Reflogic \modults_{\Unc(0)'}',u'$.
% The cases for atomic propositions and Boolean operators are standard.

% For formulas of the form $\kh(\psi,\varphi)$, we have a similar approach as in Theorem \ref{th:khbisim-to-khequiv}.
% Suppose $u \in \truthset{\modults_{\Unc(0)}}{\kh(\psi,\varphi)}$; then, there is $\strategy \in \Unc(0)$
% s.t. $\truthset{\modults_{\Unc(0)}}{\psi} \ultsExecStrat{\strategy}_{\Unc(0)} T$ and
% $T=\R_{\strategy}(\truthset{\modults_{\Unc(0)}}{\psi}) \subseteq \truthset{\modults_{\Unc(0)}}{\varphi}$.
% First, note how $Z(\truthset{\modults_{\Unc(0)}}{\psi},\Unc(0),\Unc(0)') = \truthset{\modults_{\Unc(0)'}'}{\psi}$.
% Indeed, $(\subseteq)$ if $v' \in Z(\truthset{\modults_{\Unc(0)}}{\psi},\Unc(0),\Unc(0)')$, then there is $v \in \truthset{\modults_{\Unc(0)}}{\psi}$ such that $(v,\Unc(0))Z(v',\Unc(0)')$; thus, from IH we have $v' \in \truthset{\modults_{\Unc(0)'}'}{\psi}$.
% $(\supseteq)$ If $v' \in \truthset{\modults_{\Unc(0)'}'}{\psi}$, by $\A$-Zag there is $v$ with $(v,\Unc(0))Z(v',\Unc(0)')$; thus, from IH we have $v \in \truthset{\modults_{\Unc(0)}}{\psi}$.
% Hence, $v' \in Z(\truthset{\modults_{\Unc(0)}}{\psi},\Unc(0),\Unc(0)')$.

% The set $\truthset{\modults_{\Unc(0)}}{\psi}$ is $\Reflogic$-definable.
% % \textcolor{red}{(we are not sure if it is also propositionally definable; if not, we can redefine bisimulations; from here the proof assumes it is the case) and thus propositionally definable}. \raul{creo que como la modalidad dinamica es global, esto deberia valer de todos modos, pero en el modelo adecuado}
% Thus, there are $\strategy \in \Unc(0)$ and $T \subseteq \W$ such that $\truthset{\modults_{\Unc(0)}}{\psi} \ultsExecStrat{\strategy}_{\Unc(0)} T$, so $\truthset{\modults_{\Unc(0)}}{\psi} \ultsExec_{\Unc(0)} T$.
% Then, from the latter and clause $\kh$-Zig, there is $T' \subseteq \W'$ such that

% \begin{enumerate}
% \item $Z(\truthset{\modults_{\Unc(0)}}{\psi},\Unc(0),\Unc(0)') \ultsExec_{\Unc(0)'} T'$ (so $\truthset{\modults_{\Unc(0)'}'}{\psi} \ultsExec_{\Unc(0)'} T'$, by the result above), and
% \item\label{itm:ii} $T' \subseteq Z(T,\Unc(0),\Unc(0)')$.
% \end{enumerate}

% Now, we know that $T \subseteq \truthset{\modults_{\Unc(0)}}{\varphi}$, that is, $v \in T$ implies $v \in \truthset{\modults_{\Unc(0)}}{\varphi}$.
% But then, by IH, $v \in T$ and $(v,\Unc(0))Z(v',\Unc(0)')$ implies $v' \in \truthset{\modults_{\Unc(0)'}'}{\varphi}$.
% Thus, $Z(T,\Unc(0),\Unc(0)') \subseteq \truthset{\modults_{\Unc(0)'}'}{\varphi}$.
% This, together with \itm{\ref{itm:ii}} from the previous paragraph, yields $T' \subseteq \truthset{\modults_{\Unc(0)'}'}{\varphi}$.
% Thus, we have both $\truthset{\modults_{\Unc(0)'}'}{\psi} \ultsExec_{\Unc(0)'} T'$ (there is an strongly executable strategy which, from $\psi$-worlds, reaches only $T'$-worlds) and $T' \subseteq \truthset{\modults_{\Unc(0)'}'}{\varphi}$ (every $T'$-world satisfies $\varphi$); hence, $u' \in \truthset{\modults_{\Unc(0)}'}{\kh(\psi,\varphi)}$.

% The direction from $u' \in \truthset{\modults_{\Unc(0)'}'}{\kh(\psi,\varphi)}$ to $u \in \truthset{\modults_{\Unc(0)}}{\kh(\psi,\varphi)}$ follows a similar argument, using $\A$-Zig and $\kh$-Zag instead.

% For the case $\arefdiam\varphi$, suppose $\modults_{\Unc(0)},u \models \arefdiam \varphi$.
% Then, there are $\plan_1,\plan_2 \in \ACT^*$ $\modults,w\models\dialhepis{\plan_1{\not\sim}\plan_2}\varphi$.
% That is, there is $\overline{\Unc}(0) \in \REACH{\Unc}$ s.t.
% $\splitstr{\Unc(0)}{\overline{\Unc}(0)}{\plan_1}{\plan_2}$ and
% $\modults^{\Unc(0)}_{\overline{\Unc}(0)} = \modults_{\overline{\Unc}(0)},u \models \varphi$.
% Then, by $\arefdiam$-Zig, there are $\plan'_1,\plan'_2 \in \ACT'^*$ and
% $\overline{\Unc}(0)' \in \REACH{\Unc'}$ such that
% $\splitstr{\Unc(0)'}{\overline{\Unc}(0)'}{\plan'_1}{\plan'_2}$
% and $(u,\overline{\Unc}(0))Z(u',\overline{\Unc}(0)')$.
% By IH, $\modults'_{\overline{\Unc}(0)'} = (\modults')^{\Unc(0)'}_{\overline{\Unc}(0)'},u' \models \varphi$.
% And by the definition of $\dialhepis{\plan'_1{\not\sim}\plan'_2}$,
% $\modults',u' \models \dialhepis{\plan'_1{\not\sim}\plan'_2}\varphi$.
% With this, $\modults_{\Unc'_0},u \models \arefdiam \varphi$.

% The direction from $\modults'_{\Unc(0)'},u' \models \arefdiam \varphi$
% to $\modults_{\Unc(0)},u \models \arefdiam \varphi$ follows a similar
% argument, using $\arefdiam$-Zag instead.
% \end{proof}

% % \begin{proposition}
% % Let $\Unc \in 2^{\ACT}$. If $\Unc$ is finite and every $\strategy \in \Unc$, $\strategy$ is finite, then the set $\REACH{\Unc}$ is finite.
% % \end{proposition}
% % \begin{proof}
% % $\REACH{\Unc}$ is at most $\pow(\bigcup_{\strategy \in \Unc} \strategy)$.
% % Since $\bigcup_{\strategy \in \Unc} \strategy$ is finite, then $\REACH{\Unc}$ it is also finite.
% % \end{proof}

% % \begin{theorem}\label{th:arefequiv-to-arefbisim}
% % Let $\modults, w$ and $\modults', w'$ be pointed \ultss. If $\modults$ and $\modults'$ are
% % finite then $\modults,w \modequiv_\AReflogic \modults', w'$ implies
% % $\modults,w \bisim_\AReflogic \modults', w'$.
% % \end{theorem}

% % \begin{proof}
% % As it happened in \Cref{th:khequiv-to-khbisim}, the restriction to finite models (instead of just ``image finite'') is needed because the universal modality is definable in $\KHilogic$.
% % Take $\modults=\tup{\W,\R,\Unc,\V}$ and $\modults'=\tup{\W',\R',\Unc',\V'}$.
% % The strategy is to show that the relation $\modequiv$ is already a bisimulation.
% % Thus, define $Z := \csetc{(v,\Unc(0),v',\Unc(0)')}{(\W \times \REACH{\Unc}) \times (\W' \times \REACH{\Unc'})}{\modults_{\Unc(0)}, v \modequiv_\AReflogic \modults'_{\Unc(0)'}, v'}$; in order to show that $Z$ satisfies the requirements, take any $(w,\Unc(0),w'\Unc(0)') \in Z$.

% % \begin{itemize}
% % \item \textbf{Atom}. States $w$ and $w'$ coincide in all $\AReflogic$-formulas, and in particular, in all atoms.

% % \item \textbf{$\A$-Zig}. Take $v \in \W$ and suppose, for the sake of a contradiction, that there is no $v' \in \W'$ such that $(v,\Unc(0))Z(v',\Unc(0)')$.
% % Then, from $Z$'s definition, for each $v_i'\in \W' = \set{v'_1,\ldots,v'_n}$ (recall: $\modults'_{\Unc(0)'}$ is finite) there is a $\Reflogic$-formula $\theta_i$ s.t. $\modults_{\Unc(0)},v \models \theta_i$ but $\modults'_{\Unc(0)'},v'_i \not\models \theta_i$.
% % Now take $\theta := \theta_1 \land \cdots \land \theta_n$.
% % Clearly, $\modults_{\Unc(0)'},v \models \theta$; however, $\modults'_{\Unc(0)'},v_i' \not\models \theta$ for each $v_i'\in \W'$, as each one of them makes `its' conjunct $\theta_i$ false.
% % Then, by taking $\E \varphi := \lnot \A \lnot \varphi$ we have $\modults_{\Unc(0)},w \models \E \theta$ but $\modults'_{\Unc(0)'},w' \not\models \E \theta$, contradicting $(w,\Unc(0))Z(w',\Unc(0)')$.

% % \item \textbf{$\A$-Zag}. Analogous to the $\A$-Zig case.

% % \item \textbf{$\kh$-Zig}. Take \textcolor{red}{any propositionally definable set} $\truthset{\modults_{\Unc(0)}}{\psi} \subseteq \W$ (thus, $\psi$ is propositional), and suppose $\truthset{\modults_{\Unc(0)}}{\psi} \ultsExec_{\Unc(0)} T$ for some $T \subseteq \W$.
% % We need to find a $T' \subseteq \W'$ s.t.

% % \begin{enumerate}
% % \item $Z(\truthset{\modults_{\Unc(0)}}{\psi},\Unc(0),\Unc(0)') \ultsExec_{\Unc(0)'} T'$ and
% % \item $T' \subseteq Z(T,\Unc(0),\Unc(0)')$.
% % \end{enumerate}

% % First, note that $\truthset{\modults'_{\Unc(0)'}}{\psi} = Z(\truthset{\modults_{\Unc(0)}}{\psi})$. For $\boldsymbol{(\supseteq)}$, suppose $u' \in Z(\truthset{\modults_{\Unc(0)}}{\psi})$.
% % Then, there is $u \in \truthset{\modults_{\Unc(0)}}{\psi}$ such that $(u,\Unc(0))Z(u',\Unc(0)')$, and therefore, from $Z$'s definition, $u' \in \truthset{\modults'_{\Unc(0)'}}{\psi}$.
% % For $\boldsymbol{(\subseteq)}$, suppose $u' \in \truthset{\modults'_{\Unc(0)'}}{\psi}$.
% % From $\A$-Zag, there is $u \in \W$ such that $(u,\Unc(0))Z(u',\Unc(0)')$; then, from $Z$'s definition, $u \in \truthset{\modults_{\Unc(0)}}{\psi}$ so $u' \in Z(\truthset{\modults_{\Unc(0)}}{\psi})$.

% % Thus, we are actually looking for a $T' \subseteq \W'$ such that

% % \begin{enumerate}
% % \item $\truthset{\modults'_{\Unc(0)'}}{\psi} \ultsExec_{\Unc(0)'} T'$ and
% % \item $T' \subseteq Z(T,\Unc(0),\Unc(0)')$
% % \end{enumerate}

% % Consider the two alternatives.
% % \begin{enumerate}
% % \item If $\truthset{\modults_{\Unc(0)}}{\psi} = \emptyset$, then $\truthset{\modults'_{\Unc(0)'}}{\psi} = Z(\truthset{\modults_{\Unc(0)}}{\psi}) = \emptyset$ and hence taking $T' = \emptyset$ does the job: clearly,
% % \begin{enumerate}
% % \item $\emptyset \ultsExec_{\Unc(0)'} \emptyset$ (as $\Unc(0)' \neq \emptyset$) and
% % \item $\emptyset \subseteq Z(T,\Unc(0),\Unc(0)')$.
% % \end{enumerate}

% % \item Otherwise, $\truthset{\modults_{\Unc(0)}}{\psi} \neq \emptyset$ and $T \neq \emptyset$.
% % Then, from $\A$-Zig it follows that $Z(\truthset{\modults_{\Unc(0)}}{\psi}) = \truthset{\modults'_{\Unc(0)'}}{\psi} \neq \emptyset$.
% % In order to show that there is a $T' \subseteq \W'$ satisfying both {\itshape \bfseries (i)} and {\itshape \bfseries (ii)}, the argument is by contradiction.
% % Thus, suppose there is no $T'$ with the given requirements: each $T' \subseteq \W'$ satisfying $\truthset{\modults'_{\Unc(0)'}}{\psi} \ultsExec_{\Unc(0)'} T'$ is such that $T' \not\subseteq Z(T)$.
% % In other words, for every $T' \subseteq \W'$ satisfying $\truthset{\modults'_{\Unc(0)'}}{\psi} \ultsExec_{\Unc(0)'} T'$ there is a $v'_{T'} \in T'$ such that there is no $v \in T$ with $(v,\Unc(0))Z(v'_{T'},\Unc(0)')$.
% % Due to the definition of $Z$, this means that for each $v \in T$ there is a formula $\theta^v_{T'}$ such that $\modults_{\Unc(0)}, v \models \theta^v_{T'}$ but $\modults'_{\Unc(0)'}, v'_{T'} \not\models\theta^v_{T'}$.
% % Since the models are finite, one can define both

% % \begin{nscenter}
% % \renewcommand{\arraystretch}{1.8}
% % \begin{tabular}{ccc}
% % $\displaystyle \theta_{T'} := \bigvee_{v \in T} \theta^v_{T'}$ & and &
% % $\displaystyle \theta :=
% % \bigwedge_{\csetsc{T' \subseteq \W'}{\truthset{\modults'_{\Unc(0)'}}{\psi} \ultsExec_{\Unc(0)'} T'}} \theta_{T'}$.
% % \end{tabular}
% % \renewcommand{\arraystretch}{1}
% % \end{nscenter}

% % Since $T \neq \emptyset$, $\theta_{T'}$ does not collapse to $\bot$.
% % However, $\theta$ can collapse to $\top$ since $\csetsc{T' \subseteq \W'}{\truthset{\modults'_{\Unc(0)}}{\psi} \ultsExec_{\Unc(0)'} T'}$ might be empty.
% % This is what creates the following two cases.

% % \begin{itemize}
% % \item Suppose $\csetsc{T' \subseteq \W'}{\truthset{\modults'_{\Unc(0)}}{\psi} \ultsExec_{\Unc(0)'} T'} = \emptyset$.
% % Then, consider the formula $\kh(\psi,\top)$.
% % Since $\truthset{\modults_{\Unc(0)}}{\psi} \ultsExec_{\Unc(0)'} T$ and $T \subseteq \W = \truthset{\modults_{\Unc(0)}}{\top}$, it follows that $\modults_{\Unc(0)},w \models \kh(\psi, \top)$.
% % However, $\modults'_{\Unc(0)'},w' \not\models \kh(\psi, \top)$ as, according to this case, there is no $T' \subseteq \W'$ with $\truthset{\modults'_{\Unc(0)'}}{\psi} \ultsExec_{\Unc(0)'} T'$.
% % This contradicts the $\AReflogic$-equivalence of $w$ and $w'$.

% % \item Suppose $\csetsc{T' \subseteq \W'}{\truthset{\modults'_{\Unc(0)}}{\psi} \ultsExec_{\Unc(0)'} T'} \neq \emptyset$.
% % Then, $\theta$ does not collapse to $\top$.
% % Note how $\modults_{\Unc(0)}, v \models \theta$ for all $v \in T$, as every such $v$ satisfies its `own' disjunct $\theta^v_{T'}$ in each conjunct $\theta_{T'}$.
% % Hence, from $\truthset{\modults_{\Unc(0)}}{\psi} \ultsExec_{\Unc(0)} T$ it follows that $\modults_{\Unc(0)},w \models \kh(\psi, \theta)$.
% % However, for every $T'$ satisfying  $\truthset{\modults'_{\Unc(0)'}}{\psi} \ultsExec_{\Unc(0)'} T'$ the state $v'_{T'}$ that cannot be matched with any state $v \in T$ makes all disjuncts $\theta^v_{T'}$ in $\theta_{T'}$ false, thus falsifying $\theta_{T'}$ and therefore falsifying $\theta$ too.
% % In other words, every $T' \subseteq \W'$ with $\truthset{\modults'_{\Unc(0)'}}{\psi} \ultsExec_{\Unc(0)'} T'$ contains a state $t'_{T'}$ with $\modults'_{\Unc(0)'}, t'_{T'} \not\models \theta$, that is, $\truthset{\modults'}{\psi} \ultsExec_{\Unc(0)'} T'$ implies $T' \not\subseteq \truthset{\modults_{\Unc(0)}}{\theta}$.
% % Hence, using again the fact that $\kh$-formulas are global, $\modults'_{\Unc(0)'},w' \not\models \kh(\psi, \theta)$, contradicting the $\AReflogic$-equivalence of $w$ and $w'$.
% % \end{itemize}

% % \end{enumerate}

% % \item \textbf{$\kh$-Zag}. Analogous to the $\kh$-Zig case.

% % \item \textbf{$\arefdiam$-Zig}.
% % Let $\plan_1$, $\plan_2$ in $\ACT^*$ and $\overline{\Unc}(0)$ in $\REACH{\Unc}$ such that $\splitstr{\Unc(0)}{\overline{\Unc}(0)}{\plan_1}{\plan_2}$.
% % Suppose there are no $\plan'_1$, $\plan'_2$ in $\ACT'^*$ and $\overline{\Unc}'_0$ in $\REACH{\Unc'}$ such that $\splitstr{\Unc'_0}{\overline{\Unc}'_0}{\plan_1}{\plan_2}$ and $(w,\overline{\Unc}(0))Z(w',\overline{\Unc}'_0)$.
% % That is, for all $\plan'_1$, $\plan'_2$ in $\ACT'^*$ and $\overline{\Unc}'_0 \in \REACH{\Unc'}$ such that $\splitstr{\Unc'_0}{\overline{\Unc}'_0}{\plan_1}{\plan_2}$, $(w,\overline{\Unc}(0),w',\overline{\Unc}'_0) \not\in Z$. By definition of $Z$, let $\theta_{\overline{\Unc}'_0}$ such that $\modults'_{\overline{\Unc}'_0}, w \models \theta_{\overline{\Unc}'_0}$ but $\modults_{\overline{\Unc}(0)}, w \not\models \theta_{\overline{\Unc}'_0}$.
% % Let $\theta$ the finite disjuction of all $\theta_{\overline{\Unc}'_0}$ (recall: $\REACH{S'}$ is finite), then $\modults'_{\Unc'_0}, w \models \arefbox \theta$ but $\modults_{\Unc(0)}, w \not\models \arefbox \theta$ with $\overline{\Unc}(0)$ as the witness.

% % \item \textbf{$\arefdiam$-Zag}. Analogous to the $\arefdiam$-Zig case.

% % \end{itemize}
% % \end{proof}

% % \begin{theorem}\label{aref-not-in-ref}
% % $\AReflogic \not\leq \Reflogic$ (i.e., $\Reflogic$ is not more expressive than $\AReflogic$).
% % \end{theorem}
% % \begin{proof}
% % Sketch idea: For a certain $\ACT$, $\arefdiam \varphi$ if and only iff $\bigvee_{\plan_1,\plan_2 \in \ACT^*} \dialhepis{\plan_1{\not\sim}\plan_2}\varphi$.
% % There should be no finite formula in $\Reflogic$ $\psi$ such that $\bigvee_{\plan_1,\plan_2 \in \ACT^*} \dialhepis{\plan_1{\not\sim}\plan_2}\varphi$ if and only iff $\psi$.
% % \end{proof}
% % 
% % \begin{theorem}\label{aref-not-in-ref}
% % $\Reflogic \not\leq \AReflogic$ (i.e., $\AReflogic$ is not more expressive than $\Reflogic$).
% % \end{theorem}
% % \begin{proof}
% % Sketch idea: Find $\modults$ and $\modults'$ \ultss such that $\ACT_{\cap}=\ACT\cap\ACT' \neq \emptyset$ and $\Unc,\Unc' \in 2^{\ACT_{\cap}}$ such that ($\modults,w \bisim_\AReflogic \modults',w'$) but ($\modults,w \not\modequiv_\Reflogic \modults',w'$).
% % With this, ($\modults,w \modequiv_\AReflogic \modults',w'$).
% % Then pick a certain formula of $\Reflogic$.
% % \end{proof}
% % 
% % \begin{corollary}
% % $\Reflogic$ and $\AReflogic$ are incomparable.
% % \end{corollary}
