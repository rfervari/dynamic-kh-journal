%\subsubsection{Decidability via filtrations}

Now we proceed to define the preliminaries for proving decidability of the satisfiability problem of $\KHiMLlogic$. In order to do so, we will use a \emph{filtration} technique (see e.g.,~\cite{mlbook}). This method enables us, from a given model and a set of formulas, to construct another model (the `filtration') which satisfies the same formulas from the input set. Interestingly, if we depart from a finite set of formulas, the filtration is a finite model, which is a crucial property to obtain decidability. 
Notice that the filtration we will provide, generalizes the one introduced in~\cite{AFSVQ23report} for $\KHilogic$, by also handling the basic modal logic modality $[a]$. Moreover, in~\cite{AFSVQ23report} it is not shown that a filtration always exists for a model and a set of formulas, which is a missing ingredient in the proof. This aspect is fixed here too.

We start by introducing two relations that, given a set of formulas $\Sigma$ and a model $\modults$, will be crucial to define a proper notion of filtration.

\medskip

\begin{definition}[$\Sigma$-equivalence]
Let $\modults=\tup{\W,\R,\Unc,\V}$ be an \ults and let $\Sigma$ be a set of $\KHiMLlogic$-formulas closed under subformulas.
Define the equivalence relations ${\modequiv_\Sigma} \subseteq \W\times\W$ and ${\planequiv_\Sigma} \subseteq \Unc_\AGT \times \Unc_\AGT$ (with $\Unc_\AGT := \bigcup_{i \in \AGT}\Unc(i)$) as:
\begin{center}
\begin{tabular}{lcl}
$w \modequiv_\Sigma v$ &  \iffdef &
for all $\psi \in \Sigma$, $\modults,w\models \psi$ iff $\modults,v\models \psi$,\\
$\plans \planequiv_\Sigma \plans'$ & \iffdef &
for all $i \in \AGT$ and $\khi(\psi,\varphi) \in \Sigma$,  $\plans$ is a witness for \\ 
& & $\khi(\psi,\varphi)$ in $\modults$ iff $\plans'$ is a witness for $\khi(\psi,\varphi)$ in $\modults$.
\end{tabular}
\end{center}

Given $w \in \W$ and $\plans \in \Unc_\AGT$, we define their $\Sigma$-equivalence class:
\begin{center}
$[w]_\Sigma := \setof{v \in \W}{w \modequiv_\Sigma v}$; \qquad  $[\plans]_\Sigma := \setof{\plans' \in 2^{\ACT^*}}{\plans \planequiv_\Sigma \plans'}$.
\end{center}
\end{definition}

\medskip

Although the notation $[\_]_\Sigma$ is overloaded, its argument will always disambiguate its use.

\medskip

\begin{definition}\label{def:filtration-actions}
Let $\modults = \tup{\W,\R,\Unc,\V}$ be an \ults defined over a denumerable set of actions $\ACT$, and let $\Sigma$ be a set of $\KHiMLlogic$-formulas that is closed under subformulas.
Define
\begin{itemize}
\item $\ACT^{\Sigma}_i := \setof{a_{[\plans]_\Sigma}}{\plans \in \Unc(i) \text{ is a witness of some } \khi(\psi,\varphi) \in \Sigma \text{ in } \modults}$,
\item $\ACT^\Sigma_{\square} := \setof{a \in \ACT}{\mlbox{a}\varphi \in \Sigma}$, and
\item $\ACT^\Sigma := \bigcup_{i \in \AGT} \ACT^\Sigma_i \cup \ACT^\Sigma_{\square}\cup \set{a_\emptyset}$.
\end{itemize} 
\end{definition}

\medskip

The idea behind the definition of $\ACT^{\Sigma}_i$ is that, for each $\khi(\psi,\varphi) \in \Sigma$ that is true at $\modults$, we consider an action mimicking the behaviour of those sets of plans $\plans$ that witness the satisfiability of $\khi(\psi,\varphi)$ in $\modults$. In turn, $\ACT^\Sigma_{\square}$ takes into account those action symbols that are explicitely used in formulas of $\Sigma$. Finally, $a_\emptyset$ will be used to guarantee the existence of an action in the set $\ACT^\Sigma$, as we will see without provoking any side effect. 
Now, we are in position of defining the notion of filtration. 

\medskip

\begin{definition}[Filtration of $\modults$ through $\Sigma$] \label{def:filtrations}
Let $\modults = \tup{\W,\R,\Unc,\V}$ be an \ults defined over $\ACT$; let $\Sigma$ be a set of $\KHiMLlogic$-formulas that is closed under subformulas.
An \ults defined over $\ACT^\Sigma$, $\modults^f = \tup{\W^f,\R^f,\Unc^f,\V^f}$, is \emph{a filtration of $\modults$ through~$\Sigma$} if and only if it satisfies the following conditions:

\begin{enumerate}
\item $\W^f := \setof{[w]_\Sigma}{w \in \W}$;
\item $\V^f([w]_\Sigma) := \setof{p \in \Sigma}{\modults,w\models p}$;
\item $\Unc(i)^f := \setof{\set{a_{[\plans]_{\Sigma}}}}{a_{[\plans]_{\Sigma}} \in \ACT^\Sigma_i} \cup \set{\set{a_\emptyset}}$ for each $i \in \AGT$;
\item \label{item:khiml-fil} $\R^f_{a_\emptyset} = \emptyset$ and for all $a_{[\plans]_\Sigma} \in \ACT^\Sigma_i$, there exists a non empty set $G_{[\plans]_\Sigma} \subseteq [\plans]_\Sigma$ such that $([w]_\Sigma,[v]_\Sigma) \in \R^f_{a_{[\plans]_\Sigma}}$ iff
\begin{itemize}
\item for all $w' \in [w]_\Sigma$ and $\plans' \in G_{[\plans]_\Sigma}$, we have $w' \in \stexec(\plans')$, and
\item there are $w'' \in [w]_\Sigma$, $v'' \in [v]_\Sigma$ and $\plans'' \in G_{[\plans]_\Sigma}$ such that $(w',v') \in \R_{\plans''}$;
\end{itemize}
\item \label{item:khiml-fil-a1} for all $a \in \ACT^\Sigma_\square$, if $(w,v) \in \R_a$, then $([w]_\Sigma,[v]_\Sigma) \in \R^f_a$; and
\item \label{item:khiml-fil-a2} for all $a \in \ACT^\Sigma_\square$, if $([w]_\Sigma,[v]_\Sigma) \in \R^f_a$ then for all $\mlbox{a} \varphi \in \Sigma$, if $\modults,w \models \mlbox{a}\varphi$ then $\modults,v \models \varphi$.
\end{enumerate}
\end{definition}

\medskip

\Cref{def:filtrations} deserves further comments.
The filtration is defined similarly as for the basic modal logic (see, e.g.,~\cite{mlbook}).
The most significant difference is the addition of a denumerable set of actions and relations (\Cref{item:khiml-fil}) since we use the set $\ACT^\Sigma$ as the set of action names.
As a consequence, the actions an agent $i$ considered in $\Unc^f(i)$ are separated from the actions defined in $\ACT$.
Moreover, if $\Sigma$ is finite, the relation $\planequiv_\Sigma$ enables us to obtain a finite set of witnesses for the formulas $\khi(\psi,\varphi) \in \Sigma$, from which (together with $\modequiv_\Sigma$) we also get that $\ACT^\Sigma$ and $\W^f$ are finite.
The function $\Unc^f(i)$ is well-defined since it is made only of singletons. Specially, if there is no $\plans \in \Unc(i)$ such that $\plans$ is a witness for some $\khi(\varphi,\psi) \in \Sigma$, $\Unc^f(i)\neq \emptyset$ still holds.
This is due to the addition of a vacuous action, $a_\emptyset$, which is not strongly executable in any state.
The conditions~\ref{item:khiml-fil-a1} and~\ref{item:khiml-fil-a2} are the usual ones for the basic modal logic.
In fact, notice that they are not in conflict with the new conditions, as the actions used to talk about formulas with $\kh$ are disjoint from those to talk about formulas with $[a]$ modalities.
So, new actions $a_{[\plans]_\Sigma}$ are introduced as a result of the witness for a knowing how, but as they are fresh actions, they do not impact on any formula from $\Sigma$.
Finally, \Cref{item:khiml-fil} can be rewritten as $([w]_\Sigma,[v]_\Sigma) \in \R^f_{a_{[\plans]_\Sigma}}$ if and only if the following conditions hold:

\begin{itemize}
\item $[w]_\Sigma \subseteq \bigcap_{\plans' \in G_{[\plans]_\Sigma}} \stexec(\plans')$, and
\item $(w'',v'') \in \bigcup_{\plans'' \in G_{[\plans]_\Sigma}} \R_{\plans''}$ for some $w'' \in [w]_\Sigma$ and $v'' \in [v]_\Sigma$.
\end{itemize}

Moreover, $[w]_\Sigma \in \stexec(a_{[\plans]_\Sigma})$ iff $[w]_\Sigma \subseteq \bigcap_{\plans' \in G_{[\plans]_\Sigma}} \stexec(\plans')$.
The following example illustrates how filtrations preserve relevant information for evaluating formulas from the input set of formulas $\Sigma$.

\medskip

\begin{example}\label{ex:afiltration}
Let $\modults = \tup{\W,\R,\Unc,\V}$ be an \ults defined over $\AGT=\set{k,j}$ and $\ACT = \setof{a_i,b_i}{i \in \natnum}$ as: 
\begin{enumerate}
\item $\W =
\setof{w^i_1}{i \in \natnum} \cup
\setof{w^i_2}{i \in \natnum} \cup
\setof{v^i}{i \in \natnum}$;
\item
$\R_{a_i} = \setof{(w^i_1,w^j_2)}{j \in \natnum}$,
$\R_{b_i} = \set{(w^i_2,v^i)}$;
\item
$\Unc(k) = \set{\plans_1,\plans_2,\plans_3}$, where  $\plans_1=\set{a_1b_1}$, $\plans_2=\set{a_1}$, and $\plans_3= \setof{a_i b_i}{i \in \natnum \setminus \set{1}}$, and
$\Unc(j) = \setof{\set{a_i}}{i \in \natnum \setminus \set{1}}$;
\item $\V(w^i_1) = \set{p}$, $\V(w^i_2) = \set{r}$, $\V(v^i) = \set{q,s^i}$, for all $i \in \natnum$.
\end{enumerate}

\medskip

The model just defined is depicted below:

\begin{center}
\begin{tikzpicture}[->]
\node [state, label=above:\small$w^1_1$] (w11) {$p$};
\node [state, label=above:$w^1_2$, right = 1.8cm of w11] (w12) {$r$};
\node [state, label=above:\small$v^1$, right = 1.8cm of w12] (v1) {$q,s^1$};
\node [state, label=below:\small$w^2_1$, below = 1.5cm of w11] (w21) {$p$};
\node [state, label=below:\small$w^2_2$, right = 1.8cm of w21] (w22) {$r$};
\node [state, label=below:\small$v^2$, right = 1.8cm of w22] (v2) {$q,s^2$};
\node [below = 0.35cm of w21] (w31) {$\vdots$};
\node [right = 1.85cm of w31] (w32) {$\vdots$};
\node [right = 2cm of w32] (v3) {$\vdots$};

\path (w11) edge[right] node [label-edge, above] {$a_1$} (w12);
\path (w11) edge[right] node [label-edge, right] {$a_2$} (w22);
\path (w12) edge[right] node [label-edge, above] {$b_1$} (v1);

\path (w21) edge[right] node [label-edge, above] {$a_2$} (w22);
\path (w21) edge[right] node [label-edge, left] {$a_1$} (w12);
\path (w22) edge[right] node [label-edge, above] {$b_2$} (v2);
\end{tikzpicture}
\end{center}

Let $\Sigma = \set{\kh_k(p,q),\kh_k(p,r),\neg\kh_j(p,q),\kh_j(p,r),p,q,r,\mlbox{b_1} s^1,s^1}$ a set of formulas closed by subformulas.
The respective equivalence classes for the plans in $\Unc_\AGT$ are defined as:

\begin{itemize}
    \item $[\plans_1]_\Sigma = [\plans_3]_\Sigma = \set{\plans_1,\plans_3}$,
    \item $[\plans_2]_\Sigma = [\set{a_1}]_\Sigma= \set{\set{a_1}}$, and 
    \item $[\set{a_2}]_\Sigma = [\plans_4]_\Sigma = \setof{\set{a_i}}{i \in \natnum\setminus\set{1}}$.
\end{itemize}

From now on, $\set{a_2}$ will be denoted as $\plans_4$ as we did above. 
Notice that $[\plans_1]_\Sigma$ encompasses the set of plans that are witnesses for $\kh_k(p,q)$, while $[\plans_2]_\Sigma$ encompasses the witnesses for $\kh_k(p,r)$. For the case of $\kh_j(p,q)$ there is no witness under consideration, as the formula does not hold at $\modults$.   
Finally, $[\plans_4]_\Sigma$ encompassess the witnesses for $\kh_j(p,r)$. Thus, we have $\ACT^\Sigma =\set{a_\emptyset,a_{[\plans_1]_\Sigma},a_{[\plans_2]_\Sigma},a_{[\plans_4]_\Sigma},b_1}$.

A possible filtration would be $\modults^f = \tup{\W^f,\R^f,\Unc^f,\V^f}$, with: 
\begin{enumerate}
\item $\W^f := \set{[w^1_1]_\Sigma,[w^1_2]_\Sigma,[v^1]_\Sigma,[v^2]_\Sigma}$, where:
    \begin{itemize}
        \item $[w^1_1]_\Sigma=\setof{w^i_1}{i \in \natnum}$,
        \item $[w^1_2]_\Sigma=\setof{w^i_2}{i \in \natnum}$, 
        \item $[v^1]_\Sigma=\set{v^1}$, 
        and 
        \item $[v^2]_\Sigma=\setof{v^i}{i \in \natnum \setminus \set{1}}$.
    \end{itemize}
\item $\V^f([w^1_1]_\Sigma) = \set{p}$, $\V([w^1_2]_\Sigma) = \set{r}$, $\V([v^1]_\Sigma) = \set{q,s^1}$ and $\V([v^2]_\Sigma) = \set{q}$.
\item
$\Unc^f(k) = \set{\set{a_{[\plans_1]_\Sigma}}, \set{a_{[\plans_2]_\Sigma}}, \set{a_\emptyset}}$, and 
$\Unc^f(j) = \set{\set{a_{[\plans_4]_\Sigma}}, \set{a_\emptyset}}$;
% $\Unc^f(k) = \set{\set{a_\emptyset}}$;
\item $\R^f$ is defined as follows for each $a \in \ACT^\Sigma$:
\begin{itemize}
\item $\R^f_{a_\emptyset} = \emptyset$,
\item $\R^f_{a_{[\plans_1]_\Sigma}} = \setof{([w^1_1]_\Sigma,[v^i]_\Sigma)}{i \in \natnum}$ (if $G_{[\plans_1]_\Sigma} = [\plans_1]_\Sigma$),
\item $\R^f_{a_{[\plans_2]_\Sigma}} = \set{([w^1_1]_\Sigma,[w^1_2]_\Sigma)}$ (if $G_{[\plans_2]_\Sigma} = [\plans_2]_\Sigma$),
\item $\R^f_{a_{[\plans_4]_\Sigma}} = \set{([w^1_1]_\Sigma,[w^1_2]_\Sigma)}$ (if $G_{[\plans_4]_\Sigma} = [\plans_4]_\Sigma$), and 
\item $\R^f_{b_1} = \set{([w^1_2]_\Sigma,[v^1]_\Sigma)}$. % (given $\ACT^\Sigma_\square = \set{b_1}$)
\end{itemize}
\end{enumerate}

% \medskip

The filtration $\modults^f$ is depicted below:

\begin{center}
\begin{tikzpicture}[->]
\node [state, label=above:\small{$[w^1_1]_\Sigma$}] (w11) {$p$};
\node [state, label=above:\small{$[w^1_2]_\Sigma$}, right = 1.8cm of w11] (w12) {$r$};
\node [state, label=above:\small{$[v^1]_\Sigma$}, right = 1.8cm of w12] (v1) {$q,s^1$};
\node [state, label=above:\small{$[v^2]_\Sigma$}, below = 1.5cm of v1] (v2) {$q$};
% \node [below = 0.1cm of v2] (v3) {$\vdots$};
% \node [below = 1.8cm of w12] (w32) {$\vdots$};

\path (w11) edge[right] node [label-edge, above] {$a_{[\plans_2]_\Sigma}$} (w12);
\path (w11) edge[right] node [label-edge, below] {$a_{[\plans_4]_\Sigma}$} (w12);
\path (w11) edge[below, bend right=40] node [label-edge, above] {$a_{[\plans_1]_\Sigma}$} (v1);
\path (w11) edge[right, bend right=40] node [label-edge, right] {$a_{[\plans_1]_\Sigma}$} (v2);
\path (w12) edge[right] node [label-edge, above] {$b_1$} (v1);
\end{tikzpicture}
\end{center}

If, however, we consider $G_{[\plans_1]_\Sigma} = \set{\plans_1}$, then $\R^f_{a_{[\plans_1]_\Sigma}} = \set{([w^1_1]_\Sigma,[v^1]_\Sigma)}$, as $(w^1_1,v^1)$ is the only edge that holds the two conditions. Thus, we can observe how there are at least two alternatives for defining a filtration from $\modults$ and $\Sigma$.
\end{example}

\medskip

We then proceed to prove that, given a model and its filtration via $\Sigma$, they both satisfy the same formulas in $\Sigma$.

\medskip

\begin{theorem}\label{th:filtration}
Let $\modults = \tup{\W,\R,\Unc,\V}$ be an \ults and
let $\Sigma$ be a set of $\KHiMLlogic$-formulas that is closed under subformulas.
Then, for all $\psi \in \Sigma$ and $w \in \W$, 
\[
    \modults,w\models \psi \ \mbox{  iff  } \ \modults^f,[w]_\Sigma\models \psi.
\] 
Moreover, if $\Sigma$ is finite then $\modults^f$ is a finite model.
\end{theorem}

\begin{proof}
Boolean cases work as expected. We will proceed to prove the cases for the modalities $\khi$ and $\mlbox{a}$.

\begin{itemize}
\item \textbf{Case $\khi(\psi,\varphi)$}:

$(\Rightarrow)$ Suppose $\modults \models \khi(\psi,\varphi)$, and let $\plans \in \Unc(i)$ be one of its witnesses.
By~\Cref{def:filtration-actions}, $a_{[\plans]_\Sigma} \in \ACT^\Sigma_i$ and thus, $\set{a_{[\plans]_\Sigma}} \in \Unc^f(i)$. It is enough to prove that this is a witness for $ \khi(\psi,\varphi)$ in $\modults^f$. 
By \Cref{item:khiml-fil} of~\Cref{def:filtrations}, for all $\plans' \in G_{[\plans]_\Sigma}$, $\plans \planequiv_\Sigma \plans'$.
Then, for all $\plans' \in G_{[\plans]_\Sigma}$, $\truthset{\modults}{\psi} \subseteq \stexec(\plans')$ and $\R_{\plans'}(\truthset{\modults}{\psi}) \subseteq \truthset{\modults}{\varphi}$.
    \begin{enumerate}
        \item Let $[w]_\Sigma \in \truthset{\modults^f}{\psi}$. By inductive hypothesis, for all $w' \in [w]_\Sigma = [w']_\Sigma$, $w' \in \truthset{\modults}{\psi}$,  thus $[w]_\Sigma \subseteq \stexec(\plans')$, for all $\plans' \in G_{[\plans]_\Sigma}$.
        Then, $[w]_\Sigma \in \stexec(a_{[\plans]_\Sigma})$. Therefore, $\truthset{\modults^f}{\psi} \subseteq \stexec(\set{a_{[\plans]_\Sigma}})$.

        \item Let $([w]_\Sigma,[v]_\Sigma) \in \R^f_{a_{[\plans]_\Sigma}}$ be such that $[w]_\Sigma \in \truthset{\modults^f}{\psi}$.
        Then, for some $w' \in [w]_\Sigma = [w']_\Sigma$, $v' \in [v]_\Sigma = [v']_\Sigma$ and $\plans' \in G_{[\plans]_\Sigma}$ we have $(w',v') \in \R_{\plans'}$.
        By inductive hypothesis, $w' \in \truthset{\modults}{\psi}$ and thus $v' \in \truthset{\modults}{\varphi}$.
        Again, by inductive hypothesis, $[v']_\Sigma = [v]_\Sigma \in \truthset{\modults^f}{\varphi}$.
        Then, $\R^f_{\set{a_{[\plans]_\Sigma}}}(\truthset{\modults^f}{\psi}) \subseteq \truthset{\modults^f}{\varphi}$.
    \end{enumerate}

A a consequence, we get $\modults^f \models \khi(\psi,\varphi)$. \\

$(\Leftarrow)$ Suppose $\modults^f \models \khi(\psi,\varphi)$, and let $\plans \in \Unc^f(i)$ be one of its witnesses. 
If $\plans = \set{a_\emptyset}$, then $\truthset{\modults^f}{\psi} \subseteq \stexec(\set{a_\emptyset}) = \emptyset$.
By inductive hypothesis, $\truthset{\modults}{\psi} = \emptyset$ and any $\plans \in \Unc(i)$ is a witness in $\modults$ for $\khi(\psi,\varphi)$. 
If $\plans = \set{a_{[\plans']_{\Sigma}}}$, then $\truthset{\modults^f}{\psi} \subseteq \stexec(\set{a_{[\plans']_\Sigma}})$ and $\R^f_{\set{a_{[\plans']_\Sigma}}}(\truthset{\modults^f}{\psi}) \subseteq \truthset{\modults^f}{\varphi}$.
It is enough to prove that any $\plans'' \in G_{[\plans']_\Sigma}$ does the job.

    \begin{enumerate}
        \item Let $w \in \truthset{\modults}{\psi}$, by inductive hypothesis, $[w]_\Sigma \in \truthset{\modults^f}{\psi}$ and consequently $[w]_\Sigma \in \stexec(\set{a_{[\plans']_\Sigma}})$.
        By \Cref{item:khiml-fil} in~\Cref{def:filtrations}, we have that $[w]_\Sigma \subseteq \bigcap_{\plans'' \in G_{[\plans']_\Sigma}} \stexec(\plans'')$, and naturally that $w \in \bigcap_{\plans'' \in G_{[\plans']_\Sigma}} \stexec(\plans'')$.
        Thus, $\truthset{\modults}{\psi} \subseteq \stexec(\plans'')$, for all $\plans'' \in G_{[\plans']_\Sigma}$.

        \item Let $(w,v) \in \R_{\plans''}$ such that $w \in \truthset{\modults}{\psi}$.
        Since $w \in \bigcap_{\plans'' \in G_{[\plans']_\Sigma}} \stexec(\plans'')$, necessarily we have $([w]_\Sigma,[v]_\Sigma) \in \R^f_{a_{[\plans']_\Sigma}}$.
        By inductive hypothesis, $[w]_\Sigma \in \truthset{\modults^f}{\psi}$ and with this $[v]_\Sigma \in \truthset{\modults^f}{\varphi}$.
        Again, by inductive hypothesis, $v \in \truthset{\modults}{\varphi}$. Then, $\R_{\plans''}(\truthset{\modults}{\psi}) \subseteq \truthset{\modults}{\varphi}$, for any $\plans'' \in G_{[\plans']_\Sigma}$.
    \end{enumerate}

Therefore, $\modults \models \khi(\psi,\varphi)$. 

\item \textbf{Case $\mlbox{a}\varphi$}: $(\Rightarrow)$ Suppose $\modults,w \models \mlbox{a}\varphi$.
Let $([w]_\Sigma,[v]_\Sigma) \in \R^f_a$, since $\mlbox{a}\varphi \in \Sigma$, then $\modults,v \models \varphi$.
By IH, $\modults^f,[v]_\Sigma \models \varphi$.
Hence, $\modults^f,[w]_\Sigma \models \mlbox{a}\varphi$.

$(\Leftarrow)$ Suppose $\modults^f,[w]_\Sigma \models \mlbox{a}\varphi$.
Let $(w,v) \in \R_a$, then $([w]_\Sigma,[v]_\Sigma) \in \R^f_a$ and $\modults^f,[v]_\Sigma \models \varphi$.
By IH, $\modults,v \models \varphi$.
Hence, $\modults,w \models \mlbox{a}\varphi$.
\end{itemize}

It remains to show that if $\Sigma$ is finite, then $\modults^f$ is finite.
First, note that the number of elements in $\W^f$ is at most exponential in the number of formulas in $\Sigma$.
Thus, $\W^f$ and $\V^f$ are finite. Similarly, as the number of $\khi(\psi,\varphi)$ and $\mlbox{a} \varphi$ in $\Sigma$ are finite, $\ACT^{\Sigma}$ is also finite.
Thus, $\R^f$ and $\Unc^f$ are both finite, and therefore $\modults^f$ is a finite model.
\end{proof}

\medskip 

\begin{example}\label{ex:filtration-preserves}
 Consider again~\Cref{ex:afiltration}. Notice that the filtration preserves the satisfiability of formulas in $\Sigma$. For instance, $\modults\models\kh_k(p,q)$ with $\plans_1$ as witness, while $\modults^f\models\kh_k(p,q)$ with $\set{a_{[\plans_1]_\Sigma}}$ as witness. Also, it is clear that both $\modults\not\models\kh_j(p,q)$ and $\modults^f\not\models\kh_j(p,q)$, and that $\modults,w^1_2\models[b_1] s^1$ and $\modults^f,[w^1_2]_\Sigma\models[b_1] s^1$. One can verify that this is also the case for other alternative filtrations, as established by~\Cref{th:filtration}.
\end{example}

\medskip

Considering~\Cref{th:filtration}, it remains to prove that, given a model $\modults$ and a set of formulas $\Sigma$ closed under subformulas, the set of filtrations of $\modults$ via $\Sigma$ is non empty.
This can be achieved by defining $G_{[\plans]_\Sigma} = [\plans]_\Sigma \neq \emptyset$ for each $a_{[\plans]_\Sigma} \in \ACT^\Sigma_i$, and considering at least one of the two alternatives to define $\R_a$ for each $a \in \ACT^\Sigma_\square$ given in \cite{HML,mlbook}:

\begin{itemize}
\item $([w]_\Sigma,[v]_\Sigma) \in \R^f_a$ iff there exists $w' \in [w]_\Sigma$, $v' \in [v]_\Sigma$ such that $(w',v') \in \R_a$; or
\item $([w]_\Sigma,[v]_\Sigma) \in \R^f_a$ iff for all $\mlbox{a}\varphi \in \Sigma$, $\modults,w \models \mlbox{a}\varphi$ then $\modults,v \models \varphi$.
\end{itemize}

Notice that both the different alternatives of picking $G_{[\plans]_\Sigma}$ and $\R^f_a$ enable us to obtain more than one filtration, all of them satisfying the intended properties.

\medskip

The last theorem states that every satisfiable formula of $\KHiMLlogic$, is satisfiable in a finite, bounded model.
As a consequence, the satisfiability problem for $\KHiMLlogic$ is decidable.

\medskip

\begin{corollary}\label{cor:extended-decidable}
The satisfiability problem for $\KHiMLlogic$ is decidable.
\end{corollary}

\begin{proof}
Suppose $\varphi$ is satisfiable, then there exists a model $\modults$ and a state $w$ such that $\modults,w\models\varphi$. Let $\Sigma$ be the set of subformulas of $\varphi$ (including $\varphi$).
Then, there exists a filtration $\modults^f$ such that, by~\Cref{th:filtration}, $\modults^f,[w]_\Sigma\models\varphi$. Since $\Sigma$ is finite, $\modults^f$ is finite, and of size at most exponential.
Consequently, we know there exists a finite model of $\varphi$. Thus, we can list all the models and try one by one, whether they satisfy $\varphi$ or not, by calling a model-checking algorithm.
Since model-checking is decidable (by simply combining the algorithm for basic modal logic~\cite{mlbook}, and the one for $\KHilogic$~\cite{AFSVQ21,AFSVQ23report}), the corollary follows. \raul{No tenemos demo especifica de que model-checking es decidible, dejamos asi como esta?}
\end{proof}

\medskip

It is worth noticing that the conditions regarding the treatment of $\Unc^f(i)$ in~\Cref{def:filtrations} are slightly different from the conditions introduced in~\cite{AFSVQ23report} for the basic logic $\KHilogic$. Therein, the existence of a filtration is not established. Thus, in this work we update the definition of filtrations, making easier to guarantee that we can always obtain a finite model from any set of finite formulas, and consequently~\Cref{cor:extended-decidable}.

% The rest of the section will be devoted to introduce a new family of dynamic epistemic logics extending $\KHiMLlogic$.
