\subsection{Atomic action refinement}
\label{subsec:atom-ref}


We start by introducing an operator inspired by the refinement modality
from~\Cref{sec:ref}. Herein, we introduce the modality $\srefbox{a}$, 
which publicly establishes (i.e., for all agents) that action $a$ must be distinguished from any other. Unlike the modality $\refbox{\plan_1}{\plan_2}$, this modality takes atomic actions (although it will be generalized later to arbitrary plans), and has a single parameter, instead of two parameters. So, with the new operator we isolate the effect of one action, rather that separating a class according to the effects of two actions. This is useful as for instance, an uninformed agent may receive a concrete information about the actual effect of a particular action, thus this action becomes distinguishable from any other. These ideas are formalized below.




\begin{definition}\label{def:ssrefsyntax}
Formulas of the language $\AtomReflogic$ are given by
\[
\varphi ::= p \mid \neg\varphi \mid \varphi\vee\varphi \mid \khi(\varphi,\varphi) \mid \mlbox{a}\varphi \mid \srefbox{a}\varphi,
\]
with $p \in \PROP$, $i\in\AGT$ and $a \in \ACT$. Formulas of the form $\srefbox{a}\varphi$ are read as: \emph{``after announcing that $a$ is distinguishable from any other plan, $\varphi$ holds''.} 
\end{definition}

\begin{definition}\label{def:ssrefsemantics}
Let $\modults = \tup{\W, \R, \Unc, \V}$ be an \ults, $a \in \ACT^*$ and  be a $\PlanReflogic$-formula,
\[
\modults,w \models \srefbox{a}\varphi \ \ \iff \ \ \modults^a,w \models \varphi,
\] 
where $\modults^a = \tup{\W,\R,\Unc',\V}$, with:
\begin{itemize}
\item $\Unc'(i) = (\Unc(i) \setminus \set{\plans}) \cup \set{\set{a}}$ if there is $\plans \in \Unc(i)$ such that $a \in \plans$,
\item $\Unc'(i) = \Unc(i) \cup \set{\set{a}}$ otherwise.
\end{itemize}
\end{definition}

In a few words, the new dynamic modality is an announcement of that a given, single action, has a behaviour that can be uniquely determined. In case the action is not a part of the actions that are under the consideration of a certain agent, it is added to it, making the agent aware of its existence. 
The following proposition states that the new modality is self-dual and normal.

\begin{proposition} The following propositions hold:
\begin{enumerate}
\item $\models \srefbox{a}\varphi \lra \neg\srefbox{a}\neg\varphi$. 
\item $\models \srefbox{a}(\varphi_1 \ra \varphi_2) \ra (\srefbox{a}\varphi_1 \ra \srefbox{a}\varphi_2)$.
\item If $\models \varphi$, then $\models \srefbox{a}\varphi$.
\end{enumerate}
\end{proposition}

The proof of the proposition above is straightforward. Let us introduce now an example of use of the new modality.

\begin{example}
Let $\modults$ be the single agent model depicted below, with $\Unc(i):=\set{\set{a}}$. It is clear that $\modults,w\not\models\khi(p,q)$. But we have that $\modults,w\models[!b]\khi(p,q)$, as after announcing $b$, the agent becomes aware of the existence of that plan, and she can use it to guarantee $q$ given~$p$. 
\begin{center}
\scalebox{.9}{
\begin{tikzpicture}[->, grow' = right, level/.style={sibling distance = 2em/#1}, level distance = 1.5em]
\node[state] at (0,1) (p) [label=left:$w$]{$p$};
\node[left = of p] (m) {$\modults$};
\node[state] at (1.75,0.6) (nq) {{$q$}};
\node[state] at (1.75,1.4) (q) {\phantom{$p$}};
\path (p) edge [above] node {$a$} (q);
\path (p) edge [below] node {$b$} (nq);
\end{tikzpicture} }
\end{center} 
\end{example}

\subsubsection{Axiom System}

We proceed now to introduce a complete axiom system for the logic $\AtomReflogic$. As we have previously mentioned, dealing with dynamic modalities is not easy. Fortunately, here the expressive power of the underlying static logic ($\KHiMLlogic$) is sufficient to capture the behaviour of $\srefbox{a}$, hence obtaining an axiomatization via reduction axioms (\Cref{tab:ssrefaxiom}).

The proof using the reduction axioms consists in translating every formula of $\AtomReflogic$ into a formula of $\KHiMLlogic$. This is achieved by induction on the structure of $\AtomReflogic$ formulas. The challenge is, as usual, to eliminate the occurence of a $\srefbox{a}$ modality in presence of $\khi$. Consider a formula $\srefbox{a}\khi(\psi,\varphi)$. After the ``announcement'' of action $a$ being different to any other plan, there are two possible reasons of why $\khi(\psi,\varphi)$ holds. The first possibility, is that action $a$ plays no role in the truth of $\khi(\psi,\varphi)$. In this case, we can push the dynamic modality into the front of the pre and post-conditions of the $\khi$ modality, i.e., we obtain $\khi(\srefbox{a}\psi,\srefbox{a}\varphi)$. The second possibility is that $a$ is crucial in order to raise new knowledge for the agent. If the singleton $\set{a}$ is the witness for $\khi(\psi,\varphi)$ is because: 1) $a$ is SE at every state satisfying $\psi$ after the annoucement of $a$ (notice that as $a$ is a single action, SE is equivalent to executability); and 2) from every $\psi$-state, after every execution of $a$ we always get that $\varphi$ holds (in the model updated by $\srefbox{a}$). The former is captured in formula by $\srefbox{a}\psi\ra\mldiam{a}\top$ holding everywhere, while the latter is captured by $\mlbox{a}\srefbox{a}\varphi$. Putting all together, this case is reflected by $\A(\srefbox{a}\psi\ra(\mldiam{a}\top\wedge\mlbox{a}\srefbox{a}\varphi))$. The other cases are standard reductions, so we obtain the axioms shown in~\Cref{tab:ssrefaxiom}.

\begin{table}[t]
\begin{tabular}{l@{\quad}l}
\toprule
\axm{RAtom} & $\vdash \srefbox{a}p \lra p$ \\
\axm{R$\neg$} & $\vdash \srefbox{a} \neg \varphi_1 \lra \neg \srefbox{a} \varphi_1$ \\
\axm{R$\vee$} & $\vdash \srefbox{a} (\varphi_1 \vee \varphi_2) \lra (\srefbox{a} \varphi_1 \vee \srefbox{a} \varphi_2)$ \\
\axm{R$\square$} & $\vdash \srefbox{a}\mlbox{a}\varphi_1 \lra \mlbox{a}\srefbox{a}\varphi_1$ \\
\axm{RKh} & $\vdash \srefbox{a} \khi(\varphi_1,\varphi_2) \lra (\khi(\srefbox{a} \varphi_1,\srefbox{a} \varphi_2) \vee \A(\srefbox{a} \varphi_1 \ra (\mldiam{a}\top \wedge \mlbox{a} \srefbox{a}\varphi_2)))$ \\
\axm{RE$_{\srefbox{}}$} & $\text{From } \vdash \varphi \leftrightarrow \psi \text{ derive } \vdash \srefbox{a}\varphi \leftrightarrow \srefbox{a}\psi$ \\
\bottomrule
\end{tabular}
\caption{Reduction axioms $\axset_{\AtomReflogic}$.}\label{tab:ssrefaxiom}
\end{table}

\begin{proposition}\label{pro:ssrefaxioms} The reduction axioms from~\Cref{tab:ssrefaxiom} are valid.
% \begin{enumerate}
% \item $\models \srefbox{a}p \lra p$
% \item $\models \srefbox{a} \neg \varphi_1 \lra \neg \srefbox{a} \varphi_1$
% \item $\models \srefbox{a} (\varphi_1 \vee \varphi_2) \lra (\srefbox{a} \varphi_1 \vee \srefbox{a} \varphi_2)$
% \item $\models \srefbox{a}\mlbox{a}\varphi_1 \lra \mlbox{a}\srefbox{a}\varphi_1$
% \item $\models \srefbox{a} \khi(\varphi_1,\varphi_2) \lra (\khi(\srefbox{a} \varphi_1,\srefbox{a} \varphi_2) \vee \A(\srefbox{a} \varphi_1 \ra (\mldiam{a}\top \wedge \mlbox{a} \srefbox{a}\varphi_2)))$
% \end{enumerate}
\end{proposition}
\begin{proof}
% The first and fourth items are direct since $\V$ and $\R_a$ remain unchanged throughout the evaluation. The second and third items are straightforward by using the definition of $\srefbox{a}$.
We will only show that the axiom $\axm{RKh}$ is valid (the other can be easily checked). 
Let $\modults = \tup{\W, \R, \Unc, \V}$ be an \ults.

$\Rightarrow):$ $\modults \models \srefbox{a} \khi(\varphi_1,\varphi_2)$ iff $\modults^a \models \khi(\varphi_1,\varphi_2)$ iff there is $\plans' \in \Unc'(i)$ s.t. $\truthset{\modults^a}{\varphi_1} \subseteq \stexec^{\modults^a}(\plans')$ and $\R_{\plans'}(\truthset{\modults^a}{\varphi_1}) \subseteq \truthset{\modults^a}{\varphi_2}$.
Using $\truthset{\modults^a}{\psi} = \truthset{\modults}{\srefbox{a} \psi}$ and $\stexec^{\modults^a}(\plans') = \stexec^\modults(\plans')$ (since the relations remain unchanged), this is equivalent to that there is $\plans' \in \Unc'(i)$ s.t. $\truthset{\modults}{\srefbox{a} \varphi_1} \subseteq \stexec^\modults(\plans')$ and $\R_{\plans'}(\truthset{\modults}{\srefbox{a} \varphi_1}) \subseteq \truthset{\modults}{\srefbox{a} \varphi_2}$. We need to consider two cases:
\begin{itemize}
\item If $\plans' \neq \set{a}$, then $\plans' \in \Unc(i)$ and we get $\modults \models \khi(\srefbox{a}\varphi_1,\srefbox{a}\varphi_2)$.
\item If $\plans' = \set{a}$, then $\truthset{\modults}{\srefbox{a}\varphi_1} \subseteq \stexec^\modults(a)$ and $\R_a(\truthset{\modults}{\srefbox{a}\varphi_1}) \subseteq \truthset{\modults}{\srefbox{a}\varphi_2}$.
From the first part, using the definition of SE, we can derive that for all $w \in \truthset{\modults}{\srefbox{a}\varphi_1}$, we have $\modults,w \models \mldiam{a}\top$.
From the second one, we can reach that for all $w \in \truthset{\modults}{\srefbox{a}\varphi_1}$, $\modults,w \models \mlbox{a} \srefbox{a}\varphi_2$.
Thus, for all $w \in \W$, $\modults,w \models \srefbox{a} \varphi_1 \ra (\mldiam{a}\top \wedge \mlbox{a} \srefbox{a}\varphi_2)$.
\end{itemize}

$\Leftarrow):$ Conversely, suppose $\modults \models \khi(\srefbox{a} \varphi_1,\srefbox{a} \varphi_2) \vee \A(\srefbox{a} \varphi_1 \ra (\mldiam{a}\top \wedge \mlbox{a} \srefbox{a}\varphi_2))$, then the disjunction is satified by either one of the following situations:
\begin{itemize}
    \item $\modults \models \khi(\srefbox{a} \varphi_1,\srefbox{a} \varphi_2)$: Let $\plans \in \Unc(i)$ be s.t. $\truthset{\modults}{\srefbox{a}\varphi_1} \subseteq \stexec^\modults(\plans)$ and $\R_\plans(\truthset{\modults}{\srefbox{a}\varphi_1}) \subseteq \truthset{\modults}{\srefbox{a}\varphi_2}$.
    As stated before, this is equivalent to $\truthset{\modults^a}{\varphi_1} \subseteq \stexec^{\modults^a}(\plans)$ and $\R_\plans(\truthset{\modults^a}{\varphi_1}) \subseteq \truthset{\modults^a}{\varphi_2}$.
    If $a \notin \plans$, then $\plans \in \Unc'(i)$.
    If $a \in \plans$, then $\set{a} \in \Unc'(i)$, $\stexec^{\modults^a}(\plans) \subseteq \stexec^{\modults^a}(\set{a})$ and $\R_{\set{a}}(\truthset{\modults^a}{\varphi_1}) \subseteq \R_\plans(\truthset{\modults^a}{\varphi_1})$. In any case, we get $\modults \models \srefbox{a}\khi(\varphi_1,\varphi_2)$.

    \item $\modults \models \A(\srefbox{a} \varphi_1 \ra (\mldiam{a}\top \wedge \mlbox{a} \srefbox{a}\varphi_2))$: Notice that $a$ meets the strong executability and reachability conditions.
    The only thing left to prove is that $\set{a} \in \Unc'(i)$. But that is direct since wheter $a \in \plans$ for some $\plans \in \Unc(i)$ or not, in the end $\set{a} \in \Unc'(i)$. Hence $\modults \models \srefbox{a}\khi(\varphi_1,\varphi_2)$.
\end{itemize}
\end{proof}

With the correctness of the reductions axioms from~\Cref{tab:ssrefaxiom} at hand, and using \Cref{th:cm-ults-khikt-completeness}, the following result easily follows.

\begin{theorem}\label{th:sscomplete}
The axiom system $\axset_{\AtomReflogic}+\axset_{\khi,\square}$ for $\AtomReflogic$ is sound and strongly complete w.r.t. the class of all models.
\end{theorem}

Finally, as the elimination of the dynamic modality takes a finite amount of steps, by using also~\Cref{cor:extended-decidable} we can conclude:

\begin{corollary}
$\AtomReflogic$ is decidable.
\end{corollary} \raul{ver si dejamos este claim}
