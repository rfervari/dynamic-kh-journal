So far, we discussed some interesting technical properties of the logic $\KHilogic$. However, there are some other ideas worth discussing from a conceptual perspective. In particular, the rest of the paper is devoted to explore different ways in which a \emph{dynamic} operation can be added to $\KHilogic$, with the goal of updating an agent's know how.   For example, one of the better known dynamic epistemic logic is \emph{Public Announcement Logic} (\PAL)~\cite{Plaza89:lopc}.  \PAL extends basic epistemic logic with an operator for public announcements which, semantically, updates a model by removing the worlds that do not satisfy the announced formula. In standard epistemic logic, this corresponds to an act of \emph{epistemic communication}: it is publicly stated that a formula $\varphi$ is the case, thus worlds stating the opposite are not longer considered as possible and deleted from the model.  Can we define similar operators in $\KHilogic$?

We can consider a dynamic operator as the indication to perform an update on a model, so that the evaluation of the formula should continue in the modified model.  In an $\ults$, these model transformations can be interpreted as actions that affect either the agents' abilities, or her epistemic state. 
Perhaps it is the world that changes and the agent realizes this change and adjust her epistemic state accordingly (the \emph{belief update} of the belief change literature~\cite{sep-logic-belief-revision}); or she receives new information about the world while the world remains the same (the \emph{belief revision} of the belief change literature). The former can be called \emph{ontic} change, whereas the latter can be called \emph{epistemic} change. In DEL, for example, the first can be represented by a change in valuation, while the second can be represented by changes in the agents' uncertainty~\cite{vanDitmarschKooi2008}.

In an \ults $\modults=\tup{\W,\R,\set{\Unc(i)}_{i\in\AGT},\V}$, there is a clear distinction between ontic and epistemic information. On the one hand, the accessibility relation $\R$ provides \emph{ontic}, \emph{objective} information indicating what the actions themselves can achieve, while the valuation $\V$ describes the actual propositions being true at each state. On the other hand, the \emph{epistemic} state of an agent $i$ (with respect to her \emph{knowing how} capabilities) is given by her indistinguishability relation over plans (the set $\Unc(i)$ at her disposal). 

%In the rest of this article we will consider both ontic and epistemic updates. Some of these types of updates have been studied in~\cite{AFSV22}. In this article, we introduce a novel, more general approach, in which we enrich the logic with modalities that express explicitely that a given action can be executed. Notice that in previous works, the basic language of knowing how lacks the ability of talking explicitely about the actions. With this new feature at hand, we will be able to obtain completeness results for a dynamic language via reduction axioms.
