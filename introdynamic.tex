In the rest of the paper we will explore different ways in which a \emph{dynamic} operation can be added to $\KHilogic$, aiming for updating an agent's know how.   We can consider a dynamic operator as the indication of performing an update on a model, so that the evaluation of the formula should continue in the modifed model.  Some of these model transformations can be interpreted as actions that affect the agents' abilities or her epistemic state. We will explore some of these alternatives.

For example, one of the better known dynamic epistemic logic is \emph{Public Announcement Logic} (\PAL)~\cite{Plaza89:lopc}.  \PAL extends basic epistemic logic with an operator for public announcements, which, semantically, updates a model by removing the worlds that do not satisfy the announced formula. In standard epistemic logic, this corresponds to an act of \emph{epistemic communication}: it is publicly stated that a formula $\varphi$ is the case, thus worlds stating the opposite are not longer considered as \emph{possible}.

However, there are at least two ways in which an agent's information might change. It might change because the world changes and she observes this (the \emph{belief update} of the belief change literature; \cite{sep-logic-belief-revision}), and it might change because she receives information about the world while the world remains the same (the \emph{belief revision} of the belief change literature; \cite{sep-logic-belief-revision}). The former can be called \emph{ontic} change, whereas the latter can be called \emph{epistemic} change. Within dynamic epistemic logic, the first can be represented by a change in valuation, while the second can be represented by changes in the agents' uncertainty \cite{vanDitmarschKooi2008}.

In an \ults $\modults=\tup{\W,\R,\set{\Unc(i)}_{i\in\AGT},\V}$, there is a clear distinction between ontic and epistemic information. On the one hand, while $\R$ provides \emph{ontic}, \emph{objective} information indicating what the actions themselves can achieve, $\V$ describes the actual propositions being true at each state. On the other hand, the \emph{epistemic} state of an agent $i$ (with respect to her \emph{knowing how} capabilities) is given by her indistinguishability relation over plans (the set $\Unc(i)$ at her disposal). Hence, in what follows we will consider both ontic and epistemic updates. Some of these types of updates have been studied in~\cite{AFSV22}. However, for epistemic updates, we introduce in this paper a novel, more general approach, in which we enrich the logic with modalities that express explicitely that a given action can be executed. Notice that in previous works, the basic language of knowing how lacks the ability of talking explicitely about the actions. With this new feature at hand, we will be able to obtain completeness results for a dynamic language via reduction axioms.
