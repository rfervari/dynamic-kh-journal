\medskip

After recalling basic definitions and useful results of the uncertainty-based semantics for \emph{knowing how}, it is time to discuss actions representing changes in the agents' abilities. The underlying idea is that of DEL: actions can be represented by model-changing operations, and their effect can be described by modalities whose semantic interpretation rely not only on the initial model, but also on the modified one. 

In an \ults $\tup{\W,\R,\Unc,\V}$ there is a clear distinction between ontic and epistemic information. On the one hand, the underlying \lts $\tup{\W,\R,\V}$ provides \emph{ontic}/\emph{objective} facts indicating what the actions themselves (and the plans derived from them) can do. On the other hand, the indistinguishability classes in $\Unc$ describe the \emph{epistemic} state of the agents, indicating which plans are available and the level to which the agents can discern among them. Thus, while changes in the underlying \lts can be seen as changes in a `dynamic' world to which the agents react by adjusting their epistemic state accordingly (analogous to  \emph{belief update} in the belief change literature~\cite{sep-logic-belief-revision} and to ontic changes in DEL~\cite{vanDitmarschKooi2008}), changes in the indistinguishability classes can be seen as changes in the agents' abilities (analogous to \emph{belief revision} in the belief change literature and to epistemic change in DEL).

The following sections explore different model-changing operations one can perform over \ultss, discussing their interpretations as well as some of their technical results. While ontic changes will be discussed in \Cref{sec:ontic} below, the main focus will be on epistemic updates (\Cref{sec:epistemic-basic} and \Cref{sec:extension}).


% pal for KH?
%For example, one of the better-known dynamic epistemic logic is \emph{Public Announcement Logic} (\PAL)~\cite{Plaza89:lopc}.  \PAL extends basic epistemic logic with an operator for public announcements which, semantically, updates a model by removing the worlds that do not satisfy the announced formula. In standard epistemic logic, this corresponds to an act of \emph{epistemic communication}: it is publicly stated that a formula $\varphi$ is the case, thus worlds stating the opposite are not longer considered as possible and deleted from the model.  Can we define similar operators in $\KHilogic$?



%In the rest of this article we will consider both ontic and epistemic updates. Some of these types of updates have been studied in~\cite{AFSV22}. In this article, we introduce a novel, more general approach, in which we enrich the logic with modalities that express explicitly that a given action can be executed. Notice that in previous works, the basic language of knowing how lacks the ability to talk explicitly about the actions. With this new feature at hand, we will be able to obtain completeness results for a dynamic language via reduction axioms.
